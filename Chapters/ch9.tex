\documentclass[class=article, crop=false]{standalone}
\usepackage[utf8]{inputenc}
\usepackage{import}
\usepackage[subpreambles=true]{standalone}
\usepackage{graphicx}
\usepackage{amsfonts}
\usepackage{mathrsfs}
\usepackage{mathtools}
\usepackage{enumerate}
\usepackage{fancyhdr}
\usepackage[colorlinks=true,linkcolor=black,anchorcolor=black,citecolor=black,filecolor=black,menucolor=black,runcolor=black,urlcolor=blue]{hyperref}
\usepackage{epsfig,amssymb,amsmath,multicol,tikz,pgfplots,amsthm,enumerate}

\def\naturals{{\mathbb N}}
\def\reals{{\mathbb R}}
\def\complex{{\mathbb C}}
\def\poly{{\mathbb P}}
\def\integers{{\mathbb Z}}
\def\rationals{{\mathbb Q}}
\def\irrationals{{\mathbb I}}
\def\inlinesum#1#2{\overset{#2}{\underset{#1}{\sum}}}
\def\inlineprod#1#2{\overset{#2}{\underset{#1}{\prod}}}
\def\ord{{\text{ord}}}
\def\ind{{\text{ind}}}


\begin{document}
\setcounter{section}{8}
\section{Primitive Roots}
%%%%%%%%%%%%%%%
%%% Section %%%
%%%%%%%%%%%%%%%
\subsection{The Order of an Integer \& Primitive Roots}
\rule{\textwidth}{1pt}\\
\begin{enumerate}
\item \textbf{Introduction:} 
The process of exponentiation and its inverse (logarithms) is as essential in modular arithmetic as it 
is in regular math and forms the basis for various encryption techniques. We begin by taking a base $a$
which is coprime to a modulus $m$ and looking at the powers of $a\mbox{ mod }m$.

\item \textbf{Order:}
Given a modulus $m$ and an integer $a$ with $\gcd(a,m)=1$ Euler's Theorem tells us that
$a^{\phi(m)} \equiv 1\mbox{ mod }m$. It does not however tell us that $\phi(m)$ is the lowest
power which yields 1. This leads to the following.
\begin{enumerate}[(a)]
	\item \textbf{Definition:} Suppose $\gcd(a,m)=1$ we define the \emph{order} of $a\mbox{ mod }m$
	as the smallest power $x$ such that $a^x \equiv 1\mbox{ mod }m$. This is denoted $\mbox{ord}_m a$. \\
	\textbf{Note:} $\mbox{ord}_m a \leq \phi(m)$ \\
	\textbf{Note:} We can say "order of $a$" when $m$ is contextually obvious.\\
	\textbf{Ex.} Let's find $\mbox{ord}_{11} 3$. Well,
	\begin{align*}
		3^1 &\equiv 3\mbox{ mod }11 \\
		3^2 &\equiv 9\mbox{ mod }11 \\
		3^3 &\equiv 5\mbox{ mod }11 \\
		3^4 &\equiv 4\mbox{ mod }11 \\
		3^5 &\equiv 1\mbox{ mod }11
	\end{align*}
	Thus, $\mbox{ord}_{11} 3 = 5$. \\
	\textbf{Note:} We can now start to see that the order gives us a pattern under which
	$3^x$ will repat!
	
	\item \textbf{Theorem:} For $x\in\integers^+$ we have $a^x \equiv 1\mbox{ mod }m$ if and only if
	$x\equiv 0\mbox{ mod } \mbox{ord}_m a$ if and only if $\mbox{ord}_m a \mid x$. \\
	\textbf{Ex.} We saw $\mbox{ord}_{11} 3=5$ so $3^x \equiv 1\mbox{ mod }11$ if and only if $x\equiv 0\mbox{ mod }5$
	if and only if $5\mid x$.
	\begin{proof}
		$ $\\
		$\rightarrow$ Assume $a^x\equiv 1 \mbox{ mod }m$, use the Divison Algorithm to write
		$x = q(\mbox{ord}_m a)+r$. Observe,
		$$1\equiv a^x\equiv \left(a^{\mbox{ord}_m a}\right)^q a^r \equiv a^r\mbox{ mod }m$$
		Since $\mbox{ord}_m a$ is the smallest positve power, we must have $r=0$.
		Thus, $x=q\mbox{ord}_m a$ so $\mbox{ord}_m a\mid x$. \\\\
		$\leftarrow$ Assume $\mbox{ord}_m a\mid x$. Then, 
		$$a^x \equiv a^{k\mbox{ord}_m a} \equiv \left(a^{\mbox{ord}_m a}\right)^k \equiv 1^k \equiv 1\mbox{ mod }m$$
	\end{proof}

	\item \textbf{Corollary:} We have $\mbox{ord}_m a \mid \phi(m)$.
	\begin{proof}
		The proof here is obvious because $a^{\phi(m)} \equiv 1\mbox{ mod }m$. Apply the theorem.
	\end{proof}
	\noindent So to find $\mbox{ord}_m a$ try divisors of $\phi(m)$ only. \\
	\textbf{Ex.} To find $\mbox{ord}_{11} 2$ we note that $\phi(11)=10$.
	So we need to check $1,2,5$ because if it fails for those, $\mbox{ord}_{11} 2 =10$.
	\begin{align*}
		2^1 &\equiv 2\not\equiv 1\mbox{ mod }11 \\
		2^2 &\equiv 4\not\equiv 1\mbox{ mod }11 \\
		2^5 &\equiv 10\not\equiv 1\mbox{ mod }11
	\end{align*}
	Aha, from this we can see that $2^{10} \equiv 1\mbox{ mod } 11$ by Euler's Theorem. So $\mbox{ord}_{11} 2 = 10$.

	\item \textbf{Theorem:} We have $a^x \equiv a^y \mbox{ mod }m$ if and only if 
	$\mbox{ord}_m a \mid (x-y)$ if and only if $x\equiv y\mbox{ mod }\mbox{ord}_m a$. 
	i.e. Exponents work mod $\mbox{ord}_m a$. \\
	\textbf{Ex.} $\mbox{ord}_{11} 3 = 5$ so $3^x \equiv 3^y \mbox{ mod } 11$ if and only if
	$x\equiv y\mbox{ mod }\mbox{ord}_{11} 3$ ($x\equiv y\mbox{ mod }5$).
	\begin{proof}
		$ $\\
		$\rightarrow$ Suppose $a^x \equiv a^y \mbox{ mod }m$ without loss of generality, assume
		$x>y$. Since $\gcd(a,m)=1$ we can cancel $a^y$ from each side to get $a^{x-y} \equiv 1\mbox{ mod } m$.
		By (b) above then $x-y\equiv 0\mbox{ mod } \mbox{ord}_m a$. \\\\
		$\leftarrow$ Suppose $x\equiv y\mbox{ mod }\mbox{ord}_m a$, then
		$x=y+k\mbox{ord}_m a$ for some $k$. Then
		$a^x\equiv a^y a^{k\mbox{ord}_m a} \equiv a^y \left(a^{\mbox{ord}_m a}\right)^k \equiv a^y \cdot 1 \equiv a^y\mbox{ mod } m$.
	\end{proof}
	\noindent\textbf{Summary Ex.}
	We saw $\mbox{ord}_{11} 3 = 5$. So $3^x$ repeats every $5^{\text{th}}$ power mod 11
	and $3^5 \equiv 1\mbox{ mod }11$.

\end{enumerate}

\item \textbf{Primitive Roots}
\begin{enumerate}
	\item \textbf{Introduction:} If $\gcd(a,m)=1$ we know that $a^{\phi(m)}\equiv 1\mbox{ mod }m$ by
	Euler's Theorem, but this may not be the smallest power. \\
	\textbf{Ex.} $\gcd(3,11)=1$ and so $3^{\phi(11)}\equiv 1\mbox{ mod }11$ so $3^{10}\equiv 1\mbox{ mod }11$,
	but in fact $3^{5}\equiv 1\mbox{ mod }11$ and $\mbox{ord}_{11} 3 = 5$ (smallet than 10). \\
	\textbf{Ex.} $\gcd(6,11)=1$ and so $6^{\phi(11)}\equiv 1\mbox{ mod }11$ so $6^{10} \equiv 1\mbox{ mod }11$
	and in fact this is the smallest. $\mbox{ord}_{11} 6=10=\phi(11)$.

	\item \textbf{Definition:} Suppose $\gcd(a,m) =1$, we say $a$ is a
	\textit{primitive root} modulus $m$ if $\mbox{ord}_m a = \phi(m)$. $a=3$ is not a primitive root mod 11,
	but $r=6$ is a primitive root mod 11. \\
	\textbf{Intuition:} Having a primitive root as a base results in more results when we raise it to powers.

	\item \textbf{Theorem:} Suppose $r$ is a primitive root mod $m$. Then $\left\{r, r^2, \cdots, r^{\phi(m)} \right\}$
	is a reduced residue set mod $m$, meaning there are $\phi(m)$ distinct items and all are coprime to $m$.
	\begin{proof}
		All are distinct because powers all distinct mod $\phi(m)=\mbox{ord}_m a$.
		All are coprime to $m$ because all are powers of $r$ and $r$ is coprime to $m$.
	\end{proof}
	\noindent\textbf{Intuition:} Given an $m$, finding a primitive root $r$ is nice because there
	will be $\phi(m)$ distinct powers of $r$ and that is the most we could have. \\
	Given an $m$, can we always find a primitive root?
	No. $m=8$ has no primitive roots, but if $m$ is prime then we can.
	If $m$ has a primitive root, might it have several? It might ...

	\item \textbf{Theorem:} Given a modulus $m$ and an integer $a$ with $\gcd(a,m)=1$ we have:
	$$\mbox{ord}_m \left(a^k\right) = \frac{\mbox{ord}_m a}{\gcd(\mbox{ord}_m a, k)}$$
	\textbf{Note:} In MATH403 this is the same result as the result from cyclic groups
	which states that if $|g|=n$ then $\left| g^k\right| = \frac{n}{\gcd(n,k)}$. \\
	\textbf{Ex.} $\mbox{ord}_{11} 6 = 10$. Look at $\mbox{ord}_{11} (6^2)$, intuitively it should be 5.
	$$\mbox{ord}_{11} (6^2) = \frac{\mbox{ord}_{11} 6}{\gcd(\mbox{ord}_{11} 6, 2)} = \frac{10}{\gcd(10, 2)} = \frac{10}{2} = 5$$
	\begin{proof}
		We'll first proof it is $\leq$ and $\geq$, thereby proving it is equal.
		\begin{itemize}
			\item First observe:
			\begin{align*}
				\left(a^k\right)^{\mbox{ord}_m a / \gcd(\mbox{ord}_m a, k)} &= \left(a^{\mbox{ord}_m a}\right)^{k / \gcd(\mbox{ord}_m a, k)} \\
				&\equiv 1 ^{k / \gcd(\mbox{ord}_m a, k)} \\
				&\equiv 1\mbox{ mod }m
			\end{align*}
			So, $$\mbox{ord}_m (a^k) \leq \frac{\mbox{ord}_m a}{\gcd(\mbox{ord}_m a, k)}$$

			\item Second observe:
			\begin{align*}
				a^{k \mbox{ord}_m \left(a^k\right)} &= \left(a^k\right)^{\mbox{ord}_m \left(a^k\right)} \\
				&\equiv 1\mbox{ mod }m
			\end{align*}
			So then, 
			$\mbox{ord}_m a \Big| k\mbox{ord}_m \left(a^k\right)\implies
			\frac{\mbox{ord}_m a}{\gcd(\mbox{ord}_m a, k)} \Big|
			\frac{k\cdot \mbox{ord}_m \left(a^k \right) }{\gcd(\mbox{ord}_m a, k)}$.
			Then, because $\gcd$ of two fractions is $1$ we get,
			$\frac{\mbox{ord}_m a}{\gcd(\mbox{ord}_m a, k)} \Big| \mbox{ord}_m \left(a^k\right)$,
			and so $\frac{\mbox{ord}_m a}{\gcd(\mbox{ord}_m a, k)} \leq \mbox{ord}_m \left(a^k\right)$
 		\end{itemize}
		Thus, the two results together give us that 
		$$\mbox{ord}_m \left(a^k\right) = \frac{\mbox{ord}_m a}{\gcd(\mbox{ord}_m a, k)}$$
	\end{proof}

	\item \textbf{Theorem:} Suppose $r$ is a primitive root of $m$. Then $r^k$ is a primitive root of $m$
	if and only if $\gcd(k,\phi(m))=1$.
	\begin{proof}
		Well, $r^k$ is a primitive root mod $m$ if and only if $\mbox{ord}_m \left(r^k\right) = \phi(m)=\mbox{ord}_m a$,
		by the theorem this is true if and only if and only if
		$\gcd(\mbox{ord}_m r, k)=1$ if and only if $\gcd(\phi(m), k)=1$.
	\end{proof}

	\item \textbf{Corollary:} If there is a primitive root mod $m$ then there are $\phi(\phi(m))$ of them.
	\begin{proof}
		Let $r$ be a primitive root. Since powers of $r$ form a reduced residue set mod $m$ we know that all
		other integers coprime to $m$ may be written as $r^k$ for some $k$, then by the previous theorem
		we know that $r^k$ is also a primitive root if and only if $\gcd(k,\phi(m))=1$ and there are 
		$\phi(\phi(m))$ such $k$.
	\end{proof}
	\noindent\textbf{Ex.} $r=6$ is a primitive root mod 11. Then is has $\phi(\phi(11))=\phi(10)=4$
	primitive roots. What are they? Take $k$ with $\gcd(k, \phi(11))=1$ i.e. $k$ with $\gcd(k,10)=1$.
	So $k=1,3,7,9$, therefore $6^1, 6^3, 6^7, 6^9 \implies 6,7,8,2$ are the primitive roots.
\end{enumerate}
\end{enumerate}

%%%%%%%%%%%%%%%
%%% Section %%%
%%%%%%%%%%%%%%%
\subsection{Discrete Logarithms}
\rule{\textwidth}{1pt}\\
\begin{enumerate}[1.]
\item \textbf{Introduction:} Just for reference, sections 9.2 and 9.3 concern themselves with the
existence of primitive roots. They are quite technical so we will omit them and go on to section 9.4
which addresses what we can do with them. How can we solve (or even know if solutions exist) something like 
$3^x \equiv 5\mbox{ mod }22$
or -how many solutions there might be, or -if the solutions are mod 22 or something else.
In pre-calculus with $3^x \equiv 5$ we can do $x=\log_3 5$, but we cannot do that here (yet).


\item \textbf{Back to Primitive Roots:} Recall that if $\gcd(r,m)=1$ and $r$ is a 
primitive root mod $m$ then the set $\{r^1, r^2, \cdots, r^{\phi(m)}\}$ gets us all integers coprime to $m$.\\
\textbf{Ex.} $r=3$ is a primitive root of $m=14$, because 
$3^1\equiv1, 3^2\equiv 9, 3^3\equiv 13, 3^4\equiv 11, 3^5\equiv 5, 3^6\equiv 1\mbox{ mod }14$.
Note: $\mbox{ord}_{14} 3=6=\phi(14)$ so it is a primitive root. Note: we obtain
$3,9,13,1,5,1$ are all coprime to 14. Thus, we see that we can solve $3^x \equiv a\mbox{ mod }14$ if and only if
$\gcd(a,14)=1$. \\\\
In general, when $r$ is a primitive root mod $m$ then $$r^x \equiv a\mbox{ mod }m \iff \gcd(a,m)=1$$ has solutions.

\item \textbf{Indices:}
\begin{enumerate}[(a)]
	\item \textbf{Definition:} Suppose $r$ is a primitive root mod $m$ and $\gcd(a,m)=1$.
	The exponent $x$ with $1\leq x\leq \phi(m)$ satisfying
	$r^x\equiv a\mbox{ mod }m$ is the \textit{index} of $a$ mod $m$ with primitive root $r$.
	This is denoted $\mbox{ind}_r a$. Note: $m$ is missing from the notation but it matters,
	generally it is known in the problem. We could also write $\log_r a$ too but be careful to
	not think it be a 'normal' log. \\
	\textbf{Ex.} $r=3$ is a primitive root mod 14 and:
	\begin{align*}
		3^1 \equiv 3\mbox{ mod }14 &\leftrightarrow \mbox{ind}_{3} 3 = 1 \\ 
		3^2 \equiv 9\mbox{ mod }14 &\leftrightarrow \mbox{ind}_{3} 9 = 2 \\
		3^3 \equiv 13\mbox{ mod }14 &\leftrightarrow \mbox{ind}_{3} 13 = 3 \\
		3^4 \equiv 11\mbox{ mod }14 &\leftrightarrow \mbox{ind}_{3} 11 = 4 \\
		3^5 \equiv 5\mbox{ mod }14 &\leftrightarrow \mbox{ind}_{3} 5 = 5 \\
		3^6 \equiv 1\mbox{ mod }14 &\leftrightarrow \mbox{ind}_{3} 1 = 6
	\end{align*}
	\textbf{Two Immediate Notes:} If $a,b$ coprime to $m$ and $r$ is a primitive root then:
	\begin{enumerate}
		\item $r^{\mbox{ind}_r a}=a$
		\item $a\equiv b\mbox{ mod }m \iff \mbox{ind}_r a = \mbox{ind}_r b$.
		Side note, since indices are always between $1$ and $\phi(m)$ we can actually write
		$a\equiv b\mbox{ mod }m \iff \mbox{ind}_r a \equiv \mbox{ind}_r b\mbox{ mod }\phi(m)$
	\end{enumerate}
	Idea - in pre-calculus we do things like:
	\begin{align*}
		3^x &= 4^{x-1} \\
		\ln 3^x &= \ln 4^{x-1} \\
		x\ln 3 &= (x-1)\ln 4
	\end{align*}
	So now we can do things like:
	\begin{align*}
		11^x &\equiv 5^{x-1} \mbox{ mod }14 \\
		\mbox{ind}_{3} 11^x &\equiv \mbox{ind}_3 5^{x-1} \mbox{ mod }\phi(14)
	\end{align*}
	Can we know do "log-like" rules?

	\item \textbf{Index Rules:} Indices behave like logarithms (think logarithm laws)
	but there is a quirk that arises from the order of $r$, that being $\phi(m)$. To
	see why this is, consider the logarithm rule $\log(ab) = \log a + \log b$.
	It would be tempting to write: $\mbox{ind}_r (ab) = \mbox{ind}_r a + \mbox{ind}_r b$.
	However, this is not quite right. Consider that with $m=14$ and $r=3$ if we have
	$a=13$ and $b=5$ then $ab \equiv 9\mbox{ mod }14$, the tempting statement would say:
	\begin{align*}
		\mbox{ind}_3 9 &= \mbox{ind}_3 13 + \mbox{ind}_3 5 \\
		2 &= 3+5 \\
		2 &= 8
	\end{align*}
	Which is clearly false. However, we see that $2\equiv 8\mbox{ mod }\phi(14)$. \\
	\textbf{Theorem:} Let $m$ be a modulus, $r$ be a primitive root, and $a,b$ coprime to $m$.
	Then we have:
	\begin{enumerate}
		\item $\mbox{ind}_r 1 \equiv 0\mbox{ mod }\phi(m)$
		\begin{proof}
			By Euler's Theorem we know that $r^{\phi(m)}\equiv 1\mbox{ mod }m$. So,
			$$\mbox{ind}_r 1 = \phi(m) \equiv 0\mbox{ mod }\phi(m)$$
		\end{proof}
		\item $\mbox{ind}_r (ab) \equiv \mbox{ind}_r a + \mbox{ind}_r b \mbox{ mod }\phi(m)$
		\begin{proof}
			Observe that from the definition of index:
			\begin{align*}
				r^{\mbox{ind}_r (ab)} &\equiv ab\mbox{ mod }m \\
				r^{\mbox{ind}_r a + \mbox{ind}_r b} = r^{\mbox{ind}_r a} r^{\mbox{ind}_r b} &\equiv ab\mbox{ mod }m
			\end{align*}
			Then by a theorem from section 9.1 (which states that 
			$a^x \equiv a^y \mbox{ mod }m$ if and only if $x\equiv y\mbox{ mod }\mbox{ord}_m a$)
			we get:
			$$\mbox{ind}_r (ab) \equiv \mbox{ind}_r a + \mbox{ind}_r b \mbox{ mod }\phi(m)$$
		\end{proof}
		\item $\mbox{ind}_r a^k \equiv k\mbox{ind}_r a \mbox{ mod }\phi(m)$
	\end{enumerate}
\end{enumerate}

\item \textbf{The Discrete Logarithm Problem:} Given a modulus $m$ and a primitive root $r$ we know
	how to calculate $r^x \mbox{ mod }m$ (given $x$) to reduce it. How hard is it to solve
	$r^x \equiv y\mbox{ mod }m$ if $y$ is given and we need $x$ i.e. solving $\ind_r y$. The answer,
	it is extremely hard. There is no meaningfully better way than trying all
	$1\leq x\leq \phi(m)$.
	In simple cases we can try them all.

\item \textbf{Index Arithmetic:} We can use indices to solve modular problems
	involving exponets. Suppose we work frequently with the modulus $m=17$. 
	We first find a primitive root mod $17$. \\\\
	Note: Assuming you know one exists
	\begin{itemize}
		\item Find one by finding $r$ with $\ord_{17} r = \phi(17) = 16$.
		\item There will be $\phi(\phi(17)) = \phi(16) = 8$ of them.
	\end{itemize}
	Turns out $r=3$ is a primitive root. So let's solve some problems. \\\\
	First, to find necessary discrete logs (aka indices) we will build a table:
	\begin{center}
		\scalebox{.8}{
		\begin{tabular}{|c|c|c|c|c|c|c|c|c|c|c|c|c|c|c|c|c|}
			\hline
			$a\mbox{ mod }17$ & 1 & 2 & 3 & 4 & 5 & 6 & 7 & 8 & 9 & 10 & 11 & 12 & 13 & 14 & 15 & 16 \\
			\hline
			$\ind_3 a$ & 16 & 14 & 1 & 12 & 5 & 15 & 11 & 10 & 2 & 3 & 7 & 13 & 4 & 9 & 6 & 8 \\\hline
		\end{tabular}
		}
	\end{center}
	\begin{enumerate}[(a)]
	\item \textbf{Ex.} Solve $3x^{10} \equiv 12 \mbox{ mod }17$. Take the $\ind_3$ of both sides.
	\begin{align*}
		\ind_3 (3x^{10}) &\equiv \ind_3 (12) \mbox{ mod } 16 \\
		\ind_3 3 + \ind_3 x^10 &\equiv \ind_3 12 \mbox{ mod } 16 \\
		\ind_3 3 + 10(\ind_3 x) &\equiv \ind_3 12 \mbox{ mod } 16 \\
		1 + 10(\ind_3 x) &\equiv 13 \mbox{ mod } 16 \\
		10(\fbox{$\ind_3 x$}) &\equiv 12 \mbox{ mod } 16{\color{red}\text{*}}\hspace{6mm}\text{treat this as \textit{one} variable}
	\end{align*}
	Recall: $ax\equiv b\mbox{ mod }m$ has solutions if and only if $\gcd(a,m)\mid b$ and
	if so, $\exists \gcd(a,m)$ incongruent solutions mod $m$. Obtain $x_0$ via guessing or the Euclidean Algorithm,
	then all solutions have the form $x = x_0 + k \frac{m}{\gcd(a,m)}$. \\
	Since $\gcd(10,16) = 2\mid 12$, $\exists 2$ solutions mod 16. The solutions we get are:
	$$\ind_3 x \equiv 6, 14 \mbox{ mod } 16$$
	Use the table to "un-index":
	$$x \equiv 15, 2 \mbox{ mod } 17$$
	%On line three we reference the table we formed above.
	%$$x\equiv y\mbox{ mod } m \iff \ind_r x \equiv \ind_r y \mbox{ mod }\phi(m)$$
	{\color{red}\text{*}}\textbf{Note:} We could, at this point, do
	\begin{align*}
		5\ind_3 x &\equiv 6\mbox{ mod }\frac{16}{\gcd(16, 2)} \\
		5\ind_3 x &\equiv 6\mbox{ mod } 8 \\
		\ind_3 x &\equiv 6\mbox{ mod } 8
	\end{align*}
	This is unique mod 8 because $\gcd(5,8)=1$. To "un-index" we need mod 16.
	$$\ind_3 x \equiv 6\mbox{ mod } 8 \implies \ind_3 x \equiv 6, 14\mbox{ mod } 16$$
	Now we can "un-index"

	\item \textbf{Ex.} Solve $4^x \equiv 16\mbox{ mod } 17$. We will take the $\ind_3$ of both sides.
	\begin{align*}
		\ind_3 (4^x) &\equiv \ind_3 (16) \mbox{ mod } 16 \\
		x\ind_3 4 &\equiv \ind_3 (16) \mbox{ mod } 16 \\
		x(12) &\equiv 8 \mbox{ mod } 16 \\
		12x &\equiv 8 \mbox{ mod } 16 \\
		3x &\equiv 2\mbox{ mod } \frac{16}{\gcd(4, 16)} \\
		3x &\equiv 2\mbox{ mod } 4
	\end{align*}
	Since $\gcd(3,4)=1\mid 2$, $\exists$ a solution mod 4.
	$$x \equiv 2\mbox{ mod } 4$$
	\textbf{Note:} Any of $x= \cdots, -6, -2, 2, 6, \cdots$ works. \\
	\textbf{Note:} Could also give as $x \equiv 2, 6, 10, 14 \mbox{ mod } 16$ ("un-index" back to original mod)
\end{enumerate}
\textbf{Note:} We can do either of these problems again with a completely different primitive root mod 17.
As an exercise in understanding, we could do the two examples above with a different primitive root.
\end{enumerate}
%%%%%%%%%%%%%%%
%%% Section %%%
%%%%%%%%%%%%%%%
\subsection{Problems}
\rule{\textwidth}{1pt}\\
\begin{enumerate}
\item
  Determine the following orders and justify each.
  \begin{enumerate}
  \item
	$\ord_{21}8$

  \item
	$\ord_{25}8$

  \end{enumerate}

\item
  Find all primitive roots (reduced mod $50$) for $n=50$ as follows:
  First find (with justification) the smallest primitive root.
  Then use the Theorem from class which yields all the remaining ones.

\item
  Prove that if $p$ is an odd prime and $a$ has $\ord_pa=2k$ then
  $a^k\equiv -1\mod p$

\item
  Show that if $a$ is relatively prime to $m$ and $\ord_ma=m-1$
  then $m$ is prime.

\item
  Suppose $r$ is a primitive root of an odd prime $p$.
  Prove that:
  $$\ind_r(p-a)\equiv \ind_ra+\left(\frac{p-1}{2}\right)\mod p-1$$

\item
  Show that if $n$ is an integer and $a$ and $b$ are integers which
  are relatively prime to $n$ with $\gcd(\ord_n a,\ord_n b)=1$
  then $\ord_n(ab)=(\ord_n a)(\ord_n b)$.

\item
  Let $r$ be a primitive root of the prime $p$ with $p\equiv 1\mod 4$.
  Prove that $-r$ is also a primitive root.

\item
  It's a fact that $r=7$ is a primitive root mod $13$.
  \begin{enumerate}
  \item
	Use this to construct a table of indices
	for this primitive root.

  \item
	Use the table of indices to solve the equation:
	$x^{2}\equiv 12\mod 13$.
	Your answer(s) should be mod $13$.

  \item
	Use the table of indices to solve the equation:
	$4^x\equiv 12 \mod 13$.
	Your answer(s) should be mod $12$.

  \end{enumerate}

\item
  With logarithms we have
  $\log_r a-\log_r b=\log_r\left(\frac ab\right)$
  \begin{enumerate}
  \item
	Why is it not reasonable to write
	$\equiv \ind_r a-\ind_r b\mod \phi(n) \equiv \ind_r\left(\frac ab\right)$
	when $a,b$ are coprime to $n$ and $r$ is a primitive root?

  \item
	What would be a reasonable index substitute for this logarithm rule?

  \item
	Prove this substitute.

  \end{enumerate}

\item
  Suppose $p$ is an odd prime and both $r_1$ and $r_2$
  are primitive roots for $p$.
  Prove that $r_1r_2$ is not a primitive root for $p$.

\end{enumerate}

\end{document}