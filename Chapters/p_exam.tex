\documentclass[class=article, crop=false]{standalone}
\usepackage[utf8]{inputenc}
\usepackage{import}
\usepackage[subpreambles=true]{standalone}
\usepackage{graphicx}
\usepackage{amsfonts}
\usepackage{mathrsfs}
\usepackage{mathtools}
\usepackage{enumerate}
\usepackage{fancyhdr}
\usepackage[colorlinks=true,linkcolor=black,anchorcolor=black,citecolor=black,filecolor=black,menucolor=black,runcolor=black,urlcolor=blue]{hyperref}
\usepackage{epsfig,amssymb,amsmath,multicol,tikz,pgfplots,amsthm,enumerate}

\def\naturals{{\mathbb N}}
\def\reals{{\mathbb R}}
\def\complex{{\mathbb C}}
\def\poly{{\mathbb P}}
\def\integers{{\mathbb Z}}
\def\rationals{{\mathbb Q}}
\def\irrationals{{\mathbb I}}
\def\inlinesum#1#2{\overset{#2}{\underset{#1}{\sum}}}
\def\inlineprod#1#2{\overset{#2}{\underset{#1}{\prod}}}
\def\ord{{\text{ord}}}
\def\ind{{\text{ind}}}

\begin{document}
\addcontentsline{toc}{section}{Practice Exams}
\section*{Practice Exams}

%%%%%%%%%%%%%%%
%%% Section %%%
%%%%%%%%%%%%%%%
\addcontentsline{toc}{subsection}{Exam 1 Sample A}
\subsection*{Exam 1 Sample A}
\rule{\textwidth}{1pt}\\
\begin{enumerate}[1.]
	\item Write down the prime factorization of $10!$.
	
	\item Find the least non-negative residue of $11^{67} \mod 13$.
	
	\item Find all incongruent solutions $\mod 40$, as least non-negative residues,
	to the following lienar congruence: $$12x\equiv 28\mod 40$$

	\item Use the Euclidean Algorithm to find $\gcd (390,72)$ and write this as a linear
	combination of the two.

	\item Use the Chinese Remainder Theorem to find the smallest positive solution to the
	system:
	\begin{align*}
		x &\equiv 2\mod 5 \\
		x &\equiv 1\mod 6 \\
		x &\equiv 4\mod 7
	\end{align*}

	\item Use mathematical induction to prove that: $$n!\geq n^3 \text{ for } n\geq 6$$

	\item Determine if the following sets are well-ordered or not.
	You may assume only that $\integers^+$ is well-ordered.
	\begin{align*}
		S_1 &= [0,1]\cap \rationals \\
		S_2 &= \{1-2^k \mid k\in\integers^+ \}
	\end{align*}

	\item Use the Fundamental Theorem of Arithmetic (uniqueness of prime factorization) to prove
	that $\sqrt{2}$ is irrational. Hint: Use contradiction.

	\item Suppose $a,b,c,d\in\integers$ with $a\mid c$, $b\mid c$, $d=\gcd(a,b)$, and $d^2\mid c$.
	Prove that $ab\mid c$.
\end{enumerate}

%%%%%%%%%%%%%%%
%%% Section %%%
%%%%%%%%%%%%%%%
\addcontentsline{toc}{subsection}{Exam 1 Sample B}
\subsection*{Exam 1 Sample B}
\rule{\textwidth}{1pt}\\
\begin{enumerate}[1.]
	\item 
	\begin{enumerate}[(a)]
		\item Find $\pi(18)$.
		
		\item Show that the set $\{\frac{a}{b} \mid a,b\in\integers^+, a>b\}$ is not well-ordered.
		
		\item Find how many primes there are, approximately, between one billion and two billion.
	
	\end{enumerate}
	
	\item Find the number of zeros at the end of $1000!$ with justification.
	
	\item The following are all false. Provide explicit numerical counterexamples.
	\begin{enumerate}[(a)]
		\item $a\mid bc$ implies $a\mid b$ or $a\mid c$.
		
		\item $a\mid b$ and $a\mid c$ implies $b\mid c$.
		
		\item $3\mid a$ and $3\mid b$ implies $\gcd(a,b)=3$.
	
	\end{enumerate}

	\item Simplify $\inlineprod{j=1}{n} \left(1+\frac{2}{j}\right)$. Your result should not have a 
	$\inlineprod{}{}$ in it, or any sort of long product.

	\item Use Mathematical Induction to prove $2^1+2^2+\cdots+2^n=2^{n+1}-2$ for all integers $n\geq 1$.
	
	\item Find all $n\in\integers$ with $n^2-5n+6$ prime.
	
	\item Suppose $p$ is a prime and $a$ is a positive integers less than $p$. Find all possibilities for
	$\gcd(a,7a+p)$.

	\item Use the Fundamental Theorem of Arithmetic to prove that $\sqrt{6}$ is irrational.
	
	\item Prove that for $a,b\in\integers$ and $n\in\integers^+$ that if $a^n\mid b^n$ then $a\mid b$.
	
\end{enumerate}

%%%%%%%%%%%%%%%
%%% Section %%%
%%%%%%%%%%%%%%%
\addcontentsline{toc}{subsection}{Exam 2 Sample A}
\subsection*{Exam 2 Sample A}
\rule{\textwidth}{1pt}\\
\begin{enumerate}
	\item Show that 91 is a Fermat Pseudoprime to the base 3. Note that 91 is not prime!
	
	\item Prove that if $n\geq 2$ and $\gcd(6,n)=1$ then $\phi(3n)=2\phi(2n)$.
	
	\item Classify all numbers $n$ for which $\tau(n) = 12$.
	
	\item Suppose $n$ is a perfect number and $p$ is a prime such that $pn$ is also perfect.
	Prove $\gcd(p,n)\neq 1$.
	
	\item Prove that $a^{\phi(b)} + b^{\phi(a)} \equiv 1\mbox{ mod } ab$ if $\gcd(a,b)=1$.
	
	\item Suppose that $p$ is prime and $n\in\integers^+$. Prove that $p\nmid n$ iff $\phi(pn)=(p-1)\phi(n)$.
	
	\item 
	\begin{enumerate}
		\item Show that 3 is a primitive root modulo 17.
		
		\item Find all primitive roots modulo 17.
		
	\end{enumerate}

	\item A partial table of indices for 7, a primitive root of 13 is given here:
	\begin{figure}[h]
    \centering
    \begin{tabular}{|c|c|c|c|c|c|c|c|c|c|c|c|c|}
      \hline
      $a$ & 1 & 2 & 3 & 4 & 5 & 6 & 7 & 8 & 9 & 10 & 11 & 12 \\
      \hline
      $\ind_7 a$ & 12 & $b$ & 8 & 10 & 3 & 7 & $a$ & 9 & 4 & 2 & 5 & 6 \\
      \hline
    \end{tabular}
    \end{figure}
	\begin{enumerate}
		\item Find $a$ and $b$.
		
		\item Use the table to solve the congruence $3^{x-1} \equiv 5\mbox{ mod } 13$.
		
		\item Use the table to solve the congruence $4x^5 \equiv 11 \mbox{ mod } 13$.
	
	\end{enumerate}

	\item Suppose $\ord_p a = 3$, where $p$ is an odd prime. Show $\ord_p (a+1) = 6$.
	
	\item Suppose $r$ is a primitive root modulo $m$, and $k$ is a positive integers with
	$\gcd(k, \phi(m))=1$ Prove $r^k$ is also a primitive root.
\end{enumerate}

%%%%%%%%%%%%%%%
%%% Section %%%
%%%%%%%%%%%%%%%
\addcontentsline{toc}{subsection}{Exam 2 Sample B}
\subsection*{Exam 2 Sample B}
\rule{\textwidth}{1pt}\\
\begin{enumerate}
	\item Calculate:
	\begin{enumerate}
		\item $\phi(2^3\cdot 5\cdot 11^2)$
		\item $\sigma(200)$
		\item $\tau(2000)$
	\end{enumerate}

	\item Use Wilson's Theorem to find the remainder when 16! is divided by 19.
	
	\item Find all $n$ with $\phi(n)=16$.
	
	\item Show that 25 is a Fermat Pseudoprime to the base 7.
	
	\item An abundant number is a number $n$ with $sigma(n) > 2n$. Prove that there are
	infinitely many even abundant numbers by finding on eabundant number and by showing
	that if $n$ is abundant and a prime $p$ satisfies $p\nmid n$ then $pn$ is also abundant.

	\item A partial table of indices for 2, a primitive root of 13, is given here:
	\begin{figure}[h]
    \centering
    \begin{tabular}{|c|c|c|c|c|c|c|c|c|c|c|c|c|}
      \hline
      $a$ & 1 & 2 & 3 & 4 & 5 & 6 & 7 & 8 & 9 & 10 & 11 & 12 \\
      \hline
      $\ind_2 a$ & 12 & 1 & 4 & 2 & 9 & 5 & 11 & 3 & $a$ & $b$ & 7 & 6 \\
      \hline
    \end{tabular}
    \end{figure}
	\begin{enumerate}
		\item Find $a$ and $b$ with justification.
		\item Use the table to solve the congruence $3^{2x+1}\equiv 9\mbox{ mod } 13$.
		\item Use the table to solve the congruence $7x^5 \equiv 3\mbox{ mod } 13$.
	\end{enumerate}

	\item Prove that if $\ord_n a = hk$ then $\ord_n (a^h)= k$.
	
	\item Let $r$ be a primitive root for an odd prime $p$. Prove that $\ind_r (p-1) = \frac{1}{2}(p-1)$.
	
	\item Find all positive integers $n$ such that $\phi(n)$ is prime. Explain!
	
	\item Show that if $a$ is relatively prime to $m$ and $\ord_m a=m-1$ then $m$ is prime.
\end{enumerate}

%%%%%%%%%%%%%%%
%%% Section %%%
%%%%%%%%%%%%%%%
\addcontentsline{toc}{subsection}{Final Exam Sample A}
\subsection*{Final Exam Sample A}
\rule{\textwidth}{1pt}\\

\end{document}