\documentclass[class=article, crop=false]{standalone}
\usepackage[utf8]{inputenc}
\usepackage{import}
\usepackage[subpreambles=true]{standalone}
\usepackage{graphicx}
\usepackage{amsfonts}
\usepackage{mathrsfs}
\usepackage{mathtools}
\usepackage{enumerate}
\usepackage{fancyhdr}
\usepackage[colorlinks=true,linkcolor=black,anchorcolor=black,citecolor=black,filecolor=black,menucolor=black,runcolor=black,urlcolor=blue]{hyperref}
\usepackage{epsfig,amssymb,amsmath,multicol,tikz,pgfplots,amsthm,enumerate}

\def\naturals{{\mathbb N}}
\def\reals{{\mathbb R}}
\def\complex{{\mathbb C}}
\def\poly{{\mathbb P}}
\def\integers{{\mathbb Z}}
\def\rationals{{\mathbb Q}}
\def\irrationals{{\mathbb I}}
\def\inlinesum#1#2{\overset{#2}{\underset{#1}{\sum}}}
\def\inlineprod#1#2{\overset{#2}{\underset{#1}{\prod}}}


\begin{document}

\section{Practice Exams}

%%%%%%%%%%%%%%%
%%% Section %%%
%%%%%%%%%%%%%%%
\subsection{Exam 1 Spring 2020}
\rule{\textwidth}{1pt}
Note: I have ordered these in terms of what I think is increasing difficulty.
You may have other opinions! Remember that this exam will be curved, I do not expect
you to finish all the problems in 50 minutes.\\
\rule{\textwidth}{1pt}
\begin{enumerate}[1.]
	\item Write down the prime factorization of $10!$.
	
	\item Find the least non-negative residue of $11^{67} \mod 13$. \\
	Using Fermat's Little Theorem. Well $13\nmid 11$ so $11^{}$
	
	\item Find all incongruent solutions $\mod 40$, as least non-negative residues,
	to the following lienar congruence: $$12x\equiv 28\mod 40$$

	\item Use the Euclidean Algorithm to find $\gcd (390,72)$ and write this as a linear
	combination of the two.

	\item Use the Chinese Remainder Theorem to find the smallest positive solution to the
	system:
	\begin{align*}
		x &\equiv 2\mod 5 \\
		x &\equiv 1\mod 6 \\
		x &\equiv 4\mod 7
	\end{align*}

	\item Use mathematical inductino to prove that: $$n!\geq n^3 \text{ for } n\geq 6$$
	
	\item One of the following two set is well-ordered and one is not. Decide which is which
	and justify. You may assume only that $\integers^+$ is well-ordered.
	\begin{align*}
		S_1 &= [0,1]\cap \rationals \\
		S_2 &= \{1-2^k \mid k\in\integers^+ \}
	\end{align*}

	\item Use the Fundamental Theorem of Arithmetic (uniqueness of prime factorization) to prove
	that $\sqrt{2}$ is irrational. Hint: Use contradiction.

	\item Suppose $a,b,c,d\in\integers$ with $a\mid c$, $b\mid c$, $d=\gcd(a,b)$, and $d^2\mid c$.
	Prove that $ab\mid c$.

\end{enumerate}

%%%%%%%%%%%%%%%
%%% Section %%%
%%%%%%%%%%%%%%%
\subsection{Exam 1 Summer 2016}
\rule{\textwidth}{1pt}
Note: I've ordered these by difficulty as I perceive it. Your opinion on difficulty might vary,
but knowing how I ordered them might help you decide which to do first and which to do last!\\
\rule{\textwidth}{1pt}
\begin{enumerate}[1.]
	\item 
	\begin{enumerate}[(a)]
		\item Find $\pi(18)$.
		
		\item Show that the set $\{\frac{a}{b} \mid a,b\in\integers^+, a>b\}$ is not well-ordered.
		
		\item Find how many primes there are, approximately, between one billion and two billion.
	
	\end{enumerate}

	\item Find all integer solutions to $115x+25y=10$.
	
	\item Find the number of zeros at the end of $1000!$ with justification.
	
	\item The following are all false. Provide explicit numerical counterexamples.
	\begin{enumerate}[(a)]
		\item $a\mid bc$ implies $a\mid b$ or $a\mid c$.
		
		\item $a\mid b$ and $a\mid c$ implies $b\mid c$.
		
		\item $3\mid a$ and $3\mid b$ implies $\gcd(a,b)=3$.
	
	\end{enumerate}

	\item Simplify $\inlineprod{j=1}{n} \left(1+\frac{2}{j}\right)$. Your result should not have a 
	$\inlineprod{}{}$ in it, or any sort of long product.

	\item Use Mathematical Induction to prove $2^1+2^2+\cdots+2^n=2^{n+1}-2$ for all integers $n\geq 1$.
	
	\item Find all $n\in\integers$ with $n^2-5n+6$ prime.
	
	\item Suppose $p$ is a prime and $a$ is a positive integers less than $p$. Find all possibilities for
	$\gcd(a,7a+p)$.

	\item Use the Fundamental Theorem of Arithmetic to prove that $\sqrt{6}$ is irrational.
	
	\item Prove that for $a,b\in\integers$ and $n\in\integers^+$ that if $a^n\mid b^n$ then $a\mid b$.
	
\end{enumerate}

\end{document}