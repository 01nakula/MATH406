\documentclass[class=article, crop=false]{standalone}
\usepackage[utf8]{inputenc}
\usepackage{import}
\usepackage[subpreambles=true]{standalone}
\usepackage{graphicx}
\usepackage{amsfonts}
\usepackage{mathrsfs}
\usepackage{mathtools}
\usepackage{enumerate}
\usepackage{fancyhdr}
\usepackage[colorlinks=true,linkcolor=black,anchorcolor=black,citecolor=black,filecolor=black,menucolor=black,runcolor=black,urlcolor=blue]{hyperref}
\usepackage{epsfig,amssymb,amsmath,multicol,tikz,pgfplots,amsthm,enumerate}

\def\naturals{{\mathbb N}}
\def\reals{{\mathbb R}}
\def\complex{{\mathbb C}}
\def\poly{{\mathbb P}}
\def\integers{{\mathbb Z}}
\def\rationals{{\mathbb Q}}
\def\irrationals{{\mathbb I}}
\def\inlinesum#1#2{\overset{#2}{\underset{#1}{\sum}}}
\def\inlineprod#1#2{\overset{#2}{\underset{#1}{\prod}}}


\begin{document}

\section{Practice Exams}

%%%%%%%%%%%%%%%
%%% Section %%%
%%%%%%%%%%%%%%%
\subsection{Exam 1 Spring 2020}
\rule{\textwidth}{1pt}
Note: I have ordered these in terms of what I think is increasing difficulty.
You may have other opinions! Remember that this exam will be curved, I do not expect
you to finish all the problems in 50 minutes.\\
\rule{\textwidth}{1pt}
\begin{enumerate}[1.]
	\item Write down the prime factorization of $10!$. \\\\
	Let $10!$ be written as,
	\begin{align*}
		10! &= 1\cdot 2\cdot 3\cdot 4\cdot 5\cdot 6\cdot 7\cdot 8\cdot 9\cdot 10 \\
		&= 1\cdot 2\cdot 3\cdot 2^2\cdot 5\cdot (2\times 3)\cdot 7\cdot 2^3\cdot 3^2\cdot (2\times 5) \\
		&= 1\cdot 2^8\cdot 3^4\cdot 5^2\cdot 7
	\end{align*}
	Therefore, the prime factorization of $10!$ is $2^8 3^4 5^2 7$.
	
	\item Find the least non-negative residue of $11^{67} \mod 13$. \\\\
	Using Fermat's Little Theorem. 
	Well $13\nmid 11$ so $11^{12} \equiv 1\mod 13$. Then $67= 12(5) + 7$ so,
	\begin{align*}
		11^{67} = 11^{12(5)+7} = 11^{12^5}11^7 &\equiv (1)^{10} 11^{7} \mod 13 \\
		&\equiv 11^7 \mod 13 \\
		&\equiv 11 \cdot 1771561 \mod 13 \\
		&\equiv 11 (-1) \mod 13 \\
		&\equiv -11 \mod 13 \\
		&\equiv 2 \mod 13
	\end{align*}
	So $2$ is the least non-negative residue.
	
	\item Find all incongruent solutions $\mod 40$, as least non-negative residues,
	to the following lienar congruence: $$12x\equiv 28\mod 40$$
	Since $\gcd(12,40)=4\mid 28$ there exists a solution. We use the Euclidean Algorithm to solve
	$12x' + 40y' = 4$. This gives us $12(-3)+40(1)=4$, we want a 28 on the right hand side so 
	multiple by $7$. We then get $12(-21)+40(7)=28$, so $12(-21)\equiv 28\mod 40$. Therefore,
	$x_0 \equiv 19\mod 40$, so all solutions are then
	$$x\equiv 19 + 10k\mod 40k, k=0,1,2,3$$
	That is $x\equiv 19, 29, 39, 9\mod 40$

	\item Use the Euclidean Algorithm to find $\gcd (390,72)$ and write this as a linear
	combination of the two. \\\\
	Using the Euclidean Algorithm we do the following:
	\begin{align*}
		390 &= 5(72) + 30 \\
		72 &= 2(30) + 12 \\
		30 &= 2(12) + 6 \\
		12 &= 2(6) + 0
	\end{align*}
	So the gcd is $6$. Now the find the linear combination.
	\begin{align*}
		6 &= 1(30) - 2(12) \\
		&= 1(30) - 2(72 - 2(30)) \\
		&= 5(30) - 2(72) \\
		&= 5(390 - 5(72)) - 2(72) \\
		&= 5(390) - 27(72)
	\end{align*}
	Where $\alpha = 5$ and $\beta = -27$.

	\item Use the Chinese Remainder Theorem to find the smallest positive solution to the
	system:
	\begin{align*}
		x &\equiv 2\mod 5 \\
		x &\equiv 1\mod 6 \\
		x &\equiv 4\mod 7
	\end{align*}
	Test to see if all $m_i$ are pairwise coprime, $\gcd(5,6)=\gcd(5,7)=\gcd(6,7)$. This means that
	$M=210$, $M_1=42$, $M_2=35$, and $M_3=30$.
	\begin{itemize}
		\item[] Solve for $y_1$:
			\begin{align*}
				42y_1 &\equiv 1\mod 5 \\
				2y_1 &\equiv 1\mod 5 \\
				y_1 &= 3
			\end{align*}
		
		\item[] Solve for $y_2$:
			\begin{align*}
				35y_2 &\equiv 1\mod 6 \\
				5y_2 &\equiv 1\mod 6 \\
				y_2 &= 5
			\end{align*}
		
		\item[] Solve for $y_3$:
			\begin{align*}
				30y_3 &\equiv 1\mod 7 \\
				2y_3 &\equiv 1\mod 7 \\
				y_3 &= 4
			\end{align*}
	\end{itemize}
	So we then get
	$$x = (2)(42)(3) + (1)(35)(5) + (4)(30)(4) \equiv 907\mod 210$$
	$$x\equiv 67\mod 210$$
	So the least non-negative residue is $67$.

	\item Use mathematical induction to prove that: $$n!\geq n^3 \text{ for } n\geq 6$$
	
	\begin{proof}
		$ $\\
		\begin{itemize}
			\item[] \textbf{Base Case:}
			\begin{itemize}
				\item[] Let $n=6$, $n! = 720$ and $6^3=216$, $720\geq 216$ so the base case is valid.
			\end{itemize}
			\item[] \textbf{Inductive Hypothesis:}
			\begin{itemize}
				\item[] Assume from the inductive hypothesis that the conclusion is true for some $n\geq 6$.
				This implies that $n!\geq n^3$.
			\end{itemize}
			\item[] \textbf{Inductive Step:}
			\begin{itemize}
				\item[] Then consider the equation to $n+1$:
				\begin{align*}
					(n+1)! &\geq (n+1)^3 \\
					(n+1)(n!) &\geq (n+1)^3 \\
					(n+1)n^3 &> (n+1)^3 \hspace{5mm}\text{by IH} \\
					n^3 &> (n+1)^2 \\
					n^3 &> n^2 + 2n + 1
				\end{align*}
				Which is true for any $n\geq 3$.
			\end{itemize}
			\item[] Thus for all $n\geq 6$,
			$$n!\geq n^3$$
		\end{itemize}
	\end{proof}
	
	\item Determine if the following sets are well-ordered or not.
	You may assume only that $\integers^+$ is well-ordered.
	\begin{align*}
		S_1 &= [0,1]\cap \rationals \\
		S_2 &= \{1-2^k \mid k\in\integers^+ \}
	\end{align*}
	The set $S_1$ is not well-ordered because the subset $(0,0)\cap \reals$ has no least element.
	Likewise, the set $S_2$ is also not well-ordered because the set itself has no least element.

	\item Use the Fundamental Theorem of Arithmetic (uniqueness of prime factorization) to prove
	that $\sqrt{2}$ is irrational. Hint: Use contradiction. \\\\
	Suppose that $\sqrt{2}$ is rational, this means that $\sqrt{2}$ is of the form $\frac{a}{b}$, $a,b\in\integers^+$.
	Then $2=\frac{a^2}{b^2}$ so $a^2=2b^2$. Because $a^2$ and $b^2$ are both squared the prime factorizations
	of both are even, but $a^2=2b^2$ implies there is an odd number of prime factorizations for $2$. This contradicts
	uniqueness of prime factors.

	\item Suppose $a,b,c,d\in\integers$ with $a\mid c$, $b\mid c$, $d=\gcd(a,b)$, and $d^2\mid c$.
	Prove that $ab\mid c$.\\\\
	Given that $a\mid c$, $b\mid c$, and $d^2\mid c$ given that $d=\gcd(a,b)$ we can \emph{not} conclude
	that $ab\mid c$. We will show this with a simple contradiction, let $a=2$, $b=4$, $c=4$.
	We know that $2\mid 4$ and $4\mid 4$, it follows that $\gcd(2,4)=2^2\mid 4$ but 
	$ab\nmid c$ because $2\cdot 4\nmid 4$ because $8>4$. So the statement is false.

\end{enumerate}

%%%%%%%%%%%%%%%
%%% Section %%%
%%%%%%%%%%%%%%%
\subsection{Exam 1 Summer 2016}
\rule{\textwidth}{1pt}
Note: I've ordered these by difficulty as I perceive it. Your opinion on difficulty might vary,
but knowing how I ordered them might help you decide which to do first and which to do last!\\
\rule{\textwidth}{1pt}
\begin{enumerate}[1.]
	\item 
	\begin{enumerate}[(a)]
		\item Find $\pi(18)$.\\\\
		We first list the primes up to $18$, $\{2,3,5,7,11,13,17\}$. We see that there are $7$ primes,
		therefore $\pi(18)=7$.
		
		\item Show that the set $\{\frac{a}{b} \mid a,b\in\integers^+, a>b\}$ is not well-ordered.\\\\
		Since $a>b$ we know a subset 
		$\{\frac{2}{1}, \frac{3}{2}, \frac{4}{3},\cdots\}$ exists, and it does not have a least element.
		Since the subset does not have a least element, the set is not well-ordered.
		
		\item Find how many primes there are, approximately, between one billion and two billion.\\\\
		From section 2.2 we know that for very large $x$, $\pi(x)=\frac{x}{\ln x}$.
		So there are, approximately, 
		$$\frac{2000000000}{\ln(2000000000)} - \frac{1000000000}{\ln(1000000000)}$$
		primes between one and two billion.
	
	\end{enumerate}

	%\item Find all integer solutions to $115x+25y=10$.
	
	\item Find the number of zeros at the end of $1000!$ with justification.\\\\
	Zeros at the end of numbers are from multiples of 10 which are pairs of 2 and 5, so
  	we find the number of pairs of 2's and 5's to find the number of zeros. Let $d_n(x)$
  	represent the sum of the numbers divisible by all powers of $n$ less than $x$. 
  	$$d_2(1000!) = 500 + 250 + 125 + 62 + 31 + 15 + 7 + 3 + 1 = 994$$
  	$$d_5(1000!) = 200 + 40 + 8 + 1 = 249$$
  	Thus, there can only be 249 pairs of 2's and 5's, so there are only 249 
  	10's, so there are 249 zeros at the end of $(1000!)$.
	
	\item The following are all false. Provide explicit numerical counterexamples.
	\begin{enumerate}[(a)]
		\item $a\mid bc$ implies $a\mid b$ or $a\mid c$.\\\\
		$6\mid 3\cdot4$ but $6\nmid 3$ and $6\nmid 4$.
		
		\item $a\mid b$ and $a\mid c$ implies $b\mid c$.\\\\
		$2\mid 4$ and $2\mid 6$ but $4\nmid 6$.
		
		\item $3\mid a$ and $3\mid b$ implies $\gcd(a,b)=3$.\\\\
		$3\mid 6$ and $3\mid 12$ but $\gcd(6,12)=6\neq 3$.
	
	\end{enumerate}

	\item Simplify $\inlineprod{j=1}{n} \left(1+\frac{2}{j}\right)$. Your result should not have a 
	$\inlineprod{}{}$ in it, or any sort of long product.\\\\
	$$\prod_{j=1}^{n}\left(1+\frac{2}{j}\right)= \prod_{j=1}^{n}\left(\frac{j+2}{j}\right) = \frac{3}{1}\times\frac{4}{2}\times\cdots\times\frac{n+2}{n} = \frac{(n+2)(n+1)}{2}$$

	\item Use Mathematical Induction to prove $2^1+2^2+\cdots+2^n=2^{n+1}-2$ for all integers $n\geq 1$.
	\begin{proof}
		$ $
		\begin{itemize}
			\item[] \textbf{Base Case:}
			\begin{itemize}
				\item[] Let $n=1$, $2^1=2^{1+1}-2$ is true, so the base case is valid.
			\end{itemize}
			\item[] \textbf{Inductive Hypothesis:}
			\begin{itemize}
				\item[] Assume from the inductive hypothesis that the conclusion is true for some $n\geq 1$.
				This implies that $2^1+2^2+\cdots+2^n=2^{n+1}-2$.
			\end{itemize}
			\item[] \textbf{Inductive Step:}
			\begin{itemize}
				\item[] Then consider the equation to $n+1$:
				\begin{align*}
					2^1+2^2+\cdots+2^{n+1} &= 2^1+2^2+\cdots2^n+2^{n+1} \\
					&= 2^{n+1}-2 + 2^{n+1} \hspace{5mm}\text{ by IH} \\
					&= 2^{(n+1)+1}-2
				\end{align*}
			\end{itemize}
			\item[] Thus for all $n\geq 1$, $$2^1+2^2+\cdots+2^n=2^{n+1}-2$$    
		\end{itemize}
	\end{proof}
	
	\item Find all $n\in\integers$ with $n^2-5n+6$ prime.\\\\
	Factor $n^2-5n+6$ out to be of the form $(n-2)(n-3)$. For this polynomial to be prime
	we need one factor to be $\pm1$ and the other to be a prime. We have four cases:
	\begin{itemize}
		\item If $n-2=1\implies n=3$ then $n^2-5n+6=0$ which is not prime.
		\item If $n-2=-1\implies n=1$ then $n^2-5n+6=2$ which is prime.
		\item If $n-3=1\implies n=4$ then $n^2-5n+6=2$ which is prime.
		\item If $n-3=-1\implies n=2$ then $n^2-5n+6=0$ which is not prime.
	\end{itemize}
	So the only values of $n$ such that $n^2-5n+6$ is prime is $n=1,4$.
	
	\item Suppose $p$ is a prime and $a$ is a positive integers less than $p$. Find all possibilities for
	$\gcd(a,7a+p)$.\\\\
	We know that $\gcd(a,7a+p)=\gcd(a,p)$, but since $a<p$ and the only divisors of $p$ are $1$ and
	$p$ we know that $a\nmid p$, therefore $\gcd(a,p)=1$.

	\item Use the Fundamental Theorem of Arithmetic to prove that $\sqrt{6}$ is irrational.\\\\
	Suppose that $\sqrt{6}$ is rational, this means that $\sqrt{6}$ is of the form $\frac{a}{b}$, $a,b\in\integers^+$.
	Then $6=\frac{a^2}{b^2}$ so $a^2=6b^2$. Because $a^2$ and $b^2$ are both squared the prime factorizations
	of both are even, but $a^2=6b^2$ implies there is an odd number of prime factorizations for $2$ and $3$. This contradicts
	uniqueness of prime factors.
	
	\item Prove that for $a,b\in\integers$ and $n\in\integers^+$ that if $a^n\mid b^n$ then $a\mid b$.\\\\
	Suppose that $a^n\mid b^n$, this implies that $b^n = ka^n$ for some $k\in\integers$.
	We know that any prime in the prime factorization of $k$ must be to the power of $\alpha n$.
	This implies that $k=p_1^{\alpha_1 n} p_2^{\alpha_2 n} \cdots p_i^{\alpha_i n}$ which in turn
	implies that $k=(p_1^{\alpha_1} p_2^{\alpha_2}\cdots p_i^{\alpha_i})^n$. From this we know that 
	$k$ is a perfect square, meaning that $\sqrt{k}\in\integers^+$, thus $a\sqrt{k}=b$ and
	$a\mid b$.
	
\end{enumerate}

\end{document}