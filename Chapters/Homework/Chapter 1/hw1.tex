\documentclass{article}
\usepackage{epsfig,amssymb,amsmath,multicol,tikz,pgfplots,amsthm}

\def\naturals{{\mathbb N}}
\def\reals{{\mathbb R}}
\def\complex{{\mathbb C}}
\def\poly{{\mathbb P}}
\def\integers{{\mathbb Z}}
\def\rationals{{\mathbb Q}}
\def\irrationals{{\mathbb I}}
\def\inlinesum#1#2{\overset{#2}{\underset{#1}{\sum}}}
\def\inlineprod#1#2{\overset{#2}{\underset{#1}{\prod}}}

\setlength{\oddsidemargin}{0in}
\setlength{\textwidth}{6in}
\setlength{\topmargin}{0in}
\setlength{\textheight}{8in}

\newif\ifanswers
%\answerstrue

\begin{document}
\pagestyle{empty}

\noindent\textbf{Math 406: Homework for Chapter 1}\hfill Neelam Akula

\begin{enumerate}

%%%%%%%%%%%%%%%%%%%%%
%%%%% Problem 1 %%%%%
%%%%%%%%%%%%%%%%%%%%%
\item
  Determine whether each of the following sets is well-ordered.  If
  so, give a proof which relies on the fact that $\integers^+$ is well-ordered.
  If not, give an example of a subset with no least element.
  \begin{enumerate}
  \item
    $\left\{a\,\big|\, a\in\integers,a>3\right\}$

    Is a subset of $\integers^+$ and therefore is well-ordered.
  \item
    $\left\{a\,\big|\, a\in\rationals,a>3\right\}$

    There is no least element so the set is not well-ordered.
  \item
    $\left\{\frac a2\,\big|\, a\in\integers,a\geq 10\right\}$

    Consider the set $\left\{a\,\big|\, a\in\integers,a\geq 10\right\}$, it is apparent that this is a subset of $\integers^+$ and therefore is well-ordered.
    So the set $\left\{\frac a2\,\big|\, a\in\integers,a\geq 10\right\}$ is also well-ordered because it holds a least element ($\frac{10}{5}$).
  \item
    $\left\{\frac 2a\,\big|\, a\in\integers,a>10\right\}$

    There is no least element so the set is not well-ordered.
  \end{enumerate}

%%%%%%%%%%%%%%%%%%%%%
%%%%% Problem 2 %%%%%
%%%%%%%%%%%%%%%%%%%%%
\item
  Suppose $a,b\in\integers^+$ are unknown.  Let
  $S=\left\{a-bk\,\big|\, k\in\integers,a-bk>0\right\}$.
  Explain why $S$ has a smallest element but no largest element.

  Since $S$ is a subset of $\integers^+$ by well-ordering we know that $S$ has a least element, and because $k\in\integers$,
  $k$ can be $0$ and therefore there is no most element.

%%%%%%%%%%%%%%%%%%%%%
%%%%% Problem 3 %%%%%
%%%%%%%%%%%%%%%%%%%%%
\item
  Use the well-ordering property to show that
  $\sqrt 5$ is irrational.
  \begin{proof}
    Suppose $\sqrt{5}$ is rational and is of the form $\frac{a}{b}$ where $a,b\in\mathbb{Z}^+$ and $b\neq 0$.
    Consider the set $S$,
    $$S = \left\{ k \mid k, k\sqrt{5}\in\mathbb{Z}^+ \right\}$$
    We know that $S$ is a subset of $\mathbb{Z}^+$ and that $b\in S$, by well-ordering this implies that $S$ has a least element.
    Let $l$ be the least element in $S$.\\
    Consider the properties of $l'$ where $l' = l\sqrt{5}-2l$,
    \begin{itemize}
      \item $l'=l\sqrt{5}-2l= l(\sqrt{5}-2) \implies 0 < l' < l$.
      \item Since $l\in S$ and $S\subset \mathbb{Z}^+$, both $l \text{ and }l\sqrt{5}\in\mathbb{Z}^+$ which implies $l'\in\mathbb{Z}^+$.
      \item Since $l\in\mathbb{Z}^+$ we have $5l\in\mathbb{Z}^+$ and since $l\sqrt{5}\in\mathbb{Z}^+$ we have $l'\sqrt{5}= (l\sqrt{5}-2l)\sqrt{5}= 5l-2l\sqrt{5}\in\mathbb{Z}^+$.
    \end{itemize}
    It follows that $l'\in S$ but $l' < l$ which contradicts $l$ being the least element in $S$.
  \end{proof}
  \pagebreak

%%%%%%%%%%%%%%%%%%%%%
%%%%% Problem 4 %%%%%
%%%%%%%%%%%%%%%%%%%%%
\item
  Use the identity
  $$\frac1{k^2-1}=\frac12\left(\frac1{k-1}-\frac1{k+1}\right)$$
  to evaluate the following:
  \begin{enumerate}

  \item
    $\inlinesum{k=2}{10}\frac1{k^2-1}$

    \begin{align*}
      \inlinesum{k=2}{10}\frac{1}{k^2-1} &=\inlinesum{k=2}{10}\frac{1}{2}\left(\frac{1}{k-1}-\frac{1}{k+1}\right)= \frac{1}{2}\inlinesum{k=2}{10}\left(\frac{1}{k-1}-\frac{1}{k+1}\right) \\
      &= \frac{1}{2}\left[\left(\frac{1}{1}-\frac{1}{3}\right) + \left(\frac{1}{2}-\frac{1}{4}\right) + \left(\frac{1}{3}-\frac{1}{5}\right) + \cdots + \left(\frac{1}{8}-\frac{1}{10}\right) + \left(\frac{1}{9}-\frac{1}{11}\right)\right] \\
      &= \frac{1}{2}\left[\frac{1}{1} + \frac{1}{2} - \frac{1}{10} - \frac{1}{11}\right] \\
      &= \frac{1}{2}\left(\frac{72}{55}\right) = \frac{36}{55}
    \end{align*}

  \item
    $\inlinesum{k=2}{n}\frac1{k^2-1}$

    $$\inlinesum{k=2}{n}\frac1{k^2-1}= \frac{1}{2}\left[\frac{1}{1}+\frac{1}{2} - \frac{1}{n} - \frac{1}{n+1}\right]$$

  \item
    $\inlinesum{k=1}{n}\frac1{k^2+2k}$
    \hspace{10pt}Hint: $k^2+2k=(???)^2-1$

    $$\inlinesum{k=1}{n} \frac{1}{k^2+2k} = \inlinesum{k=1}{n}\frac{1}{(k+1)^2 - 1} = \inlinesum{k=2}{n+1}\frac{1}{k^2 - 1}$$
    $$\inlinesum{k=2}{n+1}\frac{1}{k^2 - 1} = \frac{1}{2} \left[\frac{1}{1} + \frac{1}{2} - \frac{1}{n+1} - \frac{1}{n+2}\right]$$

  \end{enumerate}

%%%%%%%%%%%%%%%%%%%%%
%%%%% Problem 5 %%%%%
%%%%%%%%%%%%%%%%%%%%%  
\item
  Find the value of each of the following:

  \begin{enumerate}

  \item
    $\inlineprod{j=2}{7}\left(1-\frac1j\right)$

    \begin{align*}
      \inlineprod{j=2}{7}\left(1-\frac1j\right) &= \left[\left(1-\frac{1}{2}\right)\cdot \left(1-\frac{1}{3}\right)\cdot \left(1-\frac{1}{4}\right)\cdot \left(1-\frac{1}{5}\right)\cdot \left(1-\frac{1}{6}\right)\cdot \left(1-\frac{1}{7}\right)\right] \\
    &= \left[\frac{1}{2}\cdot\frac{2}{3}\cdot\frac{3}{4}\cdot\frac{4}{5}\cdot\frac{5}{6}\cdot\frac{6}{7}\right] \\
    &= \frac{1}{7}
    \end{align*}
    
  \item
    $\inlineprod{j=2}{n}\left(1-\frac1j\right)$

    $$\inlineprod{j=2}{n}\left(1-\frac1j\right) = \frac{1}{n}$$

  \item
    $\inlineprod{j=2}{n}\left(1-\frac1{j^2}\right)$
    \hspace{10pt}Hint: Be sneaky!

    $$\inlineprod{j=2}{n}\left(1-\frac1{j^2}\right) = \frac{n+1}{2n}$$

  \end{enumerate}

%%%%%%%%%%%%%%%%%%%%%
%%%%% Problem 6 %%%%%
%%%%%%%%%%%%%%%%%%%%%
\item
  Use weak mathematical induction to prove that
  $$\inlinesum{j=1}{n}j(j+1)=\frac{n(n+1)(n+2)}3$$
  for every positive integer $n$.
  \begin{proof}
    $ $
    \begin{enumerate}
      \item[] \textbf{Base Case:}
        \begin{enumerate}
          \item[] Let $n=1$, $\sum_{j=1}^{1} j(j+1) = 2$ and $\frac{1(1+1)(1+2)}{3} = 2$, so the
          base case is valid.
        \end{enumerate}
      \item[] \textbf{Inductive Hypothesis:}
        \begin{enumerate}
          \item[] Assume from the inductive hypothesis that the conclusion is true for some $n$.\\
          This implies that $\sum_{j=1}^{n} j(j+1) = \frac{n(n+1)(n+2)}{3}$.
        \end{enumerate}
      \item[] \textbf{Inductive Step:}
        \begin{enumerate}
          \item[] Then consider the sum to $n+1$:
          \begin{align*}
            \sum_{j=1}^{n+1}j(j+1) &= \sum_{j=1}^{n}j(j+1) + (n+1)((n+1)+1) \\
            &= \left[\frac{n(n+1)(n+2)}{3}\right] + (n+1)((n+1)+1) \hspace{5mm}\text{by IH} \\
            &= \frac{1}{3}\left(n(n+1)(n+2) + 3(n+1)(n+2)\right) \\
            &= \frac{1}{3}\left(n^3 + 3n^2 + 2n + 3n^2 + 9n + 6\right) \\
            &= \frac{1}{3}\left(n^3 + 6n^2 + 11n + 6\right) \\
            &= \frac{1}{3}\left((n+1)(n+2)(n+3)\right)
          \end{align*}  
        \end{enumerate}
      \item[] Thus for all $n\geq 1$, $$\sum_{j=1}^{n} j(j+1) = \frac{n(n+1)(n+2)}{3}$$     
    \end{enumerate}
  \end{proof}
  \pagebreak

%%%%%%%%%%%%%%%%%%%%%
%%%%% Problem 7 %%%%%
%%%%%%%%%%%%%%%%%%%%%
\item
  Use Weak Mathematical Induction
  to show that $f_nf_{n+2}=f_{n+1}^2+(-1)^{n+1}$ for all $n\geq 1$.
  \begin{proof}
    $ $
    \begin{enumerate}
      \item[] \textbf{Base Case:}
        \begin{enumerate}
          \item[] Rewrite the statement $f_nf_{n+2}=f_{n+1}^2+(-1)^{n+1}$ to be $f_nf_{n+2}-f_{n+1}^2 =(-1)^{n+1}$. \\
          Let $n=1$, $f_1 f_{1+2} - f_{1+1}^2 = 1\cdot2 - 1 = 1$ and $(-1)^{1+1} = 1$, so the base case is valid.
        \end{enumerate} 
      \item[] \textbf{Inductive Hypothesis:}
        \begin{enumerate}
          \item[] Assume from the inductive hypothesis that the conclusion is true for some $n$.\\
          This implies that $f_nf_{n+2}-f_{n+1}^2 =(-1)^{n+1}$
        \end{enumerate}
      \item[] \textbf{Inductive Step:}
        \begin{enumerate}
          \item[] Then consider the equation to $n+1$:
            \begin{align*}
              f_{(n+1)} f_{(n+1)+2} - f_{(n+1)+1}^2 &= f_{n+1} f_{n+3} - f_{n+2}^2 \\
              &= f_{n+1} \left(f_{n+1} + f_{n+2}\right) - f_{n+2}^2 \\
              &= f_{n+1}^2 + f_{n+1}f_{n+2} - f_{n+2}^2 \\
              &= f_{n+1}^2 + f_{n+2} \left(f_{n+1} - f_{n+2}\right) \\
              &= f_{n+1}^2 + f_{n+2} \left( - f_{n}\right) \\
              &= -\left(f_n f_{n+2} - f_{n+1}^2\right) \\
              &= - (-1)^{n+1} \hspace{5mm}\text{by IH} \\
              &= (-1)^{n+2}
            \end{align*} 
        \end{enumerate}
      \item[] Thus for all $n\geq 1$, $$f_nf_{n+2}-f_{n+1}^2 =(-1)^{n+1}$$
    \end{enumerate}
  \end{proof}
  \pagebreak

%%%%%%%%%%%%%%%%%%%%%
%%%%% Problem 8 %%%%%
%%%%%%%%%%%%%%%%%%%%%  
\item
  Use weak mathematical induction to show that
  a $2^n\times2^n$ chessboard with a corner missing can be tiled
  with pieces shaped like
  \begin{picture}(20,20)
    \put(0,-5){\line(0,1){20}}
    \put(10,-5){\line(0,1){20}}
    \put(20,-5){\line(0,1){10}}
    \put(0,-5){\line(1,0){20}}
    \put(0,5){\line(1,0){20}}
    \put(0,15){\line(1,0){10}}
  \end{picture}
  \,
  for every integer $n\geq 0$.
  \begin{proof}
    $ $
    \begin{enumerate}
      \item[] \textbf{Base Case:}
        \begin{enumerate}
          \item[] Let $n=1$, $2^1 \times 2^1$ is a $2\times2$ chessboard with a corner missing and can 
          be tiled by one tromino, so the base case is valid.
        \end{enumerate} 
      \item[] \textbf{Inductive Hypothesis:}
        \begin{enumerate}
          \item[] Assume from the inductive hypothesis that the conclusion is true for some $n$. This implies that any
          $2^n \times 2^n$ chessboard with a corner missing can be tiled with trominoes.
        \end{enumerate}
      \item[] \textbf{Inductive Step:}
        \begin{enumerate}
          \item[] Then consider a $2^{n+1} \times 2^{n+1}$ chessboard.
          \begin{itemize}
            \item Divide the $2^{n+1}\times 2^{n+1}$ chessboard into four quadrants of size $2^n \times 2^n$.
            \item By the Inductive Hypothesis we know that each $2^n\times 2^n$ has one corner missing.
            \item There are then four empty squares in the $2^{n+1}\times 2^{n+1}$ board.
            \item Rotate each quadrant such that the four empty squares are in the center of the board.
            \item Add another tromino into the board leaving only one empty square.
            \item Rotate the quadrant with the empty square such that the empty square is in the corner of the board.
            \item Therefore the $2^{n+1}\times 2^{n+1}$ chessboard can be tiled by trominoes with a corner missing.
          \end{itemize} 
        \end{enumerate}
      \item[] Thus, every $2^n \times 2^n$ chessboard with a corner missing can be tiled with trominoes.
    \end{enumerate}    
  \end{proof}
  \pagebreak

%%%%%%%%%%%%%%%%%%%%%
%%%%% Problem 9 %%%%%
%%%%%%%%%%%%%%%%%%%%%
\item
  Define:
  $$H_{2^n}=\inlinesum{j=1}{2^n}\frac1j$$
  Use weak mathematical induction to prove that
  for all $n\geq 1$ we have $H_{2^n}\leq 1+n$.
  \begin{proof}
    $ $
    \begin{enumerate}
      \item[] \textbf{Base Case:}
        \begin{enumerate}
          \item[] Let $n=1$, $H_{2^1}=\inlinesum{j=1}{2^n}\frac1j = \frac32$ and $\frac32 \leq 2$, so the base case is valid.
        \end{enumerate} 
      \item[] \textbf{Inductive Hypothesis:}
        \begin{enumerate}
          \item[] Assume from the inductive hypothesis that the conclusion is true for some $n$. \\
          This implies that $\inlinesum{j=1}{2^n}\frac1j \leq 1+n$.
        \end{enumerate}
      \item[] \textbf{Inductive Step:}
        \begin{enumerate}
          \item[] Then consider the equation to $n+1$:
            \begin{align*}
              H_{2^{n+1}} &= \sum_{j=1}^{2^{n+1}} \frac{1}{j} \\
              &= \sum_{j=1}^{2^n} \frac{1}{j} + \sum_{j=2^n +1}^{2^{n+1}} \frac{1}{j} \\
              &\leq \left[1+n\right] + \sum_{j=2^n +1}^{2^{n+1}} \frac{1}{j} \hspace{5mm}\text{by IH} \\
              &\leq \left[1+n\right] + \frac{1}{2^n +1}+\cdots +\frac{1}{2^{n+1}}\\
              &\leq \left[1+n\right] + 2^n\cdot \frac{1}{2^{n+1}} \\
              &\leq 1 + n + \frac{1}{2} \\
              &\leq \frac{3}{2} + n \leq 2 + n
            \end{align*}
        \end{enumerate}
      \item[] Thus for all $n\geq 1$, $$H_{2^n}\leq 1+n$$
    \end{enumerate}
  \end{proof}
  \pagebreak

%%%%%%%%%%%%%%%%%%%%%
%%%%% Problem 10 %%%%
%%%%%%%%%%%%%%%%%%%%%
\item
  Use strong mathematical induction to prove that every amount of
  postage over $53$ cents can be formed using $7$-cent and $10$-cent stamps.
  \begin{proof}
    $ $
    \begin{enumerate}
      \item[] \textbf{Inductive Step:}
        \begin{enumerate}
          \item[]
          Assume we can do $54,\cdots,k$. Because $k-6$ is in the $54,\cdots,k$ we can do $k-6$ then add a $7$-cent stamp.
          $k-6$ is in $54,\cdots,k$ only if $k-6\geq 54 \equiv k\geq60$.\\
          Thus, the inductive step is only valid for $k=60,61,\cdots$ to get to the next $k+1$.
        \end{enumerate} 
      \item[] \textbf{Base Case:}
        \begin{enumerate}
          \item[] Must do $54,55,56,57,58,59,60$ as base cases.
          \begin{align*}
            54 &= 2(7\text{-cent}) + 4(10\text{-cent})\\
            55 &= 5(7\text{-cent}) + 2(10\text{-cent})\\
            56 &= 8(7\text{-cent})\\
            57 &= 1(7\text{-cent}) + 5(10\text{-cent})\\
            58 &= 4(7\text{-cent}) + 3(10\text{-cent})\\
            59 &= 7(7\text{-cent}) + 1(10\text{-cent})\\
            60 &= 6(10\text{-cent})
          \end{align*} 
        \end{enumerate}
    \end{enumerate}
  \end{proof}

\end{enumerate}

\end{document}
