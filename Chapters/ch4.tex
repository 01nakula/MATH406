\documentclass[class=article, crop=false]{standalone}
\usepackage[utf8]{inputenc}
\usepackage{import}
\usepackage[subpreambles=true]{standalone}
\usepackage{graphicx}
\usepackage{amsfonts}
\usepackage{mathrsfs}
\usepackage{mathtools}
\usepackage{enumerate}
\usepackage{fancyhdr}
\usepackage[colorlinks=true,linkcolor=black,anchorcolor=black,citecolor=black,filecolor=black,menucolor=black,runcolor=black,urlcolor=blue]{hyperref}
\usepackage{epsfig,amssymb,amsmath,multicol,tikz,pgfplots,amsthm,enumerate}
\usepackage{xcolor}

\def\naturals{{\mathbb N}}
\def\reals{{\mathbb R}}
\def\complex{{\mathbb C}}
\def\poly{{\mathbb P}}
\def\integers{{\mathbb Z}}
\def\rationals{{\mathbb Q}}
\def\irrationals{{\mathbb I}}
\def\inlinesum#1#2{\overset{#2}{\underset{#1}{\sum}}}
\def\inlineprod#1#2{\overset{#2}{\underset{#1}{\prod}}}

\begin{document}
\setcounter{section}{3}
\section{Congruences}

%%%%%%%%%%%%%%%
%%% Section %%%
%%%%%%%%%%%%%%%
\subsection{Introduction to Congruences}
\rule{\textwidth}{1pt}\\
\begin{enumerate}[1.]
	\item \textbf{Introduction:} Suppose you wished to find $x,y\in\integers$ satisfying $2x^2 -8y=11$. There is no solution
	because no matter what, $2x^2 -8y$ is even and $11$ is odd. What if even/odd does not work...
	what else might? $3x^2-15y=8$, 3 divides the left side but not the right.
	If even/odd or divided by 3 works, there is no guarantee that it works 
	$\underbrace{3x^2-15y=9}_{\text{might work}}$.
	The idea of modular arithmetic formalizes all of this.

	\item \textbf{Definition and Equivalencies:} 
	For $a,b,m\in\integers$ with $m\geq2$ we write $a\equiv b\mbox{ mod }m$ which is
	read as "$a$ and $b$ are congruent modulo $m$." to mean that $m\mid (a-b)$. A few notes on this,
	\begin{itemize}
		\item[-] Equivalent to saying $m\mid (b-a)$.
		\item[-] Equivalent to saying $\exists c\in\integers$ such that $mc=a-b$ or $\exists x\in\integers$
		such that $mc=b-a$ (definition of divisibility).
		\item[-] Equivalent to saying that if we divide $a$ and $b$ by $m$, the remainders are the same.
	\end{itemize}
	\textbf{Ex.} $8\equiv 18\mbox{ mod } 5$ in fact $8\equiv18\equiv3\equiv-2\equiv23\equiv\cdots\mbox{ mod }5$.
	Here with remainder 3. Also note $5\mid (18-8)$ and $5\mid (8-18)$.\\\\
	Even/odd is the same as $m=2$.\\\\
	\textbf{CS Note.} In computer science we often define $\mbox{mod}(a,m)= $ remainder when $a/m=a\%m$.
	It is not uncommon to see $a=b\mbox{ mod }m$ or $a\equiv_m b$ (strongly discouraged).\\\\
	Moving forward, please use $a\equiv b\mbox{ mod }m$.
	
	\item \textbf{Properties:}
	\begin{enumerate}[(a)]
		\item \textbf{Theorem.} Congruence acts like an equals sign in the following sense:
		\begin{enumerate}[(i)]
			\item $a\equiv a\mbox{ mod }m$ (Reflexive).
			\item if $a\equiv b\mbox{ mod }m$ then $b\equiv a\mbox{ mod }m$ (Symmetric).
			\item If $a\equiv b\mbox{ mod }m$ and $b\equiv c\mbox{ mod }m$ then $a\equiv c\mbox{ mod }m$ (Transitivity).
				\begin{proof}
					$a\equiv b\mbox{ mod }m \implies \exists x$ such that $a-b=mx$,
					$b\equiv c\mbox{ mod }m \implies \exists y$ such that $b-c=my$. Then 
					$a-c = (a-b)+(b-c) =mx+my = m(x+y)$ so $m\mid (a-c)$ so $a\equiv c\mbox{ mod }m$.
				\end{proof}
			\item If $a\equiv b\mbox{ mod }m$ and $c\equiv \mbox{ mod }m$ then $a\pm c \equiv b\pm d \mbox{ mod }m$.
				\begin{itemize}
					\item[i.e.] If we know $x\equiv y\mbox{ mod } 5$ we can conclude $x+7\equiv y+7\mbox{ mod }5$ and also
						$x+7 \equiv y+12 \mbox{ mod }5$.  
				\end{itemize}
			\item If $a\equiv b\mbox{ mod }m$ and $c\equiv d\mbox{ mod }m$ then $ac\equiv bd\mbox{ mod }m$
				\begin{itemize}
					\item[i.e.] If we know $x\equiv y\mbox{ mod }5$ then we can conclude $17x\equiv 17y\mbox{ mod }5$
						but we can also conclude $17x\equiv 12y\mbox{ mod }5$  
				\end{itemize}
			\item If $a\equiv b\mbox{ mod }m$ and $k\in\integers, k\geq 1$ then $a^k \equiv b^k \mbox{ mod }m$.
				(Note: we can \emph{not} use different powers!)
		\end{enumerate}

		\item \textbf{Division Issues.} First everything must be an integer, so does
		$2\equiv 8\mbox{ mod }6\implies \frac{2}{3} \equiv \frac{8}{3} \mbox{ mod }6$ this is garbage because $\frac{2}{3}, \frac{8}{3}\notin\integers$.
		However, is $2\equiv 8\mbox{ mod }6\implies \frac{2}{2}\equiv \frac{8}{2}\mbox{ mod }6$ true? No! because
		$1\equiv 4\mbox{ mod }6$ is not true. The point is even if division makes both sides integers there is no guarantee
		that the congruence is preserved!\\\\
		\textbf{Theorem.} Suppose we have $ac \equiv bc\mbox{ mod } m$ then $a\equiv b\mbox{ mod } m/\mbox{gcd}(m,c)$.
		In other words we may cancel an integer from both sides provided we divide the modulus by the gcd of the
		modulus and the integer we're canceling.
		\begin{proof}
			Suppose $ac\equiv bc\mbox{ mod }m$, $\exists k\in\integers$ with $mk=ac-bc$. So
			$mk=c(b-a)$, $$\frac{m}{\mbox{gcd}(c,m)}k = \frac{c}{\mbox{gcd}(c,m)}(a-b)$$
			Note that from a previous theorem we know that: $$\mbox{gcd}\left(\frac{m}{\mbox{gcd}(c,m)},\frac{c}{\mbox{gcd}(c,m)}\right) = 1$$
			Then the above statement says that $\frac{m}{\mbox{gcd}(c,m)} \Big| \frac{c}{\mbox{gcd}(c,m)}(a-b)$
			which implies $\frac{m}{\mbox{gcd(c,m)}}\Big|a-b$.
			Therefore, $a\equiv b\mbox{ mod }\frac{m}{\mbox{gcd}(c,m)}$.
		\end{proof}
		\noindent\textbf{Note.} Don't think division, think cancelation when dealing with modulo.\\
		\textbf{Ex.} If we know that $4x\equiv 8y\mbox{ mod }50$ then we can conclude that\\
		$x\equiv 2y\mbox{ mod }50/\mbox{gcd}(50,4)$ and so $x\equiv 2y\mbox{ mod }25$ (think
		\emph{cancel} the 4).\\
		\textbf{Corollary.} If $ac \equiv bc\mbox{ mod }m$ and $\mbox{gcd}(c,m)=1$ then $a\equiv b\mbox{ mod }m$.\\
		\textbf{Ex.} $15x \equiv 20y\mbox{ mod }27$, note that $\mbox{gcd}(5,27)=1$ so we may cancel the $5$. So
		$3x\equiv 4y\mbox{ mod }27$.
	\end{enumerate}
	
	\item \textbf{Residue Classes:}
	\begin{enumerate}[(a)]
		\item \textbf{Introduction:} Suppose we are working $\mbox{ mod }m=5$. We know $0\equiv 5\equiv10\equiv-5\equiv\cdots\mbox{ mod }5$,
		we also know $1\equiv6\equiv11\equiv-4\equiv\cdots\mbox{ mod }5$, all of $\integers$ fall into one out of
		$m=5$ classes.
		\begin{align*}
			\{\cdots,-15,-10,-5,&0,5,10,15,\cdots\}\\
			\{\cdots,-16,-9,-4,&1,6,11,16,\cdots\}\\
			\{\cdots, -13,-8,-3,&2,7,12,17,\cdots\}\\
			\{\cdots,-12,-7,-2,&3,8,13,18,\cdots\}\\
			\{\cdots, -11,-6,-1,&4,9,14,19,\dots\}
		\end{align*}

		\item \textbf{Definition.} For a given $m\geq 2$ there are $m$ congruence classes.
		
		\item \textbf{Definition.} From each we may pick a representative of the class so those would be $m$
		representatives.\\ \textbf{Ex.} $m=5: \{0,1,2,3,4\}$ (the obvious one) or you could use $m=5: \{0,2,4,6,8\}$ (all even) 
		or $m=5:\{0,2,4,8,16\}$ (all powers of 2, except 0).\\
		\textbf{Ex.} $m=5: \{0,1,2,3,4\}$ (the obvious one) or you could use $m=5: \{0,2,4,6,8\}$ (all even) 
		or $m=5:\{0,2,4,8,16\}$ (all powers of 2, except 0).
		
		\item \textbf{Definition.} The set of representatives $\{0,\cdots,m-1\}=$ the complete set of least
		non-negative residues.

		\item[] 
		\noindent In $\reals$, $17^x = 48246319\implies x=\log_17(48246319)$. Now consider $\integers\mbox{ mod }100$,
		$6^x\equiv 88\mbox{ mod }100$ is \emph{significantly} harder to solve (the discrete logarithm problem).

		\item \textbf{Definition.} A complete set of residues (CSOR) $\mbox{ mod }m$ is a set of $m$ integers, no two of which
		are congruent mod $m$.\\
		\textbf{Ex.} $m=5$: here are 3 CSORs: $\{0,1,2,3,4\}$, $\{0,2,4,6,8\}$, $\{0,2,4,8,16\}$, and more!

		\item \textbf{Theorem.} A subset $S$ of $\integers$ is a CSOR mod $m$ if and only if every integer is
		congruent to exactly one element in $S$.\\
		\noindent\textbf{Ex.} $m=4$: $S=\{0,9,14,3\}$ some observations:
		\begin{itemize}
			\item $m=4$ of them.
			\item No two are congruent to each other.
			\item Any $a\in\integers$ is congruent to exactly one of these.
		\end{itemize}

		\item \textbf{Theorem.} If $\{r_1,r_2,\cdots, r_m\}$ is a CSOR mod $m$ and if $a,b\in\integers$ with 
		gcd($a,m)=1$ then $\{ar_1+b, ar_2+b,\cdots, ar_m+b\}$ if also a CSOR mod $m$.
		\begin{proof}
			We will show that no two are congruent mod $m$. Suppose $ar_i+b\equiv ar_j+b\mbox{ mod }m$ with
			$i\neq j$. Then $ar_i\equiv ar_j\mbox{ mod }m \implies r_i\equiv r_j\mbox{ mod }m$ because
			gcd$(a,m)=1$. Contradiction because the $r_i, r_j$ came from a CSOR mod $m$.
		\end{proof}
		\noindent\textbf{Ex.} $\{0,1,2,3,4\}$ CSOR mod $5$. Pick $a=9, b=42$,
		$\{0\cdot9 +42,1\cdot9 +42, 2\cdot9 +42, 3\cdot9 +42, 4\cdot9 +42\}$ is also a CSOR mod $5$.
	\end{enumerate}

	\item \textbf{Fast Arithmetic - Fast Exponentiation.} Suppose we wished to calculate $2^{503}\equiv a\mbox{ mod }5$,
	$a=0,1,2,3,4$ but which one? 
	{\color{red} Warning:} Do not reduce exponent mod $5$! $2^{503}\equiv 2^x \mbox{ mod }5$.
	\begin{enumerate}[(a)]
		\item Look for patterns: $2^1 \equiv 2\mbox{ mod }5$, $2^2\equiv 4\mbox{ mod }5$, $2^3\equiv 3\mbox{ mod }5$,
		$2^4\equiv 1\mbox{ mod }5$, $2^5\equiv 2\mbox{ mod }5$. This last one is a repeat, so it repeats
		every 4. Note $503=4(125)+3$ so 
		\begin{align*}
			2^{503} &\equiv2^{4(503)}2^3\\
			&\equiv (1)^{125} 2^3\mbox{ mod }5\\
			&\equiv (1) 8 \mbox{ mod }5\\
			&\equiv 3\mbox{ mod }5
		\end{align*}
		\item Use binary expansions. Suppose we want $3^{81}\equiv a\mbox{ mod }5$.
		$3^1\equiv 3$, $3^2\equiv 4$, $3^4\equiv 1$, $3^8\equiv 1$, $3^{16}\equiv 1$,
		$3^{32}\equiv 1$, $3^{64}\equiv 1$. Then $81=64+16+1$ so 
		\begin{align*}
			3^{81}&=3^{64}3^{16}3^1 \\
			&\equiv 1\cdot 1\cdot 3 \\
			&\equiv 3\mbox{ mod }5
		\end{align*}
	\end{enumerate}
\end{enumerate}

%%%%%%%%%%%%%%%
%%% Section %%%
%%%%%%%%%%%%%%%
\subsection{Solving Linear Congruences}
\rule{\textwidth}{1pt}\\
\begin{enumerate}[1.]
	\item \textbf{Introduction:} The idea is that we would ideally like to solve "equations" like
	$3x^2+x\equiv 5\mbox{ mod }72$, $8^x\equiv 12\mbox{ mod }5$, etc... So let's go back to basics.\\
	\textbf{Definition:} A linear congruence has the form $ax\equiv b\mbox{ mod }m$. We would like to
	find all possible solutions, whatever that means.\\
	\textbf{Process:}
	\begin{enumerate}[(a)]
		\item Do solutions exist?
		\item If so, can we find just one?
		\item Can we find more?
		\item When are they "different"
	\end{enumerate}
	
	\item \textbf{Do Solutions Exist:} To say that $ax\equiv b\mbox{ mod }m$ has a solution means,
	$\exists x$ such that $ax\equiv b\mbox{ mod }m$ which in turn means $\exists x, \exists y$ such
	that $ax+my=b$ ($ax\equiv b\mbox{ mod }m \implies m \mid (ax-b)\implies my=ax-b \implies ax-my=b$).
	This means that $b$ is a linear combination of $a,m$.\\
	\textbf{Recall:} \{Linear combination of $a,m\} = \{$ multiples of gcd$(a,m)\}$.\\
	Thus, $b$ is a linear combination of $a,m$ when $b=$multiple of gcd$(a,m)$, so 
	$ax\equiv b\mbox{ mod }m$ has solution(s) if and only if gcd($a,m)\mid b$.\\
	\textbf{Ex.} $2x\equiv 8\mbox{ mod }18$ has solutions, because gcd(2,18)=$2\mid8$.\\
	$6x\equiv 8\mbox{ mod }36$ does not, because gcd(6,36)=$6\nmid 8$.
	
	\item \textbf{Finding One Solution:} We would like to solve $ax+my=b$, with $b$ as a multiple of
	gcd($a,m$). Well, we can solve $ax'+my'=\mbox{gcd}(a,m)$! But how? With the Euclidean Algorithm.
	Use the Euclidean Algorithm to solve $ax'+my'=\mbox{gcd}(a,m)$ then multiple both sides to get $b$ on the right.\\
	\textbf{Ex.} Consider $4x\equiv 6\mbox{ mod }50$. We have gcd(4,50)=$2\mid 6$ so solutions exist.
	First we use the Euclidean Algorithm to solve: $$4x'+50y'=2$$
	This gives us $4 \underbrace{(-12)}_{x'}+50\underbrace{(1)}_{y'}=2$, we want to get a $6$ on the right hand side so multiple by $3$.
	So then we get $4 \underbrace{(-36)}_{x}+50\underbrace{(3)}_{y}=6$, so $4(-36)\equiv 6\mbox{ mod }50$.
	Typically, we will use the least non-negative residue (add until you get a non-negative). So here
	the solution is $x_0 = (-36)+50 = 14$.
	
	\item \textbf{Finding All Solutions:} Suppose we have our one solution, $x_0 \implies ax_0 \equiv b\mbox{ mod }m$.
	Suppose now $x$ is another, this implies $ax\equiv b\mbox{ mod }m$. So we subtract the second from the first
	\begin{align*}
		a(x)-a(x_0) &\equiv b-b \mbox{ mod } m\\
		a(x-x_0) &\equiv 0\mbox{ mod } m\\
		x-x_0 &\equiv 0 \mbox{ mod } \frac{m}{\mbox{gcd}(a,m)}
	\end{align*}
	So, $$x=x_0 +k\left(\frac{m}{\mbox{gcd}(a,m)}\right)$$
	{\color{red} Warning!} Solutions must look like this but are all things which look like this actually
	solutions? \\\\
	We would like $ax\equiv b\mbox{ mod }m$.
	\begin{align*}
		ax&\equiv a\left(x_0 +k\left(\frac{m}{\mbox{gcd}(a,m)}\right)\right) \mbox{ mod }m\\
		ax&\equiv \underbrace{ax_0}_{b} + \underbrace{k\left(\frac{m}{\mbox{gcd}(a,m)}\right)}_{\mbox{lcm}} \mbox{ mod }m \\
		ax&\equiv b + k\mbox{lcm}(a,m) \mbox{ mod }m\\
		ax&\equiv b\mbox{ mod }m 
	\end{align*}
	Therefore all solutons can be gained by doing, $x=x_0 +k\left(\frac{m}{\mbox{gcd}(a,m)}\right), \forall k\in\integers$.\\\\
	Lastly, when are they unique mod $m$?\\
	Consider that two of them with $k_1$ and $k_2$ are identical mod $m$ when:
	\begin{align*}
		x_0 + k_1\left(\frac{m}{\mbox{gcd}(a,m)}\right) &\equiv x_0 +k_2\left(\frac{m}{\mbox{gcd}(a,m)}\right) \mbox{ mod }m\\
		k_1\left(\frac{m}{\mbox{gcd}(a,m)}\right) &\equiv k_2\left(\frac{m}{\mbox{gcd}(a,m)}\right) \mbox{ mod }m\\
		k_1 &\equiv k_2\mbox{ mod }\frac{m}{m/\mbox{gcd}(a,m)} \\
		k_1 &\equiv k_2\mbox{ mod }\mbox{gcd}(a,m)
	\end{align*}
	Therefore, it follows that solutions will be congruent mod $m$ when $k$-values are congruent mod gcd($a,m)$.
	So solutions are not congruent mod $m$ by ensuring that the $k$-values are not congruent mod gcd$(a,m)$.
	This can be done using $k=0,1,2,\cdots, \mbox{gcd}(a,m)-1$.

	\item \textbf{Summary Theorem:} The linear congruence $ax\equiv b\mbox{ mod }m$ has solutions
	if and only if gcd$(a,m)\mid b$. If it has solutions then it has gcd($a,m)$ unique solutions mod $m$.
	If $x_0$ is one of those then all are $$x=x_0 + k\cdot \frac{m}{\mbox{gcd}(a,m)}, \text{ for } k=0,1,2,\cdots, \mbox{gcd}(a,m)-1$$
	\textbf{Ex.} $20x\equiv 15\mbox{ mod }65$, gcd(20,65)=$5\mid 15$ so $\exists$5 incongruent solutions mod 65.
	The Euclidean Algorithm gives us a solution $x_0 \equiv 56\mbox{ mod }65$. So all solutions are then
	$$x\equiv 56 + k\cdot \frac{65}{\mbox{gcd}(20,65)}\mbox{ mod }m, \text{ for } k=0,1,2,3,4$$
	$$x\equiv 56+13k\mbox{ mod }65, k=0,1,2,3,4$$
	That is $x\equiv 56,4,17,30,43 \mbox{ mod }65$.\\\\
	\textbf{Note:} If gcd($a,m)=1$ there exists only one solution mod $m$.
	
\end{enumerate}

%%%%%%%%%%%%%%%
%%% Section %%%
%%%%%%%%%%%%%%%
\subsection{The Chinese Remainder Theorem}
\rule{\textwidth}{1pt}\\
\begin{enumerate}[1.]
	\item \textbf{Introduction:}
	How can we solve systems of linear congruences? For example,
	suppose we wished to find $x$ satisfying all of these:
	\begin{align*}
		x &\equiv 2\mod 6 \\
		x &\equiv 4\mod 7 \\
		x &\equiv 3\mod 25
	\end{align*}
	Is it always possible to find a solution to something like this? No!
	However, under certain circumstances, yes!
	
	\item \textbf{Chinese Remainder Theorem:}
	Suppose we have a system of the form
	\begin{align*}
		x &\equiv a_1\mod m_1 \\
		x &\equiv a_2\mod m_2 \\
		&\vdots \\
		x &\equiv a_n\mod m_n
	\end{align*}
	If all the $m_i$ are pairwise coprime (so gcd($m_i, m_j)=1, \forall i, j$), then
	$\exists !$ solution mod $M=m_1 m_2\cdots m_n$. So for our example, since 
	$6,7,25$ are all pairwise coprime, $\exists!$ solution mod $(6)(7)(25)=1050$.
	\begin{proof}
		For each $i$ define $M_i = M/m_i$, then consider the equation:
		$$M_i y_i \equiv 1\mod m_i$$
		Note that $\gcd (M_i, m_i)=1$
		\footnote{Recall: $ax\equiv b\mod m$ solutions if and only if $\gcd(a,m)|b$
		$\exists\gcd(a,m)$ solutions.}. because the $m_i$ are all coprime.
		Since $\gcd(M_i, m_i)=1\mid 1, \exists!$ solution mod $m_i$. Let $y_i$ be that solution.
		Take all $y_i$ and construct the integer:
		$$x=a_1M_1y_1 + a_2M_2y_2+ \cdots + a_nM_ny_n$$
		Claim that this is a solution to the system. Pick some $i$ and observe that
		\begin{align*}
			x &\equiv a_1M_1y_1 + a_2M_2y_2+ \cdots + a_nM_ny_n \mod m_i \\
			&\equiv 0 + 0 +\cdots + a_iM_iy_i + \cdots + 0 \mod m_i \\
			&\left(\text{because } M_i\equiv 0\mod m_i \text{ when } j\neq i\right) \\
			x &\equiv a_i (1) \mod m_i \\
			x &\equiv a_i \mod m_i
		\end{align*}
		Claim $x$ is unique mod $M$. Suppose $x_1, x_2$ are both solutions to the original system.
		$$x_1\equiv a_1\mod m_1\text{ and } x_2\equiv a_1\mod m_1$$
		$$\vdots$$
		$$x_1\equiv a_n\mod m_n\text{ and } x_2\equiv a_n\mod m_n$$
		From here we get,
		\begin{align*}
			x_1 &\equiv x_2\mod m_1 \implies m_1\mid (x_1-x_2) \\
			x_1 &\equiv x_2\mod m_2 \implies m_2\mid (x_1-x_2) \\
			&\vdots \\
			x_1 &\equiv x_2\mod m_n \implies m_n\mid (x_1-x_2)
		\end{align*}
		Since the $m_i$ are all pairwise coprime, we get
		$$m_1 m_2\cdots m_n \mid (x_1 - x_2)$$
		Thus, $x_1\equiv x_2\mod M$.
	\end{proof}

	\item \textbf{Example:} Take a look at
	\begin{align*}
		x &\equiv 2\mod 6 \\
		x &\equiv 4\mod 7 \\
		x &\equiv 3\mod 25
	\end{align*}
	This means that $M=(6)(7)(25)=1050$ and that $M_1 = \frac{1050}{6}=175$, $M_2 = \frac{1050}{7}=150$,
	$M_3 = \frac{1050}{25}=42$.
	\begin{itemize}
		\item[] Solve for $y_1$:
		\begin{align*}
			M_1 y_1 &\equiv 1\mod m_1 \\
			175 y_1 &\equiv 1\mod 6 \\
			1y_1 &\equiv 1\mod 6 \\
			y_1 &= 1
		\end{align*}
		
		\item[] Solve $y_2$: \begin{align*}
			M_2 y_2 &\equiv 1\mod m_2 \\
			150 y_2 &\equiv 1\mod 7 \\
			3y_2 &\equiv 1\mod 7 \\
			y_2 &\equiv 5\mod 7 \\
			y_2 &= 5
		\end{align*}

		\item[] Solve $y_3$: \begin{align*}
			M_3 y_3 &\equiv 1\mod m_3 \\
			42 y_3 &\equiv 1\mod 25 \\
			17y_3 &\equiv 1\mod 25 \\
			y_3 &\equiv 3\mod 25 \\
			y_3 &= 3
		\end{align*}
	\end{itemize}
	Now for the solution,
	\begin{align*}
		x&\equiv(2)(175)(1) + (4)(150)(5) + (3)(42)(3)\mod 1050 \\
		x &\equiv 3728\equiv 578\mod 1050
	\end{align*}
	
\end{enumerate}

%%%%%%%%%%%%%%%
%%% Section %%%
%%%%%%%%%%%%%%%
% \setcounter{subsection}{5}
\subsection{Factoring Using Pollard's Rho Method}
\rule{\textwidth}{1pt}\\
\begin{enumerate}
	\item \textbf{Introduction:} John Pollard invented the Rho factorization algorithm in
	1975. It does a fairly fast job for numbers with small prime factors, even if those numbers
	themsevles are large, it also has a small memory footprint. So it is a useful tool for
	inital probing.

	\item \textbf{Idea:} We have some $n$ and wish to find a factor.
	Suppose $p$ is a prime factor of $n$. The Goal is to look at a sequence of integers
	$x_0,x_1,x_2,\ldots$ until we find two $x_i$ and $x_j$ with the properties that:
	$x_i\not\equiv x_j\mbox{ mod }n$ and $x_i\equiv x_j\mbox{ mod }p$.
	Suppose then, that somehow we obtain such $x_i$ and $x_j$.
	Then observe $p\mid (x_j-x_i)$ and $p\mid n$,
	so then $\gcd(x_j-x_i,n)\geq p$. Note: we can calculate the $\gcd$ easily
	via the Euclidean Algorithm. \\\\
	So the idea will be to generate a sequence $x_0,x_1,x_2,\dots$
	and then check $\gcd(x_j-x_i,n)$ but to do this in a way which is systematic
	and guarantees that eventually we will get $\gcd(x_j-x_i,n)\neq 1$
	which will then give us a factor. Suppose we are given
	$x_0,x_1,x_2,\ldots$ if we consider these mod $p$,
	eventually they repeat since there are only $p$ distinct values mod $p$.
	Once they repeat, they keep repeating. In other words,
	if $\alpha,\beta\geq i$ then $x_{\alpha}\equiv x_{\beta}\mbox{ mod }p$ if and only if
	$(i-j)\mid(\alpha-\beta)$. \\\\
	Suppose $s$ is the smallest multiple of $(j-i)$ which is larger than $i$.
	Observe that since $s,2s\geq i$ and $(j-i)\mid s$, we have $(j-i)\mid(2s-s)$ and so
	$x_{2s}\equiv x_s \mbox{ mod }p$.
	So instead of checking all combinations of $x_i$ and $x_j$, we will just check $x_{2s}$
	and $x_s$ when possible.

	\item \textbf{Pollard's Rho Method:} 
	Generate our $x_0,x_1,x_2,\ldots$ as follows: Let $x_0$ be some starting value, say
	$x_0=2$. Define $f(x)=x^2+1$ and put $x_1=f(x_0)\mbox{ mod }n$ (so $x_1\equiv x_0^2+1\mbox{ mod }n$).
	This function creates a pseudorandom sequence of integers mod $n$.
	Everytime we calculate $x_{2s}$ (even subscript) check $\gcd(x_{2s}-x_s, n)$.
	Eventually, we will get the gcd to be not equal to 1. \\\\
	Thus: The assumption that $n$ has a "small" factor $p$, $p\mid n$,
	suggests that $x_i\equiv x_j\mbox{ mod }p$ fairly quickly
	which then suggests that $\gcd(x_{2s}-x_s,n)\neq1$ also
	fairly quickly. \\\\
	\textbf{Ex.} Let $n=1111$, then set $x_0=2$ and $f(x)=x^2+1$. Then we have,
	\begin{align*}
		x_1&\equiv 2^2+1\equiv 5\mbox{ mod }1111 \\
		x_2&\equiv 5^2+1\equiv 26\mbox{ mod }1111 \implies \gcd(x_2-x_1,n)=\gcd(21,1111)=1 \\
		x_3&\equiv 26^2+1\equiv 677\mbox{ mod }1111 \\
		x_4&\equiv 677^2+1\equiv 598\mbox{ mod }1111 \implies \gcd(x_4-x_2,n)=\gcd(572,1111)=11
	\end{align*}
	So we get $11$ as a factor of $1111$ (no surprise there). \\\\
	\textbf{Ex.} Let $n=1189$, then set $x_0=2$ and $f(x)=x^2+1$. Then we have,
	\begin{align*}
		x_1&\equiv 5 \\
		x_2&\equiv 26 \implies \gcd(26-5, 1189)=1\\
		x_3&\equiv 677 \\
		x_4&\equiv 565 \implies \gcd(565-26, 1189)=1\\
		x_5&\equiv 574 \\
		x_6&\equiv 124 \implies \gcd(124-677,1189)=1\\
		x_7&\equiv 1109 \\
		x_8&\equiv 456 \implies \gcd(456-565,1189)=1\\
		x_9&\equiv 1051 \\
		x_{10}&\equiv 21 \implies \gcd(21-574,1189)=1\\
		x_{11}&\equiv 442 \\
		x_{12}&\equiv 369 \implies \gcd(369-124, 1189)=1\\
		x_{13}&\equiv 616 \\
		x_{14}&\equiv 166 \implies \gcd(166-1109, 1189)=41
	\end{align*}
	So we get $41$ as a factor of $1189$.
\end{enumerate}

%%%%%%%%%%%%%%%
%%% Section %%%
%%%%%%%%%%%%%%%
\subsection{Problems}
\rule{\textwidth}{1pt}\\
\begin{enumerate}
\item
  Calculate the least positive residues modulo 47 of each of
  the following with justification:
  \begin{enumerate}
  \item $2^{543}$
  \item $32^{932}$
  \item $46^{327349287323}$
  \end{enumerate}

\item
  Exhibit a complete set of residues mod 17 composed entirely of
  multiples of 3.

\item
  Show that if $a,b,m\in\integers$ with $m>0$ and if
  $a\equiv b\mod m$ then $\gcd(a,m)=\gcd(b,m)$.

\item
  Suppose $p$ is prime and $x\in\integers$ satisfies
  $x^2\equiv x\mod p$.  Prove that $x\equiv 0\mod p$ or $x\equiv 1\mod p$.
  Show with a counterexample that this fails if $p$ is not prime.

\item
  Show that if $n$ is an odd positive integer or if $n$ is a positive
  integer divisible by $4$ that:
  $$1^3+2^3+...+(n-1)^3\equiv 0\mod n$$

\item
  Find all solutions (mod the given value) to each of the following.
  \begin{enumerate}
  \item $10x \equiv 25\mod 75$
  \item $9x \equiv 8\mod 12$
  \end{enumerate}

\item
  Solve each of the following linear congruences using inverses.
  \begin{enumerate}
  \item $3x \equiv 5\mod 17$
  \item $10x \equiv 3\mod 11$
  \end{enumerate}

\item
  What could the prime factorization of $m$ look like so that
  $6x\equiv 10\mod m$ has at least one solution?  Explain.

\item
  Use the Chinese Remainder Theorem to solve:
  \\A troop of monkeys has a store of bananas.
  When they arrange them into 7 piles, none remain.
  When they arrange them into 10 piles there are 3 left over.
  When they arrange them into 11 piles there are 2 left over.
  What is the smallest positive number of bananas they can have?
  What is the second smallest positive number?

\item
  Solve the system of linear congruences:
  \begin{align*}
	2x+1 & \equiv 3\mod 10 \\
	x+2 & \equiv 7\mod 9 \\
	4x & \equiv 1\mod 7
  \end{align*}

\end{enumerate}

\end{document}