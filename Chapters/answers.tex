\documentclass[class=article, crop=false]{standalone}
\usepackage[utf8]{inputenc}
\usepackage{import}
\usepackage[subpreambles=true]{standalone}
\usepackage{graphicx}
\usepackage{amsfonts}
\usepackage{mathrsfs}
\usepackage{mathtools}
\usepackage{enumerate}
\usepackage{fancyhdr}
\usepackage[colorlinks=true,linkcolor=black,anchorcolor=black,citecolor=black,filecolor=black,menucolor=black,runcolor=black,urlcolor=blue]{hyperref}
\usepackage{epsfig,amssymb,amsmath,multicol,tikz,pgfplots,amsthm,enumerate}
\usepackage{listings}
\usepackage{xcolor}

\definecolor{codegray}{rgb}{0.95,0.95,0.92}

\lstdefinestyle{mystyle}{
    backgroundcolor=\color{codegray},
    breakatwhitespace=false,         
    breaklines=true,                 
    captionpos=b,                    
    keepspaces=true,                 
    numbers=left,                    
    numbersep=5pt,                  
    showspaces=false,                
    showstringspaces=false,
    showtabs=false,                  
    tabsize=4
}

\lstset{style=mystyle}

\def\naturals{{\mathbb N}}
\def\reals{{\mathbb R}}
\def\complex{{\mathbb C}}
\def\poly{{\mathbb P}}
\def\integers{{\mathbb Z}}
\def\rationals{{\mathbb Q}}
\def\irrationals{{\mathbb I}}
\def\E{{\mathcal E}}
\def\inlinesum#1#2{\overset{#2}{\underset{#1}{\sum}}}
\def\inlineprod#1#2{\overset{#2}{\underset{#1}{\prod}}}
\def\ord{{\text{ord}}}
\def\ind{{\text{ind}}}
\def\leg#1#2{\left(\frac{#1}{#2}\right)}

\begin{document}
\setcounter{section}{0}
\section{Solutions}

%%%%%%%%%%%%%%%
%%% Section %%%
%%%%%%%%%%%%%%%
\setcounter{subsection}{0}
\subsection{Chapter 1}
\rule{\textwidth}{1pt}\\
\subsubsection{}
\begin{enumerate}[(a)]
  \item
	Is a subset of $\integers^+$ and therefore is well-ordered.
  
	\item
	There is no least element so the set is not well-ordered.
  
	\item
	Consider the set $\left\{a\,\big|\, a\in\integers,a\geq 10\right\}$, it is apparent that this is a subset of $\integers^+$ and therefore is well-ordered.
	So the set $\left\{\frac a2\,\big|\, a\in\integers,a\geq 10\right\}$ is also well-ordered because it holds a least element ($\frac{10}{5}$).
  
	\item
	There is no least element so the set is not well-ordered.
  \end{enumerate}
\subsubsection{}
Since $S$ is a subset of $\integers^+$ by well-ordering we know that $S$ has a least element, and because $k\in\integers$,
$k$ can be $0$ and therefore there is no most element.
\subsubsection{}
\begin{proof}
	Suppose $\sqrt{5}$ is rational and is of the form $\frac{a}{b}$ where $a,b\in\mathbb{Z}^+$ and $b\neq 0$.
	Consider the set $S$,
	$$S = \left\{ k \mid k, k\sqrt{5}\in\mathbb{Z}^+ \right\}$$
	We know that $S$ is a subset of $\mathbb{Z}^+$ and that $b\in S$, by well-ordering this implies that $S$ has a least element.
	Let $l$ be the least element in $S$.\\
	Consider the properties of $l'$ where $l' = l\sqrt{5}-2l$,
	\begin{itemize}
	  \item $l'=l\sqrt{5}-2l= l(\sqrt{5}-2) \implies 0 < l' < l$.
	  \item Since $l\in S$ and $S\subset \mathbb{Z}^+$, both $l \text{ and }l\sqrt{5}\in\mathbb{Z}^+$ which implies $l'\in\mathbb{Z}^+$.
	  \item Since $l\in\mathbb{Z}^+$ we have $5l\in\mathbb{Z}^+$ and since $l\sqrt{5}\in\mathbb{Z}^+$ we have $l'\sqrt{5}= (l\sqrt{5}-2l)\sqrt{5}= 5l-2l\sqrt{5}\in\mathbb{Z}^+$.
	\end{itemize}
	It follows that $l'\in S$ but $l' < l$ which contradicts $l$ being the least element in $S$.
  \end{proof}
\subsubsection{}
\begin{enumerate}[(a)]
  \item
	\begin{align*}
	  \inlinesum{k=2}{10}\frac{1}{k^2-1} &=\inlinesum{k=2}{10}\frac{1}{2}\left(\frac{1}{k-1}-\frac{1}{k+1}\right)= \frac{1}{2}\inlinesum{k=2}{10}\left(\frac{1}{k-1}-\frac{1}{k+1}\right) \\
	  &= \frac{1}{2}\left[\left(\frac{1}{1}-\frac{1}{3}\right) + \cdots + \left(\frac{1}{8}-\frac{1}{10}\right) + \left(\frac{1}{9}-\frac{1}{11}\right)\right] \\
	  &= \frac{1}{2}\left[\frac{1}{1} + \frac{1}{2} - \frac{1}{10} - \frac{1}{11}\right] \\
	  &= \frac{1}{2}\left(\frac{72}{55}\right) = \frac{36}{55}
	\end{align*}

  \item
	$$\inlinesum{k=2}{n}\frac1{k^2-1}= \frac{1}{2}\left[\frac{1}{1}+\frac{1}{2} - \frac{1}{n} - \frac{1}{n+1}\right]$$

  \item
	$$\inlinesum{k=1}{n} \frac{1}{k^2+2k} = \inlinesum{k=1}{n}\frac{1}{(k+1)^2 - 1} = \inlinesum{k=2}{n+1}\frac{1}{k^2 - 1}$$
	$$\inlinesum{k=2}{n+1}\frac{1}{k^2 - 1} = \frac{1}{2} \left[\frac{1}{1} + \frac{1}{2} - \frac{1}{n+1} - \frac{1}{n+2}\right]$$

  \end{enumerate}
\subsubsection{}
\begin{enumerate}[(a)]

  \item
	$\inlineprod{j=2}{7}\left(1-\frac1j\right)
	= \left[\frac{1}{2}\cdot\frac{2}{3}\cdot\frac{3}{4}\cdot\frac{4}{5}\cdot\frac{5}{6}\cdot\frac{6}{7}\right]
	= \frac{1}{7}$
	
  \item
	$\inlineprod{j=2}{n}\left(1-\frac1j\right) = \frac{1}{n}$

  \item
	$\inlineprod{j=2}{n}\left(1-\frac1{j^2}\right) = \frac{n+1}{2n}$

  \end{enumerate}
\subsubsection{}
\begin{proof}
	$ $
	\begin{enumerate}
	  \item[] \textbf{Base Case:}
		\begin{enumerate}
		  \item[] Let $n=1$, $\sum_{j=1}^{1} j(j+1) = 2$ and $\frac{1(1+1)(1+2)}{3} = 2$, so the
		  base case is valid.
		\end{enumerate}
	  \item[] \textbf{Inductive Hypothesis:}
		\begin{enumerate}
		  \item[] Assume from the inductive hypothesis that the conclusion is true for some $n$.\\
		  This implies that $\sum_{j=1}^{n} j(j+1) = \frac{n(n+1)(n+2)}{3}$.
		\end{enumerate}
	  \item[] \textbf{Inductive Step:}
		\begin{enumerate}
		  \item[] Then consider the sum to $n+1$:
		  \begin{align*}
			\sum_{j=1}^{n+1}j(j+1) &= \sum_{j=1}^{n}j(j+1) + (n+1)((n+1)+1) \\
			&= \left[\frac{n(n+1)(n+2)}{3}\right] + (n+1)((n+1)+1) \text{ by IH} \\
			&= \frac{1}{3}\left(n(n+1)(n+2) + 3(n+1)(n+2)\right) \\
			&= \frac{1}{3}\left(n^3 + 3n^2 + 2n + 3n^2 + 9n + 6\right) \\
			&= \frac{1}{3}\left(n^3 + 6n^2 + 11n + 6\right) \\
			&= \frac{1}{3}\left((n+1)(n+2)(n+3)\right)
		  \end{align*}  
		\end{enumerate}
	  \item[] Thus for all $n\geq 1$, $$\sum_{j=1}^{n} j(j+1) = \frac{n(n+1)(n+2)}{3}$$     
	\end{enumerate}
  \end{proof}
\subsubsection{}
\begin{proof}
	$ $
	\begin{enumerate}
	  \item[] \textbf{Base Case:}
		\begin{enumerate}
		  \item[] Rewrite the statement $f_nf_{n+2}=f_{n+1}^2+(-1)^{n+1}$ to be $f_nf_{n+2}-f_{n+1}^2 =(-1)^{n+1}$. \\
		  Let $n=1$, $f_1 f_{1+2} - f_{1+1}^2 = 1\cdot2 - 1 = 1$ and $(-1)^{1+1} = 1$, so the base case is valid.
		\end{enumerate} 
	  \item[] \textbf{Inductive Hypothesis:}
		\begin{enumerate}
		  \item[] Assume from the inductive hypothesis that the conclusion is true for some $n$.\\
		  This implies that $f_nf_{n+2}-f_{n+1}^2 =(-1)^{n+1}$
		\end{enumerate}
	  \item[] \textbf{Inductive Step:}
		\begin{enumerate}
		  \item[] Then consider the equation to $n+1$:
			\begin{align*}
			  f_{(n+1)} f_{(n+1)+2} - f_{(n+1)+1}^2 &= f_{n+1} f_{n+3} - f_{n+2}^2 \\
			  &= f_{n+1} \left(f_{n+1} + f_{n+2}\right) - f_{n+2}^2 \\
			  &= f_{n+1}^2 + f_{n+1}f_{n+2} - f_{n+2}^2 \\
			  &= f_{n+1}^2 + f_{n+2} \left(f_{n+1} - f_{n+2}\right) \\
			  &= f_{n+1}^2 + f_{n+2} \left( - f_{n}\right) \\
			  &= -\left(f_n f_{n+2} - f_{n+1}^2\right) \\
			  &= - (-1)^{n+1} \hspace{5mm}\text{by IH} \\
			  &= (-1)^{n+2}
			\end{align*} 
		\end{enumerate}
	  \item[] Thus for all $n\geq 1$, $$f_nf_{n+2}-f_{n+1}^2 =(-1)^{n+1}$$
	\end{enumerate}
  \end{proof}
\subsubsection{}
\begin{proof}
	$ $
	\begin{enumerate}
	  \item[] \textbf{Base Case:}
		\begin{enumerate}
		  \item[] Let $n=1$, $2^1 \times 2^1$ is a $2\times2$ chessboard with a corner missing and can 
		  be tiled by one tromino, so the base case is valid.
		\end{enumerate} 
	  \item[] \textbf{Inductive Hypothesis:}
		\begin{enumerate}
		  \item[] Assume from the inductive hypothesis that the conclusion is true for some $n$. This implies that any
		  $2^n \times 2^n$ chessboard with a corner missing can be tiled with trominoes.
		\end{enumerate}
	  \item[] \textbf{Inductive Step:}
		\begin{enumerate}
		  \item[] Then consider a $2^{n+1} \times 2^{n+1}$ chessboard.
		  \begin{itemize}
			\item Divide the $2^{n+1}\times 2^{n+1}$ chessboard into four quadrants of size $2^n \times 2^n$.
			\item By the Inductive Hypothesis we know that each $2^n\times 2^n$ has one corner missing.
			\item There are then four empty squares in the $2^{n+1}\times 2^{n+1}$ board.
			\item Rotate each quadrant such that the four empty squares are in the center of the board.
			\item Add another tromino into the board leaving only one empty square.
			\item Rotate the quadrant with the empty square such that the empty square is in the corner of the board.
			\item Therefore the $2^{n+1}\times 2^{n+1}$ chessboard can be tiled by trominoes with a corner missing.
		  \end{itemize} 
		\end{enumerate}
	  \item[] Thus, every $2^n \times 2^n$ chessboard with a corner missing can be tiled with trominoes.
	\end{enumerate}    
  \end{proof}
\subsubsection{}
\begin{proof}
	$ $
	\begin{enumerate}
	  \item[] \textbf{Base Case:}
		\begin{enumerate}
		  \item[] Let $n=1$, $H_{2^1}=\inlinesum{j=1}{2^n}\frac1j = \frac32$ and $\frac32 \leq 2$, so the base case is valid.
		\end{enumerate} 
	  \item[] \textbf{Inductive Hypothesis:}
		\begin{enumerate}
		  \item[] Assume from the inductive hypothesis that the conclusion is true for some $n$. \\
		  This implies that $\inlinesum{j=1}{2^n}\frac1j \leq 1+n$.
		\end{enumerate}
	  \item[] \textbf{Inductive Step:}
		\begin{enumerate}
		  \item[] Then consider the equation to $n+1$:
			\begin{align*}
			  H_{2^{n+1}} &= \sum_{j=1}^{2^{n+1}} \frac{1}{j} \\
			  &= \sum_{j=1}^{2^n} \frac{1}{j} + \sum_{j=2^n +1}^{2^{n+1}} \frac{1}{j} \\
			  &\leq \left[1+n\right] + \sum_{j=2^n +1}^{2^{n+1}} \frac{1}{j} \hspace{5mm}\text{by IH} \\
			  &\leq \left[1+n\right] + \frac{1}{2^n +1}+\cdots +\frac{1}{2^{n+1}}\\
			  &\leq \left[1+n\right] + 2^n\cdot \frac{1}{2^{n+1}} \\
			  &\leq \frac{3}{2} + n \leq 2 + n
			\end{align*}
		\end{enumerate}
	  \item[] Thus for all $n\geq 1$, $$H_{2^n}\leq 1+n$$
	\end{enumerate}
  \end{proof}
\subsubsection{}
\begin{proof}
	$ $
	\begin{enumerate}
	  \item[] \textbf{Inductive Step:}
		\begin{enumerate}
		  \item[]
		  Assume we can do $54,\cdots,k$. Because $k-6$ is in the $54,\cdots,k$ we can do $k-6$ then add a $7$-cent stamp.
		  $k-6$ is in $54,\cdots,k$ only if $k-6\geq 54 \equiv k\geq60$.
		  Thus, the inductive step is only valid for $k=60,61,\cdots$ to get to the next $k+1$.
		\end{enumerate} 
	  \item[] \textbf{Base Case:}
		\begin{enumerate}
		  \item[] Must do $54,55,56,57,58,59,60$ as base cases.
		  \begin{align*}
			54 &= 2(7\text{-cent}) + 4(10\text{-cent})\\
			55 &= 5(7\text{-cent}) + 2(10\text{-cent})\\
			56 &= 8(7\text{-cent})\\
			57 &= 1(7\text{-cent}) + 5(10\text{-cent})\\
			58 &= 4(7\text{-cent}) + 3(10\text{-cent})\\
			59 &= 7(7\text{-cent}) + 1(10\text{-cent})\\
			60 &= 6(10\text{-cent})
		  \end{align*} 
		\end{enumerate}
	\end{enumerate}
  \end{proof}

%%%%%%%%%%%%%%%
%%% Section %%%
%%%%%%%%%%%%%%%
\setcounter{subsection}{2}
\subsection{Chapter 3}
\rule{\textwidth}{1pt}\\
\begin{enumerate}
\item
 Use the Euclidean Algorithm to calculate $d=\mbox{gcd}(510,140)$
  and then use the result to find $\alpha$ and $\beta$ so
  that $d=510\alpha+140\beta$.\hfill\fbox{10/10}

  Need to find gcd$(510,140)$.
  \begin{align*}
	510 &= 3(140) + 90 \\
	140 &= 1(90) + 50 \\
	90 &= 1(50) + 40 \\
	50 &= 1(40) + 10 \\
	40 &= 4(10) + 0
  \end{align*}

  So the gcd is 10. Now to find the linear combination.
  \begin{align*}
	10 &= 1(50) - 1(40) \\
	&= 1(50) - 1(90-1(50)) \\
	&= 2(50) - 1(90) \\
	&= 2(140-1(90)) - 1(90) \\
	&= 2(140) -3(90) \\
	&= 2(140) -3(510-3(140)) \\
	&= -3(510) +11(140) \\
	&= \alpha a+ \beta b
  \end{align*}
  where $\alpha = -3$ and $\beta = 11$.

\item
  Use the Euclidean Algorithm to show that if $k\in\integers^+$ that
  $3k+2$ and $5k+3$ are relatively prime.\hfill\fbox{8/10}

  Need to show that gcd$(3k+2, 5k+3)=1$ for all $k\in\integers^+$.
  \begin{align*}
	5k+3 &= 1(3k+2) + (2k+1) \\
	3k+2 &= 1(2k+1) + (k+1) \\
	2k+1 &= 1(k+1) + k \\
	k+1 &= 1(k) + 1
  \end{align*}
  So the gcd$(3k+2,5k+3)=1$, therefore $3k+2$ and $5k+3$ are relatively prime.

\item
  How many zeros are there at the end of $(1000!)$?  Do not do this
  by brute force.  Explain your method.\hfill\fbox{10/10}

  Zeros at the end of numbers are from multiples of 10 which are pairs of 2 and 5, so
  we find the number of pairs of 2's and 5's to find the number of zeros. Let $d_n(x)$
  represent the sum of the numbers divisible by all powers of $n$ less than $x$. 
  $$d_2(1000!) = 500 + 250 + 125 + 62 + 31 + 15 + 7 + 3 + 1 = 994$$
  $$d_5(1000!) = 200 + 40 + 8 + 1 = 249$$
  Thus, there can only be 249 pairs of 2's and 5's, so there are only 249 
  10's, so there are 249 zeros at the end of $(1000!)$.

\item
  Let $a=1038180$ and $b=92950$.
  First find the prime factorizations of $a$ and $b$.
  Then use these to calculate $\mbox{gcd}(a,b)$ and $\mbox{lcm}(a,b)$.\hfill\fbox{10/10}

  \begin{itemize}
	\item[] Find the prime factorization of $a$.
		\begin{align*}
		  1038180 &= 2^2 (259545) \\
		  &= 2^2 3^1 (86515) \\
		  &= 2^2 3^1 5^1 (17303) \\
		  &= 2^2 3^1 5^1 11^3 (13) \\
		  &= 2^2 3^1 5^1 11^3 13^1
		\end{align*} 
	\item[] Find the prime factorization of $b$. 
		\begin{align*}
		  92950 &= 2^1 (46475) \\
		  &= 2^1 5^2 (1859) \\
		  &= 2^1 5^2 11^1 (169) \\
		  &= 2^1 5^2 11^1 13^2
		\end{align*}
  \end{itemize}
  Now, to find the gcd($a,b$) and lcm($a,b$).
  $$\text{gcd}(a,b) = \text{gcd}(2^2 3^1 5^1 11^3 13^1, 2^1 5^2 11^1 13^2) = 2^1 5^1 11^1 13^1 = 1430$$
  $$\text{lcm}(a,b) = \text{lcm}(2^2 3^1 5^1 11^3 13^1, 2^1 5^2 11^1 13^2) = 2^2 3^1 5^2 11^3 13^2 = 67481700$$

\item
  Which pairs of integers have gcd of 18 and lcm of 540?  Explain.\hfill\fbox{10/10}

  \begin{itemize}
	\item[] Find the prime factorization of 18.
		\begin{align*}
		  18 &= 2^1 (9) \\
		  &= 2^1 3^2
		\end{align*} 
	\item[] Find the prime factorization of 540. 
		\begin{align*}
		  540 &= 2^2 (135) \\
		  &= 2^2 3^3 (5) \\
		  &= 2^2 3^3 5^1
		\end{align*}
  \end{itemize}
  
  From the prime factors of 18 and 540 we know that $x = 2^a 3^b 5^c$ and $y = 2^e 3^f 5^g$. The gcd is the
  minimum power of common prime factors, similarly the lcm is the maximum power of common prime factors. Therefore,
  the list of all possible pairs of integers is:
  \begin{align*}
	x=2^1 3^2 5^0 &, y=2^2 3^3 5^1 \\
	x=2^1 3^3 5^0 &, y=2^2 3^2 5^1 \\
	x=2^2 3^2 5^0 &, y=2^1 3^3 5^1 \\
	x=2^2 3^3 5^0 &, y=2^1 3^2 5^1
  \end{align*}

\item
  Suppose that $a\in\integers$ is a perfect square
  divisible by at least two distinct primes.
  Show that $a$ has at least seven distinct factors.\hfill\fbox{5/10}

  Since $a$ is a perfect square it can be represented by the form $a=b^2$, and since $a$ has 
  at \emph{least} 2 prime factors we can say that $b=p_1^{\alpha} p_2^{\beta}$. 
  It follows that $a=p_1^{2\alpha} p_2^{2\beta}$. Therefore $a$ has factors
  $1, p_1, p_2, p_1, p_2, p_1^2, p_2^2, a$.

\item
  Show that if $a,b\in\integers^+$ with $a^3\big|b^2$ then $a\big|b$.\hfill\fbox{10/10}

  Let $a= p_1^{\alpha_1} p_2^{\alpha_2}\cdots p_n^{\alpha_n}$ and $b= p_1^{\beta_1} p_2^{\beta_2}\cdots p_n^{\beta_n}$.
  Since $a^3\mid b^2$ we know that, 
  $$p_1^{3\alpha_1} p_2^{3\alpha_2}\cdots p_n^{3\alpha_n} \Big| p_1^{2\beta_1} p_2^{2\beta_2}\cdots p_n^{2\beta_n}$$
  Therefore, $3\alpha_n \leq 2\beta_n$. Now to show $a\mid b$ we need to show that $\alpha \leq \beta$.
  $$3\alpha\leq 2\beta \implies \alpha \leq \frac{2\beta}{3} \leq \beta$$
  Thus, if $a^3\mid b^2$ then $a\mid b$.

\item For which positive integers $m$ is each of the following statements true:\hfill\fbox{6/10}
\begin{enumerate}
\item
  $34\equiv 10 \mod m$ \\
  $$m = 12, 24$$
\item
  $1000\equiv 1 \mod m$ \\
  $$m = 3, 9, 27, 37, 111, 333, 999$$
\item
  $100\equiv 0 \mod m$ \\
  $$m = 1, 2, 4, 5, 10, 20, 25, 50, 100$$
\end{enumerate}

\end{enumerate}

%%%%%%%%%%%%%%%
%%% Section %%%
%%%%%%%%%%%%%%%
\setcounter{subsection}{3}
\subsection{Chapter 4}
\rule{\textwidth}{1pt}\\
\begin{enumerate}
\item
  Calculate the least positive residues modulo 47 of each of
  the following with justification:
  \begin{enumerate}
  \item $2^{543}$ \\\\
  Using binary expansion we see that $2^1 \equiv 2\mbox{ mod } 47$, $2^2 \equiv 4\mbox{ mod } 47$,
  $2^4 \equiv 16\mbox{ mod } 47$, $2^8 \equiv 21\mbox{ mod } 47$, $2^{16} \equiv 18\mbox{ mod } 47$, $2^{32} \equiv 42\mbox{ mod } 47$,
  $2^{64}\equiv 25\mbox{ mod } 47$, $2^{128}\equiv 14\mbox{ mod } 47$, $2^{256}\equiv 8\mbox{ mod } 47$, and $2^{512}\equiv 17\mbox{ mod } 47$.\\
  Then $543 = 512 + 16 + 8 + 4 + 2 + 1$ so,
  \begin{align*}
	2^{543} = 2^{512}2^{16}2^{8}2^{4}2^{2}2^{1} &\equiv \\
	&\equiv 17\cdot18\cdot 21\cdot 16\cdot 4\cdot 2 \mbox{ mod } 47 \\
	&\equiv 822528 \mbox{ mod } 47 \\
	&\equiv 28 \mbox{ mod } 47
  \end{align*}
  So $28$ is the least non-negative residue.
  
  \item $32^{932}$ \\\\
  Using binary expansion we see that $32^1 \equiv 32\mbox{ mod }47$,
  $32^2 \equiv 37\mbox{ mod }47$, $32^4\equiv 6\mbox{ mod }47$, $32^8\equiv 47\mbox{ mod }47$,
  $32^{16}\equiv 27\mbox{ mod }47$, $32^{32}\equiv 24\mbox{ mod }47$, $32^{64}\equiv 12\mbox{ mod }47$,
  $32^{128}\equiv 3\mbox{ mod }47$, $32^{256}\equiv 9$, and $32^{512}\equiv 34$.
  Then $932 = 512 + 256 + 128 + 32 + 4$ so,
  \begin{align*}
	32^{932} = 32^{512}32^{256}32^{128}32^{32}32^4 &\equiv \\
	&\equiv 34\cdot 9\cdot 3\cdot 24\cdot 6 \mbox{ mod }47 \\
	&\equiv 132192 \mbox{ mod }47 \\
	&\equiv 28 \mbox{ mod }47
  \end{align*}
  So $28$ is the least non-negative residue.
  
  \item $46^{327349287323}$ \\\\
  Since $46\equiv -1\mbox{ mod } 47$ we know $46^{327349287323} \equiv (-1)^{327349287323}$.
  We also know $2\nmid 327349287323$ so $(-1)^{327349287323} \equiv (-1)^1 \equiv -1 \equiv 46\mbox{ mod } 47$.
  So $46$ is the least non-negative residue.

  \end{enumerate}

\item
  Exhibit a complete set of residues mod 17 composed entirely of
  multiples of 3. \\\\
  Let $S= \{0,1,2,\cdots,16\}$ be the set of residues mod 17. Because $\gcd(3,17)=1$ 
  the set consisting of only multiples of 3 would be,
  $$\{0,3,6,9,12,15,18,21,24,27,30,33,36,39,42,45,48\}$$

\item
  Show that if $a,b,m\in\integers$ with $m>0$ and if
  $a\equiv b\mbox{ mod } m$ then $\gcd(a,m)=\gcd(b,m)$. \\\\
  If $a\equiv b\mbox{ mod } m$ then $\exists x\in\integers$ such that $a=b+xm$.
  So $\gcd(a,m)=\gcd(b+xm,m)=\gcd(b,m)$.

\item
  Suppose $p$ is prime and $x\in\integers$ satisfies
  $x^2\equiv x\mbox{ mod } p$.  Prove that $x\equiv 0\mbox{ mod } p$ or $x\equiv 1\mbox{ mod } p$.
  Show with a counterexample that this fails if $p$ is not prime. \\\\
  Because $x^2\equiv x\mbox{ mod } p$ we know that $x^2-x\equiv 0\mbox{ mod } p$,
  which is the same as $x(x-1) \equiv 0\mbox{ mod } p$. This implies that
  either $p\mid x$, $p\mid (x-1)$, or both.
  $$p\mid x \implies x\equiv 0\mbox{ mod } p$$
  $$p\mid (x-1) \implies x\equiv 1\mbox{ mod } p$$
  If $p$ is not prime, say $p=6$ we see,
  $$3^2 \equiv 3 \mbox{ mod } 6$$
  Where $3\not\equiv 0\mbox{ mod }6$ and $3\not\equiv 1\mbox{ mod }6$. So $p$ must be prime for the
  statement to hold true.

\item
  Show that if $n$ is an odd positive integer or if $n$ is a positive
  integer divisible by $4$ that:
  $$1^3+2^3+...+(n-1)^3\equiv 0\mbox{ mod } n$$
  There are two cases to look at, when $n$ is an odd positive integer 
  and when $n$ is divisible by $4$.
  \begin{itemize}
	\item If $n$ is an odd positive integer then $n-1$ is even so we have an even amount of
	numbers. Consider the set $S = \{1^3, 2^3, \cdots, (n-1)^3\}$. Then consider
	two subsets of $S$, both with $(n-1)/2$ elements $S_1$ and $S_2$, where 
	$$\sum S = \sum S_1 + \sum S_2 = 1^3+ 2^3 + \cdots + (n-1)^3$$
	The set $S_1 = \{1^3,2^3,3^3,\cdots\}$ and the set $S_2 = \{\cdots, (n-3)^3, (n-2)^3, (n-1)^3\}$.
	Because we know that $a-b \equiv -b \mbox{ mod } a$ we also know that $(a-b)^3\equiv (-b)^3\mbox{ mod } a$.
	So we can say that for all elements in $S_2 \mbox{ mod } n$, $S_2 = \{\cdots, (-3)^3, (-2)^3, (-1)^n\}$.
	Now if we look at $\sum S_1 + \sum S_2 \mbox{ mod } n$ we see that the first element of $S_1$
	is cancelled out by the last element of $S_2$ and so forth until there are no elements left.
	Thus, $1^3+2^3+...+(n-1)^3\equiv 0\mbox{ mod } n$.

	\item If $n$ is divisble by 4 then $n-1$ is odd so we have an odd amount of numbers.
	Consider the set $S = \{1^3, 2^3, \cdots, (n-1)^3\}$. Then consider two subsets of $S$,
	both with $(n-1)/2 - 1$ elements $S_1$ and $S_2$, where
	$$\sum S = \sum S_1 + \left(\frac{n}{2}\right)^3 + \sum S_2 = 1^3+ 2^3 + \cdots + (n-1)^3$$
	The set $S_1 = \{1^3,2^3,3^3,\cdots\}$ and the set $S_2 = \{\cdots, (n-3)^3, (n-2)^3, (n-1)^3\}$.
	Because we know that $a-b \equiv -b \mbox{ mod } a$ we also know that $(a-b)^3\equiv (-b)^3\mbox{ mod } a$.
	So we can say that for all elements in $S_2 \mbox{ mod } n$, $S_2 = \{\cdots, (-3)^3, (-2)^3, (-1)^n\}$.
	Now if we look at $\sum S \mbox{ mod } n$, like in the case above we can see that sets $S_1$ and $S_2$ will
	cancel one another out. This leaves us with
	$$1^3+2^3+...+(n-1)^3\equiv \left(\frac{n}{2}\right)^3\mbox{ mod } n$$
	Because we know that $4\mid n$ we know that $n=4x$ for some $x\in\integers$. It follows that,
	$$\left(\frac{n}{2}\right)^3 = \frac{n^3}{8} = \frac{64x^3}{8} = 8x^3 = (2x^2)n$$
	So $$\left(\frac{n^3}{2}\right)^3 \equiv (2x^2)n \equiv 0\mbox{ mod } n$$
	Thus, $1^3+2^3+...+(n-1)^3\equiv 0\mbox{ mod } n$.
  \end{itemize}

\item
  Find all solutions (mod the given value) to each of the following.
  \begin{enumerate}
  \item $10x \equiv 25 \mbox{ mod } 75$\\\\
  Because the $\gcd(10,75)=5$ and $5\mid 25$ we know that solutions exist.
  Let $x_0 \equiv10\mbox{ mod } 75$, so all solutions are then
  $$x\equiv 10 + k \cdot \frac{75}{\gcd(10,75)}\mbox{ mod } 75,\text{ for } k=0,1,2,3,4$$
  $$x\equiv 10 + 15k\mbox{ mod } 75,\text{ for } k=0,1,2,3,4$$
  Therefore, $x\equiv 10, 25, 40, 55, 70$.

  
  \item $9x \equiv 8\mbox{ mod } 12$\\\\
  Because the $\gcd(9,12)=3$ and $3\nmid 8$ so there are no solutions.
  
  \end{enumerate}

\item
  Solve each of the following linear congruences using inverses.
  \begin{enumerate}
  \item $3x \equiv 5\mbox{ mod } 17$\\\\
  Since $6$ is the inverse of $3\mbox{ mod } 17$ we get, 
  $6\cdot3x\equiv 6\cdot 5\mbox{ mod } 17$ which implies
  $$x\equiv 30\mbox{ mod } 17\equiv 13\mbox{ mod } 17$$
  Therefore, $x\equiv 13$.

  \item $10x \equiv 3\mbox{ mod } 11$\\\\
  Since $10$ is the inverse of $10\mbox{ mod } 11$ we get,
  $10\cdot10x\equiv 10\cdot 3\mbox{ mod } 11$ which implies
  $$x\equiv 30\mbox{ mod } 11\equiv 8\mbox{ mod } 11$$
  Therefore, $x\equiv 8$.

  \end{enumerate}

\item
  What could the prime factorization of $m$ look like so that
  $6x\equiv 10\mbox{ mod } m$ has at least one solution?  Explain.\\\\
  In order for $ax\equiv b\mbox{ mod } m$ to have a solution(s),
  $\gcd(a,m)\mid b$. So in the context of this problem we have that
  $\gcd(6,m)\mid 10$. We are looking for an $m$ such that $\gcd(6,m)=2$.
  One possible $m$ could be $m=2^1$.

\item
  Use the Chinese Remainder Theorem to solve:
  \\A troop of monkeys has a store of bananas.
  When they arrange them into 7 piles, none remain.
  When they arrange them into 10 piles there are 3 left over.
  When they arrange them into 11 piles there are 2 left over.
  What is the smallest positive number of bananas they can have?
  What is the second smallest positive number?\\\\
  Let $x$ be the number of bananas, we have
  \begin{align*}
	x &\equiv 0\mbox{ mod } 7 \\
	x &\equiv 3\mbox{ mod } 10 \\
	x &\equiv 2\mbox{ mod } 11
  \end{align*}
  Test to see if all $m_i$ are pairwise coprime, $\gcd(7,10)=\gcd(7,11)=\gcd(10,11)=1$.
  This means that $M=770$, $M_1=110$, $M_2=77$, and $M_3=70$.
  \begin{itemize}
	\item[] Solve for $y_1$:
	  \begin{align*}
		110y_1 &\equiv 1\mbox{ mod } 7 \\
		5y_1 &\equiv 1\mbox{ mod } 7 \\
		y_1 &= 3
	  \end{align*}

	\item[] Solve for $y_2$:
	  \begin{align*}
		77y_2 &\equiv 1\mbox{ mod } 10 \\
		7y_2 &\equiv 1\mbox{ mod } 10 \\
		y_2 &= 3
	  \end{align*}
	
	\item[] Solve for $y_3$:
	  \begin{align*}
		70y_3 &\equiv 1\mbox{ mod } 11 \\
		4y_3 &\equiv 1\mbox{ mod } 11 \\
		y_3 &= 3
	  \end{align*}
  \end{itemize}
  So we then get 
  $$x = (0)(110)(3) + (3)(77)(3) + (2)(70)(3) \equiv 1113 \mbox{ mod } 770$$
  $$x \equiv 343 \mbox{ mod } 770$$
  The smallest number of bananas they can have is $343$ and the second smallest is $1113$.

\item
  Solve the system of linear congruences:
  \begin{align*}
	2x+1 & \equiv 3\mbox{ mod } 10 \\
	x+2 & \equiv 7\mbox{ mod } 9 \\
	4x & \equiv 1\mbox{ mod } 7
  \end{align*}
  Rewrite the system of linear congruences to be (properties of congruences):
  \begin{align*}
	2x &\equiv 2\mbox{ mod } 10 \\
	x &\equiv 5\mbox{ mod } 9 \\
	4x &\equiv 1\mbox{ mod } 7
  \end{align*}
  Which then becomes
  \begin{align*}
	x &\equiv 1\mbox{ mod } 5 \\
	x &\equiv 5\mbox{ mod } 9 \\
	x &\equiv 2\mbox{ mod } 7
  \end{align*}
  Then test to see if all $m_i$ are pairwise coprime, 
  $\gcd(5,9)=\gcd(5,7)=\gcd(7,9)=1$. This means that
  $M=315$, $M_1=63$, $M_2=35$, and $M_3=45$.
  \begin{itemize}
	\item[] Solve for $y_1$:
	  \begin{align*}
		63y_1 &\equiv 1\mbox{ mod }5 \\
		3y_1 &\equiv 1\mbox{ mod }5 \\
		y_1 &= 2
	  \end{align*}
	\item[] Solve for $y_2$:
	  \begin{align*}
		35y_2 &\equiv 1\mbox{ mod }9 \\
		8y_2 &\equiv 1\mbox{ mod }9 \\
		y_2 &= 8
	  \end{align*}
	\item[] Solve for $y_3$: 
	  \begin{align*}
		45y_3 &\equiv 1\mbox{ mod }7 \\
		3y_3 &\equiv 1\mbox{ mod }7 \\
		y_3 &=5
	  \end{align*}
  \end{itemize}
  So we then get
  $$x=(1)(63)(2) + (5)(35)(8) + (2)(45)(5) \equiv 1976 \mbox{ mod } 315$$
  $$x\equiv 86\mbox{ mod }315$$

\end{enumerate}


%%%%%%%%%%%%%%%
%%% Section %%%
%%%%%%%%%%%%%%%
\setcounter{subsection}{5}
\subsection{Chapter 6}
\rule{\textwidth}{1pt}\\
\begin{enumerate}
\item
  Use Fermat's Little Theorem
  to find the least nonnegative residue of
  $2^{1000003}\mod 17$. \\\\
  Well $17\nmid 2$ so $2^{16} \equiv 1\mbox{ mod }17$. Then $1000003 = 16(62500) + 3$ so
  \begin{align*}
	2^{1000003}= 2^{16^{62500}} 2^3 &\equiv (1)^{62500} 2^{3} \mbox{ mod }17 \\
	&\equiv 2^3\mbox{ mod }17 \\
	&\equiv 8\mbox{ mod }17
  \end{align*}
  So $8$ is the least non-negative residue.

\item
  Use Fermat's Little Theorem to solve the following,
  giving the result as the least nonnegative residue.
  \begin{enumerate}
  \item
	$7x\equiv 12\mod 17$ \\\\
	By FLiT we know $7^{17-1} \equiv 1\mbox{ mod }17$, it follows that
	$7^{16}\cdot 12 \equiv 1\mbox{ mod }17$ therefore $7x = 7^{16}\cdot 12 = 7^{15}\cdot 12$.
	Then reduce $7^{15}\cdot 12\mbox{ mod }17$. So $9$ is the least non-negative residue.

  \item
	$10x\equiv 13\mod 19$ \\\\
	By FLiT we know $10^{19-1} \equiv 1\mbox{ mod }19$, it follows that
	$10^{18}\cdot 13 \equiv 1\mbox{ mod }19$ therefore $10x = 10^{18}\cdot 13 = 10^{17}\cdot 13$.
	Then reduce $10^{17}\cdot 13\mbox{ mod }19$. So $7$ is the least non-negative residue.

\end{enumerate}

\item
  Use Fermat's Little Theorem
  to show that
  $30\big|(n^9-n)$
  for all positive integers $n$.
  \begin{proof}
	Note that $30 = 2\cdot 3\cdot 5$, from Fermat's Little Theorem we know that
	$a^p \equiv a\mbox{ mod }p$ when $p$ is prime.
	Let,
	\begin{align*}
	  x &= (n^5 -n)(n^4 +1) \\
	  y &= (n^3 -n)(n^2 +1)(n^4 +1) \\
	  z &= (n^2 -n)(n^3 +n^2 +n +1) (n^4 +1)
	\end{align*}
	Note that $x = y = z = n^9 -n$. Then observe the following,
	\begin{itemize}
	  \item By FLiT $5\mid n^5 - n \implies 5\mid x \implies 5\mid n^9 - n$.
	  \item By FLiT $3\mid n^3 - n \implies 3\mid y \implies 3\mid n^9 - n$.
	  \item By FLiT $2\mid n^2 - n \implies 2\mid z \implies 2\mid n^9 - n$.
	\end{itemize}
	It follows that $2\cdot 3\cdot 5 \mid n^9 -n \implies 30\mid n^9 -n$.
  \end{proof}

\item
  The definition of $n$ being a Fermat pseudoprime to base $b$
  does not actually require that $\gcd(b,n)=1$
  because it's not possible to have $b^{n-1}\equiv 1\mod n$ with $\gcd(b,n)\neq 1$.
  Prove this.
  \begin{proof}
	Let $\gcd(b,n)=a$ where $a\neq 1$, this implies there exists a prime $p$ such that
	$p\mid b$ and $p\mid n$. It follows that there exists a linear combination of $b$ and $n$
	such that $xb + yn = 1$ for $x,y\in\integers$. But if $p\mid b$ and $p\mid n$ then 
	$p\mid xb+yn = 1 \implies p\mid 1$ but this is a contradiction to the fact that $p$ is prime.
  \end{proof}

\item
  We didn't exclude even integers from the definition of a Fermat Pseudoprime.
  Some books do.
  Show that with our definition
  $4$ is a Fermat Pseudoprime to a certain base. \\\\
  From $5^{4-1}= 5^3 \equiv 1\mbox{ mod }4$, we see that $4$ is a Fermat Pseudoprime to the base $5$.

\item
  Prove that if $n$ is an odd Fermat Pseudoprime to some base
  then it must be so to an even number of bases.
  \begin{proof}
	Let $n$ be an odd Fermat Pseudoprime, then let $b$ be a base of $n$ where $b^{n-1}\equiv 1\mbox{ mod }n$.
	Because $n$ is odd $n-1$ is even, this means that $b^{n-1} = (-b)^{n-1}$, it follows that any base $b$
	has a pair $-b$ that is also a base as long as $-b\not\equiv b\mbox{ mod }n$.\\\\
	If $-b\equiv b\mbox{ mod }n$ then $n$ would divide $2b$ but since $n$ is odd this implies $n\mid b$.
	Then, $b^{n-1} \equiv 0\mbox{ mod }n$ which contradicts $b^{n-1}\equiv 1\mbox{ mod }n$.\\\\
	So it is not possible for $-b\equiv b\mbox{ mod }n$, therefore any base $b$ of $n$ has a respective pair
	$-b$ such that $n$ has an even number of bases.
  \end{proof}

\item
  Prove that 1105 is a Carmichael number.
  \begin{proof}
	  Note that $1105=5\cdot 13\cdot 17$. Suppose $b$ satisfies $\gcd(b,1105)=1$. Then,
	\begin{itemize}
	  \item $\gcd(b,5)=1$ so by FLiT $b^4 \equiv 1\mbox{ mod }5$. So
	  $b^{1104}=(b^4)^{276}\equiv 1 \mbox{ mod }5$ so $5\mid b^{1104}-1$.

	  \item $\gcd(b,13)=1$ so by FLiT $b^{12}\equiv 1\mbox{ mod }13$. So
	  $b^{1104}=(b^{12})^{92}\equiv 1\mbox{ mod }13$ so $13\mid b^{1104}-1$.

	  \item $\gcd(b,17)=1$ so by FLiT $b^{16}\equiv 1\mbox{ mod }17$. So
	  $b^{1104}=(b^{16})^{69}\equiv 1 \mbox{ mod }17$ so $17\mid b^{1104}-1$. 
	\end{itemize}
	So $5\cdot 13\cdot 17 \mid b^{1104}-1 \implies 1105\mid b^{1104}-1$.
	Therefore $b^{1104} \equiv 1\mbox{ mod }1105$.
  \end{proof}

\item
  Use Euler's Theorem to find the units digit of $7^{999999}$.\\\\
  The units digit is the least non-negative residue mod 10.
  Since $\phi(10)=4$ we have $7^{4} \equiv 1 \mbox{ mod }10$ and so:
  $$7^{999999}=(7^4)^{249999} 7^3\equiv 7^3 \equiv 3\mbox{ mod }10$$
  The units digit of $7{999999}$ is $3$.

\item
  Solve each of the following using Euler's Theorem.
  Solutions should be least nonnegative residues.
\begin{enumerate}
\item
  $5x\equiv 3\mod 14$\\\\
  Since $\gcd(5,14)=1$ then $5^{\phi(14)}\equiv 1\mbox{ mod }14$.
  Then $5^6 = 5^5 \cdot 5\equiv 1\mbox{ mod }14$ where $5^5\equiv 3$ is the inverse.
  Then $5x\equiv 3\mbox{ mod }14$ can be reduced to $x\equiv 3\cdot 3\mbox{ mod }14$.
  So $9$ is the least non-negative residue.
\item
  $4x\equiv 7\mod 15$\\\\
  Since $\gcd(4,15)=1$ then $4^{\phi(15)}\equiv 1\mbox{ mod }15$.
  Then $4^{8} = 4^7\cdot 4 \equiv 1\mbox{ mod }15$ where $4^7\equiv 4$ is the inverse.
  Then $4x\equiv 7\mbox{ mod }15$ can be reduced to $x\equiv 7\cdot 4\mbox{ mod }15$.
  So $13$ is the least non-negative residue.
\item
  $3x\equiv 5\mod 16$\\\\
  Since $\gcd(3,16)=1$ then $3^{\phi(16)}\equiv 1 \mbox{ mod }16$.
  Then $3^{8} = 3^7 \cdot 3 \equiv 1\mbox{ mod }16$ where $3^7 \equiv 11$ is the inverse.
  Then $3x\equiv 5\mbox{ mod }16$ can be reduced to $x\equiv 5\cdot 11\mbox{ mod }16$.
  So $7$ is the least non-negative residue.
\end{enumerate}

\item
  Prove that if $\text{gcd}(a,30)=1$ then $60\mid a^4+59$.
  \begin{proof}
	Note that $60 = 5\cdot 12$, because $\gcd(a,30)=1$ we know $\gcd(a,5)=1$ and $\gcd(a,12)=1$.
	\begin{itemize}
	  \item $a^{\phi(5)} = a^4 \equiv 1\mbox{ mod }5$, so $5\mid a^4-1$.

	  \item $a^{\phi(12)} = a^4 \equiv 1\mbox{ mod }12$, so $12\mid a^4-1$.
	\end{itemize}
	Since $\gcd(5,12)=1$ it follows that $60\mid a^4 -1 \implies 60\mid (a^4 -1)+60 \implies 60\mid a^4 +59$.
  \end{proof}

\end{enumerate}

%%%%%%%%%%%%%%%
%%% Section %%%
%%%%%%%%%%%%%%%
\setcounter{subsection}{6}
\subsection{Chapter 7}
\rule{\textwidth}{1pt}\\
\begin{enumerate}
\item
  Find all $n$ satisfying $\phi(n)=18$. \\\\
  Suppose $p\mid n$, then $p-1\mid \phi(n)$ so $p-1 = 1,2,3,6,9,18 \implies p = 2,3,4,7,10, 19$.
  But $4$ and $10$ are not prime so $p = 2,3,7,19$. Therefore $n = 2^{\alpha} 3^{\beta} 7^{\gamma} 19^{\delta}$
  for some $\alpha$, $\beta$, $\gamma$, and $\delta$. Then, 
  \begin{itemize}
	\item If $2^{\alpha}\mid n$ with $\alpha >0$ then $2^{\alpha -1}\mid\phi(n)$ so $\alpha = 0, 1, 2$.
	\item If $3^{\beta}\mid n$ with $\beta >0$ then $3^{\beta -1}\mid\phi(n)$ so $\beta = 0, 1, 2, 3$.
	\item If $7^{\gamma}\mid n$ with $\gamma >0$ then $7^{\gamma -1}\mid\phi(n)$ so $\gamma = 0, 1$.
	\item If $19^{\delta}\mid n$ with $\delta >0$ then $19^{\delta -1}\mid\phi(n)$ so $\delta = 0, 1$.
  \end{itemize}
  Then these are the cases,
  \begin{center}
	\begin{tabular}{ l l l }
	  $\bullet$ $n=2^0 3^0 7^0 19^0 = 1$ &    $\bullet$ $n=2^1 3^0 7^0 19^0 = 2$ & $\bullet$ $n=2^2 3^0 7^0 19^1 = 4$\\
	  $\bullet$\fbox{$n=2^0 3^0 7^0 19^1 = 19$} &   $\bullet$\fbox{$n=2^1 3^0 7^0 19^1 = 38$} & $\bullet$ $n=2^2 3^0 7^0 19^1 = 76$\\
	  $\bullet$ $n=2^0 3^0 7^1 19^0 = 7$ &    $\bullet$ $n=2^1 3^0 7^1 19^0 = 14$ & $\bullet$ $n=2^2 3^0 7^1 19^1 = 28$\\
	  $\bullet$ $n=2^0 3^0 7^1 19^1 = 133$ &   $\bullet$ $n=2^1 3^0 7^1 19^1 = 266$ & $\bullet$ $n=2^2 3^0 7^1 19^1 = 532$\\
	  $\bullet$ $n=2^0 3^1 7^0 19^0 = 3$ &    $\bullet$ $n=2^1 3^1 7^0 19^0 = 6$ & $\bullet$ $n=2^2 3^1 7^0 19^1 = 12$\\
	  $\bullet$ $n=2^0 3^1 7^0 19^1 = 57$ &   $\bullet$ $n=2^1 3^1 7^0 19^1 = 114$ & $\bullet$ $n=2^2 3^1 7^0 19^1 = 228$\\
	  $\bullet$ $n=2^0 3^1 7^1 19^0 = 21$ &    $\bullet$ $n=2^1 3^1 7^1 19^0 = 42$ & $\bullet$ $n=2^2 3^1 7^1 19^1 = 84$\\
	  $\bullet$ $n=2^0 3^1 7^1 19^1 = 399$ &   $\bullet$ $n=2^1 3^1 7^1 19^1 = 798$ & $\bullet$ $n=2^2 3^1 7^1 19^1 = 1596$\\
	  $\bullet$ $n=2^0 3^2 7^0 19^0 = 9$ &    $\bullet$ $n=2^1 3^2 7^0 19^0 = 18$ & $\bullet$ $n=2^2 3^2 7^0 19^1 = 36$\\
	  $\bullet$ $n=2^0 3^2 7^0 19^1 = 171$ &   $\bullet$ $n=2^1 3^2 7^0 19^1 = 342$ & $\bullet$ $n=2^2 3^2 7^0 19^1 = 684$\\
	  $\bullet$ $n=2^0 3^2 7^1 19^0 = 63$ &    $\bullet$ $n=2^1 3^2 7^1 19^0 = 126$ & $\bullet$ $n=2^2 3^2 7^1 19^1 = 252$\\
	  $\bullet$ $n=2^0 3^2 7^1 19^1 = 1197$ &   $\bullet$ $n=2^1 3^2 7^1 19^1 = 2394$ & $\bullet$ $n=2^2 3^2 7^1 19^1 = 4788$\\
	  $\bullet$\fbox{$n=2^0 3^3 7^0 19^0 = 27$} &    $\bullet$\fbox{$n=2^1 3^3 7^0 19^0 = 54$} & $\bullet$ $n=2^2 3^3 7^0 19^1 = 108$\\
	  $\bullet$ $n=2^0 3^3 7^0 19^1 = 513$ &   $\bullet$ $n=2^1 3^3 7^0 19^1 = 1026$ & $\bullet$ $n=2^2 3^3 7^0 19^1 = 2052$\\
	  $\bullet$ $n=2^0 3^3 7^1 19^0 = 189$ &    $\bullet$ $n=2^1 3^3 7^1 19^0 = 378$ & $\bullet$ $n=2^2 3^3 7^1 19^1 = 756$\\
	  $\bullet$ $n=2^0 3^3 7^1 19^1 = 3591$ &   $\bullet$ $n=2^1 3^3 7^1 19^1 = 7182$ & $\bullet$ $n=2^2 3^3 7^1 19^1 = 14364$
	\end{tabular}
  \end{center}
  Then evaluating all $n$ as $\phi(n)$ we get that for $n=19,27,38,54$, $\phi(n)=18$.

\item
  Show there are no $n$ with $\phi(n)=14$. \\\\
  Suppose $\phi(n)=14$ for some $n$, then $7\mid p^{\alpha} - p^{\alpha -1}$
  for some odd prime $p$. Since the factors of $14$ are $2$ and $7$ we have two cases.
  If $p=7$ and $\alpha >1$ which implies $6\mid 14$ which is not true.
  Or if $7\mid p-1$ but $p-1$ is even, so $p=15$ which is not prime. Therefore
  there are no $n$ with $\phi(n)=14$.

\item
  For what values of $n$ is $\phi(n)$ odd?
  Justify. \\\\
  Since $\phi(p^{\alpha})=p^{\alpha - 1} (p-1)$ we know that it will be even for all $p>2$,
  therefore $n$ cannot have any prime factors greater than 2. It follows that for $n=1,2$
  $\phi(n)$ is odd.

\item
  Prove that $f(n)=\gcd(n,3)$ is multiplicative.
  (This is actually true if 3 is replaced by any positive integer.)
  \begin{proof}
	We wish to show $f(mn)=f(m)\cdot f(n)$ when $\gcd(m,n)=1$.
	Suppose that $\gcd(m,n)=1$, then
	$\gcd(\gcd(m,3), \gcd(n,3)) = p$ for $p\in\integers^+$.
	This implies that $p\mid \gcd(m,3)$ and $p\mid\gcd(n,3)$ which implies
	$p\mid m$ and $p\mid n$, but since $\gcd(m,n)=1$ we have $p=1$. \\
	Let $\gcd(m,3)=x$, this implies $x\mid 3$ and $x\mid m$. Then,
	$$x\mid m \implies x\mid mn \implies x\mid \gcd(mn, 3)$$
	Likewise, let $\gcd(n,3)=y$, this implies $y\mid 3$ and $y\mid n$. Then,
	$$y\mid n \implies y\mid mn \implies y\mid \gcd(mn, 3)$$
	Then because $x\mid \gcd(mn,3)$ and $y\mid\gcd(mn,3)$ and $\gcd(x,y)=1$
	we have, $xy\mid \gcd(mn, 3)$.
	Thus, $\gcd(mn,3) = \gcd(m,3)\cdot \gcd(n,3)$ and $f(n)$ is multiplicative.
  \end{proof}

\item
  Find $\tau(2\cdot 3^2\cdot 5^3\cdot 11^5\cdot 13^4\cdot 17^5\cdot 19^5)$
  $$=(1+1)(2+1)(3+1)(5+1)(4+1)(5+1)(5+1) = 2\cdot 3\cdot 4\cdot 6\cdot 5\cdot 6\cdot 6 = 25920$$

\item
  Find $\sigma(2\cdot 3^2\cdot 5^3\cdot 11^5\cdot 13^4\cdot 17^5\cdot 19^5)$
  $$=\left(\frac{2^2 - 1}{2-1}\right)\left(\frac{3^3-1}{3-1}\right)\left(\frac{5^4-1}{5-1}\right)
  \left(\frac{11^6-1}{11-1}\right)\left(\frac{13^5-1}{15-1}\right)\left(\frac{17^6-1}{17-1}\right)
  \left(\frac{19^6-1}{19-1}\right)$$

\item
  Find $\tau(20!)$. \\\\
  First we need the prime factorization of $20!$,
  \begin{align*}
	20! &= \left(2\right) \left(3\right) \left(2^2\right) \left(5\right) \left(2\cdot 3\right)
	\left(7\right) \left(2^3\right) \left(3^2\right) \left(2\cdot 5\right) \left(11\right)
	\left(2^2\cdot 3\right)\\
	& \left(13\right) \left(2\cdot7\right) \left(3\cdot 5\right)
	\left(2^4\right) \left(17\right) \left(2\cdot 3^2\right) \left(19\right) \left(2^2 \cdot 5\right)
  \end{align*}
  Thus, $20! = 2^{18} \cdot 3^8 \cdot 5^4 \cdot 7^2 \cdot 11 \cdot 13 \cdot 17 \cdot 19$.
  Therefore, $\tau(20!)=(18+1)(8+1)(4+1)(2+1)(1+1)(1+1)(1+1)(1+1) = 41040$.

\item
  Classify all $n$ with $\tau(n)=30$. Explain! \\\\
  Suppose $n = p_1^{\alpha_1} \cdots p_k^{\alpha_k}$ with $a_i > 0$, it follows that
  $\tau(n) = (1+\alpha_1) \cdots (1+\alpha_k)=30$. Since $(1+\alpha_i) \geq 2$ we get
  the following cases:
  \begin{itemize}
	\item $k=15$ and $a_1 = \cdots = a_{15} = 1$.
	\item $k=10$ and $a_1 = \cdots = a_{10} = 2$.
	\item $k=6$ and $a_1 = \cdots = a_{6} = 4$.
	\item $k=5$ and $a_1 = \cdots = a_{5} = 5$.
	\item $k=3$ and $a_1 = \cdots = a_{3} = 9$.
	\item $k=1$ and $a_1 = 29$.
  \end{itemize}
  Then we have,
  \begin{itemize}
	\item $n=p_1\cdots p_{15}$
	\item $n=p_1^2\cdots p_{10}^2$
	\item $n=p_1^4\cdots p_{6}^4$
	\item $n=p_1^5\cdots p_{5}^5$
	\item $n=p_1^9 p_2^9 p_3^9$
	\item $n=p_1^{29}$
  \end{itemize}

\item
  Prove that $\sigma(n)=k$
  has at most a finite number of solutions when $k$ is a positive integer.
  \begin{proof}
	Since $\sigma(n)$ is multiplicative we know that $1\mid n$ and $n\mid n$ implies that
	$\sigma(n) \geq 1+n$. It follows then that $\sigma(n)=k \implies 1+n\leq k \implies n\leq k-1$.
	Thus, $\sigma(n)=k$ has a finite number of solutions.
  \end{proof}

\item
  Show that if $a$ and $b$ are positive integers
  and $p$ and $q$ are distinct odd primes
  then $n=p^aq^b$ is deficient. \\\\
  We wish to show $\sigma(n) < 2n$,
  $$\sigma(p^a q^b) = \left(\frac{p^{a+1}-1}{p-1}\right) \left(\frac{q^{b+1}-1}{q-1}\right)
  < \left(\frac{p^{a+1}}{p-1}\right) \left(\frac{q^{b+1}}{q-1}\right)
  = \left(\frac{p}{p-1}\right)\left(\frac{q}{q-1}\right) p^a q^b$$
  Since we wish to show $\sigma(n) < 2n$ we then need 
  $\left(\frac{p}{p-1}\right)\left(\frac{q}{q-1}\right) < 2$.
  \begin{align*}
	\left(\frac{p}{p-1}\right)\left(\frac{q}{q-1}\right) &< 2 \\
	pq &< 2(p-1)(q-1) \\
	pq &< 2\left(pq -p -q + 1\right) \\
	pq &< 2pq -2p -2q +2 \\
	0 &< pq -2p -2q +2 \\
	0 &< (p-2)(q-2) -2 \\
	-2 &< (p-2)(q-1)
  \end{align*}
  Recall that $p$ and $q$ are \textit{distinct} odd primes, so let's assume that
  $p>q>2$. Since let $p=5$ and $q=3$, we then see $2 < (5-2)(3-2)$, thus the
  inequality holds. Therefore $p^a q^b$ is deficient.

\item
  Prove that a perfect square cannot be a perfect number.
  \begin{proof}
	Let $n$ be a perfect square, first observe that when $n$ is even by Euler's Theorem 
	$n= 2^{p-1} \left(2^p -1\right) \implies \sqrt{n} = \sqrt{2^{p-1} \left(2^p -1\right)}$
	but then this implies that $\sqrt{n} \not\in \integers$ which contradicts $n$ being a
	perfect square. Therefore $n$ must be odd. So 
	$n= 2^{k} \left(p_1^{\alpha_1}\cdots p_i^{\alpha_i}\right)$, note that for 
	$p_1^{\alpha_1}\cdots p_i^{\alpha_i}$ each $p$ and $\alpha$ is even.
	Then it follows that $n$ has an odd number of odd divisors. Thus $\sigma(n)$ is an odd number,
	but a perfect number is when $\sigma(n)=2n$ which implies $\sigma(n)$ must be an even number.
	Therefore a perfect square cannot be a perfect number.
  \end{proof}

\item
  Use Theorem 7.12 to determine whether each of the following Mersenne numbers
  is a Mersenne prime:
  \begin{enumerate}
  \item[] Personally, I found it easiest to create a simple python script to test 
  for primality of Mersenne numbers.
\begin{center}
\begin{lstlisting}[language=Python]
import math

def Mersenne_Prime (n):
  M = 2**n -1

  for k in range (1, int(math.sqrt(M))):
	factor = (2*n)*k +1
	if M % factor == 0:
	  print(M/factor)
	  return false
  
  return True
\end{lstlisting}
\end{center}


  \item
	$M_{11}$ \\\\
	First we see $M_{11} = 2^{11} - 1= 2047$, then the factors of 2047 are of the form
	$2(11)k+1 = 22k + 1$. Look at $k$ up to $\sqrt{2047}\approx 45.24$. We find that
	$2047 = 23\cdot 89$, therefore $M_{11}$ is not prime.
  \item
	$M_{21}$ \\\\
	First we observe that $21=3\cdot 7$, then by the definition of a Mersenne prime, since $21$ is
	not prime then $M_{21}$ is not prime.
  \item
	$M_{31}$ \\\\
	First we see $M_{31} = 2^{31} -1=2147483647$, then the factors of 2147483647 are of the form
	$2(31)k+1 = 62k+1$. Look at $k$ up to $\sqrt{2147483647} \approx 46340.95$. We find that
	there is no factor of $2147483647$ of the form $62k+1$ for $k\leq 46341$, therefore $M_{31}$ is prime.
  \end{enumerate}

\end{enumerate}

%%%%%%%%%%%%%%%
%%% Section %%%
%%%%%%%%%%%%%%%
\setcounter{subsection}{8}
\subsection{Chapter 9}
\rule{\textwidth}{1pt}\\
\subsubsection{}
\begin{enumerate}
  \item
    $\ord_{21}8$ \\\\
    Since $\phi(21)= 12$ we have $\ord_{21} = 1,2,3,4,6,12$. Then we see,
    \begin{align*}
      8^{1} &\equiv 8 \mbox{ mod } 21 \\
      8^{2} &\equiv 1 \mbox{ mod } 21
    \end{align*}
    Therefore, $\ord_{21} 8 = 2$.

  \item
    $\ord_{25}8$ \\\\
    Since $\phi(25)= 20$ we have $\ord_{25} = 1,2,4,5,10,20$. Then we see,
    \begin{align*}
      8^{1} &\equiv 8 \mbox{ mod } 25 \\
      8^{2} &\equiv 14 \mbox{ mod } 25 \\
      8^{4} &\equiv 21 \mbox{ mod } 25 \\
      8^{5} &\equiv 18 \mbox{ mod } 25 \\
      8^{10} &\equiv 24 \mbox{ mod } 25 \\
      8^{20} &\equiv 1 \mbox{ mod } 25
    \end{align*}
    Therefore, $\ord_{25} 8 = 20$.

  \end{enumerate}

\subsubsection{}
We wish to find a primtive root $r$ such that $r^{\phi(n)} \equiv 1\mbox{ mod }n$.
  First we see that $\phi(50) = 20$ so we want to find an $r$ for $r^{20} \equiv 1\mbox{ mod } 50$.
  Let $r=3$, we can then observe:
  \begin{align*}
    3^{1} &\equiv 3 \mbox{ mod } 50 \\
    3^{2} &\equiv 9 \mbox{ mod } 50 \\
    3^{4} &\equiv 31 \mbox{ mod } 50 \\
    3^{5} &\equiv 43 \mbox{ mod } 50 \\
    3^{10} &\equiv 49 \mbox{ mod } 50 \\
    3^{20} &\equiv 1 \mbox{ mod } 50 
  \end{align*}
  So we have $r=3$ as a primitive root for $n=50$. Then we see that there are a total of
  $\phi(\phi(50)) = \phi(20) = 8$ primitive roots for $n=50$. Take $k$ with 
  $\gcd(k,\phi(50))=1 \implies \gcd(k, 20)=1$. So $k=1,3,7,9,11,13,17,19$. Then,
  \begin{align*}
    3^{1} &\equiv 3 \mbox{ mod } 50 \\
    3^{3} &\equiv 27 \mbox{ mod } 50 \\
    3^{7} &\equiv 37 \mbox{ mod } 50 \\
    3^{9} &\equiv 33 \mbox{ mod } 50 \\
    3^{11} &\equiv 47 \mbox{ mod } 50 \\
    3^{13} &\equiv 23 \mbox{ mod } 50 \\
    3^{17} &\equiv 13 \mbox{ mod } 50 \\
    3^{19} &\equiv 17 \mbox{ mod } 50
  \end{align*}
  So we get $3,13,17,23,27,33,37,47$ as the primitive roots of $n=50$.

\subsubsection{}
\begin{proof}
	Since $\ord_p a = 2k$ we have that $a^{2k}\equiv 1\mbox{ mod }p$ which implies that
	$p\mid\left(a^{2k}-1\right)$ where $a^{2k} -1 = (a^k + 1)(a^k -1)$. So we have two cases,
	\begin{itemize}
		\item If $p\mid (a^k + 1)$ then we get $a^k \equiv -1 \mbox{ mod }p$.
		\item If $p\mid (a^k - 1)$ then we get $a^k \equiv 1 \mbox{ mod }p$, but this
		contradicts the fact that $\ord_p a = 2k$.
	\end{itemize}
	Thus, $p$ can only divide $a^k + 1$ and therefore $a^k \equiv -1 \mbox{ mod }p$.
\end{proof}

\subsubsection{}
Since $\ord_m a = m-1$ we know that because $\ord_m a\mid \phi(m)$ we have $(m-1)\mid\phi(m)$.
  But from the definition of the Euler Phi function we know that $\phi(m) \leq m-1$ so therefore
  $\phi(m) = m-1$. Then it follows that $m$ must be prime.

\subsubsection{}
\begin{proof}
	Using the proof in problem three we can see:
	$$
		\ind_r a + \left(\frac{p-1}{2}\right) \implies
		\ind_r a + \ind_r(p-1) \implies
		\ind_r (ap - a)
	$$
	It then follows that
	$\ind_r(p-a) \equiv \ind_r(ap - a)\mbox{ mod } p-1$.
	From here we can "un-index" to then get
	$(p-a) \equiv (ap-a) \mbox{ mod }p$
	which we know to be true from the definition of congruence.
	Thus, $\ind_r(p-a)\equiv \ind_ra+\left(\frac{p-1}{2}\right)\mod p-1$.
\end{proof}

\subsubsection{}
We will show $\ord_n(ab)=(\ord_n a)(\ord_n b)$ by two directions, 
  first the left side divides the right and then the right side divides the left.
  \begin{itemize}
    \item Observe that 
    $(ab)^{\ord_n a \cdot \ord_n b} = a^{\ord_n a \cdot \ord_n b} \cdot b^{\ord_n a \cdot \ord_n b}
    = 1^{\ord_n b} \cdot 1^{\ord_n a} \equiv 1 \mbox{ mod } n$. This implies that $\ord_n (ab) \mid (\ord_n a)(\ord_n b)$.

    \item We know that $(ab)^{\ord_n (ab)}\equiv 1$, so we can see the following.
    \begin{align*}
      \begin{split}
        (ab)^{\ord_n (ab)} &\equiv 1 \\
        \left((ab)^{\ord_n (ab)}\right)^{\ord_n a} &\equiv 1^{\ord_n a} \\
        b^{\ord_n (ab) \cdot \ord_n a} &\equiv 1
      \end{split}
      \begin{split}
        (ab)^{\ord_n (ab)} &\equiv 1 \\
        \left((ab)^{\ord_n (ab)}\right)^{\ord_n b} &\equiv 1^{\ord_n b} \\
        a^{\ord_n (ab) \cdot \ord_n b} &\equiv 1
      \end{split}
    \end{align*}
    Looking at $b^{\ord_n (ab) \cdot \ord_n a} \equiv 1$ we see that 
    $\ord_n b \mid (\ord_n (ab))(\ord_n a)$. But, since $\ord_n a$ and $\ord_n b$ are coprime to
    one another we get,
    $\ord_n b \mid \ord_n (ab)$. Likewise, the same can be said about $\ord_n a$, therefore
    $\ord_n a \mid \ord_n (ab)$. Then we have $(\ord_n a)(\ord_n b)\mid \ord_n (ab)$.
  \end{itemize}
	Then we see that since $\ord_n (ab)\mid (\ord_n a)(\ord_n b)$ and $(\ord_n a)(\ord_n b)\mid \ord_n (ab)$
  we get $\ord_n(ab)=(\ord_n a)(\ord_n b)$.

\subsubsection{}
\begin{proof}
	Since $p\equiv 1\mbox{ mod } 4$ let $p=4k+1$ for some $k\in\integers$.
	We know that for a primtive root $r$, $r^{\phi(p)/2} \equiv -1\mbox{ mod }p$.
	It then follows that $r^{((4k+1)-1)/2} = r^{4k/2} = r^{2k} \equiv -1\mbox{ mod }p$.
	Thus, $(-r)^{2k} \equiv -1\mbox{ mod }p$ and then taking $-r$ to some power gives us
	congruence to $r$. Therefore $-r$ is a primtive root of $p$.
\end{proof}

\subsubsection{}
\begin{enumerate}
  \item $ $
    \begin{figure}[h]
    \centering
    \begin{tabular}{|c|c|c|c|c|c|c|c|c|c|c|c|c|}
      \hline
      $a$ & 1 & 2 & 3 & 4 & 5 & 6 & 7 & 8 & 9 & 10 & 11 & 12 \\
      \hline
      $\ind_7 a$ & 12 & 11 & 8 & 10 & 3 & 7 & 1 & 9 & 4 & 2 & 5 & 6 \\
      \hline
    \end{tabular}
    \end{figure}

  \item
    \begin{align*}
      x^{2} &\equiv 12\mbox{ mod } 13 \\
      \ind_7 (x^2) &\equiv \ind_7 12 \mbox{ mod } \phi(13) \\
      2\ind_7 x &\equiv 6 \mbox{ mod } 12 \\
      \ind_7 x &\equiv 3 \mbox{ mod } 6 \\
      \ind_7 x &\equiv 3,9 \mbox{ mod } 12 \\
      x &\equiv 5,8 \mbox{ mod } 13
    \end{align*}

  \item
    \begin{align*}
      4^x &\equiv 12\mbox{ mod } 13 \\
      \ind_7 4^x &\equiv \ind_7 12 \mbox{ mod } \phi(13) \\
      x\ind_7 4 &\equiv 6 \mbox{ mod } 12 \\
      10x &\equiv 6 \mbox{ mod } 12 \\
      5x &\equiv 3 \mbox{ mod } 6 \\
      (5)(5x) &\equiv (5)(3) \mbox{ mod } 6\\
      x &\equiv 3 \mbox{ mod } 6 \\
      x &\equiv 3, 9 \mbox{ mod } 12
    \end{align*}

  \end{enumerate}
\subsubsection{}
\begin{enumerate}
  \item
    The problem with this is the use of the fraction $\frac{a}{b}$, we can not
    always guarantee $\frac{a}{b}$ to be an integer, furthermore there is no such
    thing as "divison" in this context.

  \item
    We know that if $\gcd(b,\phi(n))=1$ then $\exists b'$ such that $b\cdot b' \equiv 1\mbox{ mod } \phi(n)$.
    Then we can substitute $\ind_r (\frac{a}{b})$ with $\ind_r (a\cdot b')$.
    Formally, if $\gcd(a,n)=\gcd(b,n)=1$ and $r$ is a primitive root then,
    $$\ind_r a - \ind_r b \equiv \ind_r (a \cdot b') \mbox{ mod } \phi(n)$$

  \item
    \begin{proof}
      Suppose $\gcd(a,n) = \gcd(b,n) = 1$ and $r$ is a primitive root.
      Observe then,
      \begin{align*}
        \ind_r a - \ind_r b \mbox{ mod } \phi(n)
        &\equiv r^{\ind_r a} \cdot r^{-\ind_r b} \mbox{ mod } n \\
        &\equiv r^{\ind_r a} \cdot r^{\ind_r b'} \mbox{ mod } n \\
        &\equiv r^{\ind_r (a\cdot b')} \mbox{ mod } n \\
        &\equiv \ind_r(a\cdot b') \mbox{ mod } \phi(n)
      \end{align*}
      Thus, we see then that $\ind_r a - \ind_r b \equiv \ind_r (a \cdot b') \mbox{ mod } \phi(n)$.
    \end{proof}

  \end{enumerate}

\subsubsection{}
\begin{proof}
	If $r_1$ and $r_2$ are primitive roots for some odd prime $p$ we know that,
	$r_1^{(p-1)} \equiv r_2^{(p-1)} \equiv 1$ but
	$r_1^{(p-1)/2} \equiv r_2^{(p-1)/2} \equiv -1$. In the first case we see that
	$(r_1\cdot r_2)^{(p-1)} \equiv r_1^{(p-1)} \cdot r^{(p-1)} \equiv 1$ so that works.
	In the second case we see that
	$(r_1\cdot r_2)^{(p-1)/2} \equiv r_1^{(p-1)/2} \cdot r_2^{(p-1)/2} \equiv -1\cdot -1 \equiv 1$.
	Therefore $r_1 r_2$ is not a primitive root of $p$.
\end{proof}

%%%%%%%%%%%%%%%
%%% Section %%%
%%%%%%%%%%%%%%%
\setcounter{subsection}{10}
\subsection{Chapter 11}
\rule{\textwidth}{1pt}\\
\subsubsection{}
Observe,
  \begin{table}[h!]
  \begin{center}
  \begin{tabular}{cccc}
    $1^2\equiv1\mbox{ mod }17$ & $5^2\equiv8\mbox{ mod }17$ & $9^2\equiv13\mbox{ mod }17$ & $13^2\equiv16\mbox{ mod }17$ \\
    $2^2\equiv4\mbox{ mod }17$ & $6^2\equiv2\mbox{ mod }17$ & $10^2\equiv15\mbox{ mod }17$ & $14^2\equiv9\mbox{ mod }17$ \\
    $3^2\equiv9\mbox{ mod }17$ & $7^2\equiv15\mbox{ mod }17$ & $11^2\equiv2\mbox{ mod }17$ & $15^2\equiv4\mbox{ mod }17$\\
    $4^2\equiv16\mbox{ mod }17$ & $8^2\equiv13\mbox{ mod }17$ & $12^2\equiv8\mbox{ mod }17$ & $16^2\equiv1\mbox{ mod }17$
  \end{tabular}
  \end{center}
  \end{table}

  From here we can see that $1,2,4,8,9,13,15,$ and $16$ are Quadratic Residues mod 17.
\subsubsection{}
\begin{enumerate}[(a)]
	\item
		\begin{align*}
			\leg{3}{17} \equiv 3^{(17-1)/2} = 3^{8} \equiv 16 \equiv -1\mbox{ mod }17
		\end{align*}
		Thus, 3 is a QNR mod 17.
	\item
		By Gauss's Lemma we have the set $\{3,2\cdot3, \cdots, ((17-1)/2)\cdot 3\}$ which is $\{3,6,9,12,15,18,21,24\}$.
		Then we take them mod 17, to get $\{3,6,9,12,15,1,4,7\}$.
		We want to see how many are greater than $17/2=8.5$, we then see that 3 are greater than 8.5.
		Then $(-1)^3=-1$, so we have $\leg{3}{17}=-1$ and therefore 3 is a QNR mod 17.
	\end{enumerate}

\subsubsection{}
\begin{proof}
	We wish to show that $\ord_q(-4)=\phi(q)$. Since we know $q$ to be an odd prime, we have that
	$\phi(q)=q-1\implies (2p+1)-1=2p$. So we then get $\ord_q(-4)\mid 2p$, of which $\ord_q(-4)=1,2,p,$ or $2p$.
	\begin{itemize}
		\item If $\ord_q(-4)=1$ then $(-4)^1\equiv 1\mbox{ mod }q \implies q\mid -5 \implies q=5$, but this implies that
		$p=2$ since $q=2p+1$, and we know both $p$ and $q$ to be odd primes, so $\ord_q(-4)\neq 1$.

		\item If $\ord_q(-4)=2$ then $(-4)^2\equiv 1\mbox{ mod }q \implies s\mid 15 \implies q=5,3$, but this implies
		$p$ to be either be $1,2$, and we know both $p$ and $q$ to be odd primes, so $\ord_q(-4)\neq 2$.

		\item If $\ord_q(-4)=p$ then $(-4)^p\equiv 1\mbox{ mod }q \implies \frac{-4}{q}\equiv 1\mbox{ mod }q \implies
		\frac{-1}{2p+1} (1) \equiv 1\mbox{ mod }q \implies -1\equiv 1\mbox{ mod }q$, but this implies $q\mid 2$ which
		is false.
	\end{itemize}
	Thus, by process of elimination the only value that $\ord_q(-4)$ is equivalent to is $2p$.
\end{proof}

\subsubsection{}
\begin{proof}
	We will first prove $4$ to be a Quadratic Residue,
	$$\leg{-4}{p} = \leg{-1}{p}\leg{2}{p}^2 = 1$$
	Then, to prove $\frac{p-1}{4}$,
	$$\leg{\frac{p-1}{4}}{p} = \leg{\frac{p-1}{4}}{p}\leg{4}{p} = \leg{p-1}{p} = \leg{-1}{p} = 1$$
	We can substitute in $\leg{4}{p}$ during the second step because in the first part of the proof
	we showed $\leg{4}{p}$ to be a Quadratic Residue. Thus, both -4 and $(p-1)/4$ are Quadratic Residues of $p$
	when $p\equiv 1\mbox{ mod }4$.
\end{proof}

\subsubsection{}
\begin{enumerate}[(a)]
  \item
    \begin{align*}
      \leg{21}{59} &= \leg{3}{59}\leg{7}{59} \text{by splitting.} \\
      &= \Big[-\leg{59}{3}\Big] \Big[-\leg{59}{7}\Big] \text{by LoQR since }3,7\equiv 3\mbox{ mod }4. \\
      &= \leg{2}{3}\leg{3}{7} \text{by reducing.} \\
      &= (-1)^{1} \Big[-\leg{7}{3}\Big] \text{by 2 rule and LoQR since }3\equiv 3\mbox{ mod }4. \\
      &= \leg{1}{3} \text{by reducing.} \\
      &= 1
    \end{align*}
  \item
    \begin{align*}
      \leg{1463}{89} &= \leg{7}{89}\leg{11}{89}\leg{19}{89} \text{by splitting.}\\
      &= \leg{89}{7}\leg{89}{11}\leg{89}{19} \text{by LoQR.} \\
      &= \leg{5}{7}\leg{1}{11}\leg{13}{19} \text{by reducing.} \\
      &= \leg{7}{5}\leg{19}{13} \text{by LoQR.} \\
      &= \leg{2}{5}\leg{6}{13} \text{by reducing.} \\
      &= (-1)^{(25-1)/8}\leg{2}{13}\leg{3}{13} \text{by 2 rule and splitting.} \\
      &= (-1)(-1)^{(13^2-1)/8} \leg{13}{3} \text{by 2 rule and LoQR.} \\
      &= \leg{1}{3} \text{by reducing.} \\
      &= 1
    \end{align*}
  \item
    \begin{align*}
      \leg{1547}{1913} &= \leg{7}{1913}\leg{13}{1913}\leg{17}{1913} \text{by splitting.} \\
      &= \leg{1913}{7}\leg{1913}{13}\leg{1913}{17} \text{by LoQR.} \\
      &= \leg{1}{7}\leg{2}{13}\leg{9}{17} \text{by reducing.} \\
      &= (-1)^{(13^2-1)/8} \text{by 2 rule} \\
      &= -1
    \end{align*}
  \end{enumerate}
\subsubsection{}
We have two main cases, $p\equiv 1$ or $p\equiv 3$ mod 4.
  \begin{enumerate}[1.]
    \item If $p\equiv 1\mbox{ mod }4 \implies \leg{3}{p}=\leg{p}{3}$. Then,
    \begin{itemize}
      \item $p\equiv 1\mbox{ mod }3\implies \leg{p}{3}=\leg{1}{3}=1\implies p\equiv 1\mbox{ mod }12$
      \item $p\equiv 2\mbox{ mod }3\implies \leg{p}{3}=\leg{2}{3}=-1\implies p\equiv 5\mbox{ mod }12$
    \end{itemize}
    \item If $p\equiv 3\mbox{ mod }4 \implies \leg{3}{p}=-\leg{p}{3}$. Then,
    \begin{itemize}
      \item $p\equiv 1\mbox{ mod }3\implies -\leg{p}{3}=-\leq{1}{3}=-1\implies p\equiv 7\equiv -5\mbox{ mod }12$
      \item $p\equiv 2\mbox{ mod }3\implies -\leg{p}{3}=-\leg{2}{3}=1\implies p\equiv 11\equiv -1\mbox{ mod }12$
    \end{itemize}
  \end{enumerate}

\subsubsection{}
By the Law of Quadratic Residues since $5\equiv 1\mbox{ mod }4$ we have that $\leg{5}{p}=\leg{p}{5}$.
  Then because $\gcd(5,p)=1$ we know that $p\mbox{ mod }5$ can only be $1,2,3$ or $4$.
  \begin{itemize}
    \item $p\equiv 1\mbox{ mod }5\implies \leg{1}{5} = 1$
    \item $p\equiv 2\mbox{ mod }5\implies \leg{2}{5} = -1$
    \item $p\equiv 3\mbox{ mod }5\implies \leg{3}{5} = -1$
    \item $p\equiv 4\mbox{ mod }5\implies \leg{4}{5} = 1$
  \end{itemize}
  So we see that in order for $\leg{5}{p}=1$, $p$ must be either $1$ or $4$ mod 5.

\subsubsection{}
\begin{enumerate}[(a)]
  \item
    \begin{align*}
      \leg{5}{21} &= \leg{21}{5} \text{by LoQR since } 5\equiv 1\mbox{ mod }4. \\
      &= \leg{1}{5} \text{by reducing.} \\
      &= 1
    \end{align*}

  \item
    \begin{align*}
      \leg{1009}{2307} &= \leg{2307}{1009} \text{by LoQR since } 1009\equiv 1\mbox{ mod }4. \\
      &= \leg{289}{1009} \text{by reducing.} \\
      &= \leg{1009}{289} \text{by LoQR since } 289\equiv 1 \mbox{ mod }4. \\
      &= \leg{142}{289} \text{by reducing.} \\
      &= \leg{2}{289}\leg{71}{289} \text{by splitting.} \\
      &= \leg{289}{71} \text{by 2 rule and LoQR since }289\equiv 1\mbox{ mod }4. \\
      &= \leg{5}{71} \text{by reducing.} \\
      &= \leg{71}{5} \text{by LoQR since }5\equiv 1\mbox{ mod }4. \\
      &= \leg{1}{5} \text{by reducing.} \\
      &= 1
    \end{align*}

  \item
    \begin{align*}
      \leg{27}{101} &= \leg{101}{27} \text{by LoQR since }101\equiv 1\mbox{ mod }4. \\
      &= \leg{20}{27} \text{by reducing.} \\
      &= \leg{4}{27}\leg{5}{27} \text{by splitting.} \\
      &= \leg{27}{5} \text{by LoQR since }5\equiv 1\mbox{ mod }4. \\
      &= \leg{2}{5} \text{by reducing.} \\
      &= -1
    \end{align*}
  \end{enumerate}

\subsubsection{}
Observe that $\leg{n}{3}\leg{n}{5}=\leg{n}{15}=\leg{15}{n}=1$.
  It follows then that for $n\equiv 1\mbox{ mod }4$ we have cases, $\leg{n}{3}=1$ and $\leg{n}{5}=1$,
  and $\leg{n}{3}=-1$ and $\leg{n}{5}=-1$.
  \begin{itemize}
    \item If $\leg{n}{3}=1$ and $\leg{n}{5}=1$ then $n\equiv1\mbox{ mod }3$ and $n\equiv 1,4\mbox{ mod }5$.
    \item If $\leg{n}{3}=-1$ and $\leg{n}{5}=-1$ then $n\equiv2\mbox{ mod }3$ and $n\equiv2,3\mbox{ mod }5$.
  \end{itemize}
  Then, for $n\equiv 3\mbox{ mod }4$ we have cases, $\leg{n}{3}=1$ and $\leg{n}{5}=-1$, and
  $\leg{n}{3}=-1$ and $\leg{n}{5}=1$.
  \begin{itemize}
    \item If $\leg{n}{3}=1$ and $\leg{n}{5}=-1$ then $n\equiv1\mbox{ mod }3$ and $n\equiv 2,3\mbox{ mod }5$.
    \item If $\leg{n}{5}=-1$ and $\leg{n}{5}=1$ then $n\equiv2\mbox{ mod }3$ and $n\equiv 1,4\mbox{ mod }5$.
  \end{itemize}

\subsubsection{}
First observe that the prime factorization of $a$ can include primes to both
  even and odd powers. Let $p$ denote prime factors of $a$ to odd powers,
  similarly let $q$ denote prime factors of $a$ to even powers. We then see,
  $a = p_1\cdots p_i q_1\cdots q_j$.
  Then from rules of the Jacobi Symbol we can split $\leg{a}{n}$,
  it follows then that we do not care about the prime factors of $a$ raised to
  even powers (denoted $q$) since their Jacobi Symbol will always be 1.
  So then we get
  $\leg{a}{n}=\leg{p_1}{n}\cdots\leg{p_i}{n}$
  We then need to handle the specific case of when $p_1$ is 2.
  If $p_1=2$ we have:
  \begin{align*}
    \leg{a}{n} &=\leg{p_1}{n}\cdots\leg{p_i}{n} \\
    &= \leg{2}{n}\cdots\leg{p_i}{n} \\
    &= \leg{2}{n}\cdots\leg{n}{p_i} \\
    &= (-1) \cdots (1) \\
    &= -1
  \end{align*}
  In order for this to work we would need $n$ such that $n\equiv 5\mbox{ mod }8$ (2 rule)
  and $n\equiv 1\mbox{ mod }p_k$ for all $k$ (LoQR used in line 3). Then for the more general
  case of $p_1>2$ we have:
  \begin{align*}
    \leg{a}{n} &=\leg{p_1}{n}\cdots\leg{p_i}{n} \\
    &=\leg{n}{p_1}\cdots\leg{n}{p_1} \\
    &=\leg{x}{p_1}\cdots(1)\\
    &= -1
  \end{align*}
  In order for this to work we would need $n$ such that $n\equiv x\mbox{ mod }p_1$ (where $x$ is a QNR of $p_1$),
  $n\equiv 1\mbox{ mod }4$, and
  $n\equiv 1\mbox{ mod }p_k$ for all $k$.
  Thus, we get
  $$\text{if }p_1=2, n \equiv\begin{cases}5\mbox{ mod }8 &\\1\mbox{ mod }p_k&\forall k\end{cases}
  \hspace{5mm}\text{if }p_1>2, n\equiv\begin{cases}x\mbox{ mod }p_1 &(x\text{ is a QNR of }p_1) \\
    1\mbox{ mod }4 &\\1\mbox{ mod }p_k &\forall k\end{cases}$$

%%%%%%%%%%%%%%%
%%% Section %%%
%%%%%%%%%%%%%%%
\setcounter{subsection}{7}
\subsection{Chapter 8}
\rule{\textwidth}{1pt}\\
\subsubsection{}
For each letter we use $C\equiv 11P + 8\mbox{ mod }26$ and get \verb|ZSYJAVJGSJJQSEA|
  as the ciphertext.
\subsubsection{}
\begin{enumerate}[(a)]
	\item  First observe that the two most frequent letters in the ciphertext are
	\verb|S| and \verb|N|, so lets assume that these correspond to
	\verb|E| and \verb|T| respectively. Then we have
	\begin{align*}
		18 &\equiv 4a+b\mbox{ mod }26 \\
		13 &\equiv 19a+b\mbox{ mod }26
	\end{align*}
	This then gives us
	\begin{align*}
		-5 &\equiv a(19-4)\mbox{ mod }26 \\
		-5 &\equiv a(15)\mbox{ mod }26 \\
		(5)(3a) &\equiv -5\mbox{ mod }26 \\
		-5(5)(3a) &\equiv -5(-5)\mbox{ mod }26 \\
		3a &\equiv 25\mbox{ mod }26 \\
		a &\equiv 17\mbox{ mod }26
	\end{align*}
	Then we find $b\equiv 18-4a=18-4(17)\equiv 2\mbox{ mod }26$. Together,
	we get that $a=17,b=2$.
	\item First observe that the multiplicative inverse of $a=17$ is $a^{-1}=23$.
	Then we get $P=23(C-2)\mbox{ mod }26$, resulting in
	\verb|YESTHEALIENSHAVELOCATEDITWHATNEXT|.
\end{enumerate}
\subsubsection{}
\begin{table}[h]
	\centering
	\begin{tabular}{c l l l l l l l l}
		$ $& \verb|NE|& \verb|ED|& \verb|BA|& \verb|CK|& \verb|UP|& \verb|NO|& \verb|WX|& \\
		$ $& 1304& 0403& 0100& 0210& 2015& 1314& 2223& \\
		$ $& $1304^{71}$& $403^{71}$& $100^{71}$& $210^{71}$& $2015^{71}$& $1314^{71}$& $2223^{71}$& \\
		\hline
		$\equiv$& 2755& 3464& 2222& 3183& 1023& 2590& 2540& $\mbox{ mod }3637$
	\end{tabular}
\end{table}
Then we get \verb|2755 3464 2222 3183 1023 2590 2540| as our ciphertext.
\subsubsection{}
\begin{enumerate}[(a)]
  \item
    We know that $d$ has to satisfy $71(d) \equiv 1\mbox{ mod }3636$.
    Observe then that $71^{\phi(3636)}\equiv 1\mbox{ mod }3636$
    so then we see that $d\equiv 71^{\phi(3636)-1}\equiv2663\mbox{ mod }3636$.

  \item
    \begin{table}[htb!]
      \begin{center}
      \begin{tabular}{c l l l l l l l l l}
        $ $& $1333$& $0513$& $0452$& $0767$& $2130$& $1395$& $1097$& $3597$ \\
        $ $& $1333^{2663}$& $513^{2663}$& $452^{2663}$& $767^{2663}$& $2130^{2663}$& $1395^{2663}$& $1097^{2663}$& $3597^{2663}$ \\
        \hline \\
        $\equiv$& 1200& 1308& 0002& 1413& 1907& 0411& 1414& 1804\\
        $ $& \verb|MA|& \verb|NI|& \verb|AC|& \verb|ON|& \verb|TH|& \verb|EL|& \verb|OO|& \verb|SE|
      \end{tabular}
      \end{center}
    \end{table}
    Thus, the message is \verb|MANIACONTHELOOSE|.

\end{enumerate}
\subsubsection{}
\begin{enumerate}[(a)]
  \item 
  $e\cdot\ind_r 18 \equiv \ind_r 11\mbox{ mod }28$

  \item
    First note that $\ind_2 18 = 11$ and $\ind_2 11 = 25$ so we then get
    $11e\equiv 25\mbox{ mod }28 \implies e=15$. 

  \item
    We want to find $d$ such that $15d \equiv 1\mbox{ mod }28 \implies d=15$.

  \item
    Take the ciphertext and decrypt it like so, $P \equiv C^{15}\mbox{ mod }29$,
    \begin{center}
      \texttt{18 12 00 17 19 24 15 00 13 19 18}$\implies$\verb|SMARTYPANTS|
    \end{center}

  \end{enumerate}
\subsubsection{}
\begin{enumerate}[(a)]
  \item
    \begin{enumerate}[i.]
    \item
      The message becomes \verb|EV EI SL IS TE NI NG| which in turn then becomes,
      \begin{center}
        \verb|0421 0408 1811 0818 1904 1308 1306|
      \end{center}

    \item
      Take the text and sign it like so, 
      $S\equiv P^{2599}\mbox{ mod }3551$,
      \begin{center}
        \verb|0724 2163 0430 2945 2663 3473 0993|
      \end{center}
      
      
    \item
      Encrypt the text like so, $C \equiv S^{27}\mbox{ mod }4189$,
      \begin{center}
        \verb|3425 0345 0521 3573 1463 1546 0567|
      \end{center}

    \end{enumerate}

  \item
    \begin{enumerate}[i.]
    \item
      Take the text and decrypt it like so,
      $S\equiv C^{1203}\mbox{ mod } 4189$
      \begin{center}
        \verb|1616 2799 3244 1237 1617 0457|
      \end{center}

    \item
      Take the text and unsign it like so,
      $P\equiv S^{103}\mbox{ mod }3511$
      \begin{center}
        \verb|1800 2104 2414 2017 1804 1105|$\implies$\verb|SAVEYOURSELF|
      \end{center}

    \end{enumerate}

\end{enumerate}
\subsubsection{}
First we factor $n=288319 = 401\cdot 719$, then $\phi(n)$ follows as
  $\phi(n)=(p-1)(q-1)=400\cdot718=287200$. Then we want to find $d$
  such that $(5201)d \equiv 1\mbox{ mod }287200$, which results in $d=272401$.
  Then we decrypt the message like so, $P\equiv C^{272401}\mbox{ mod }288319$
  \begin{center}
    \verb|220724 181412 041908 120418 082104 010411 080421 040300 181200 132400 181808| \\
    \verb|230812 151418 180801 110419 070813 061801 040514 170401 170400 100500 181923|
  \end{center}
  Thus, the message is 
  \begin{center}
    \verb|WHYSOMETIMESIVEBELIEVEDASMANYASSIXIMPOSSIBLETHINGSBEFOREBREAKFASTX|
  \end{center}
\subsubsection{}
Since we have $\gcd(e_1, e_2)=1$
  we need to find $\alpha$ and $\beta$ such that $\alpha e_1 + \beta e_2 = 1$.
  Using the Euclidean Algorithm we get $\alpha = -24$ and $\beta = 49$.
  Since $\alpha=-24$ we have to first find the multiplicative inverse of $C_1$ before we can
  raise it to the power of $\alpha$. We get $4280^{-1}\equiv 1097\mbox{ mod }4757$.
  Then, all together we have
  $$1097^{24}\cdot 330^{49}\equiv 2404\mbox{ mod }4757$$
  Thus, $P=2404$.
\subsubsection{}
We have the system of linear congruences,
  \begin{align*}
    x &\equiv 1533 \mbox{ mod } 5353 \\
    x &\equiv 3561 \mbox{ mod } 5251 \\
    x &\equiv  835 \mbox{ mod } 5893
  \end{align*}
  We get $x= 2893640625= P^3 = (1425)^3 \implies P=$\verb|1425|.
\subsubsection{}
\begin{enumerate}[(a)]
  \item
    We have that 
    $\overline C \equiv Cr^e \equiv 156\cdot888^{27}\equiv 1099\mbox{ mod } 4189$.

  \item
    We get that $r^{-1}=2203$ then we take the trash and multiplies it by the inverse of $r$,
    $2203\cdot 662\equiv 614\mbox{ mod }4189$ so we get $P$ to be \verb|0614|.

\end{enumerate}

%%%%%%%%%%%%%%%
%%% Section %%%
%%%%%%%%%%%%%%%
\setcounter{subsection}{11}
\subsection{Additional Topics}
\rule{\textwidth}{1pt}\\
\subsubsection{}
\begin{enumerate}[(a)]
  \item
    Set $x_0 = 2$ and $f(x) =x^2+1$. Then we have,
    \begin{align*}
      x_1 &\equiv 2^2 + 1\equiv 5\mbox{ mod }143 \\
      x_2 &\equiv 5^2 + 1\equiv 26\mbox{ mod }143 &\gcd(26-5, 143)=1\\
      x_3 &\equiv 26^2 + 1\equiv 104\mbox{ mod }143 \\
      x_4 &\equiv 104^2 + 1\equiv 92\mbox{ mod }143 &\gcd(92-26, 143)=11
    \end{align*}
    So we get $11$ as a factor of $143$.

  \item
    Set $x_0=2$ and $f(x) = x^2+1$. Then we have,
    \begin{align*}
      x_1 &\equiv 2^2 + 1\equiv 5\mbox{ mod }5473 \\
      x_2 &\equiv 5^2 + 1\equiv 26\mbox{ mod }5473 &\gcd(26-5, 5473)=1\\
      x_3 &\equiv 26^2 + 1\equiv 677\mbox{ mod }5473 \\
      x_4 &\equiv 677^2 + 1\equiv 4071\mbox{ mod }5473 &\gcd(4071-26, 5473)=1\\
      x_5 &\equiv 4071^2 + 1\equiv 798\mbox{ mod }5473 \\
      x_6 &\equiv 798^2 + 1\equiv 1937\mbox{ mod }5473 &\gcd(1937-677, 5473)=1\\
      x_7 &\equiv 1937^2 + 1\equiv 2965\mbox{ mod }5473 \\
      x_8 &\equiv 2965^2 + 1\equiv 1588\mbox{ mod }5473 &\gcd(1588-4071, 5473)=13
    \end{align*}
    So we get $13$ as a factor of $5473$.

  \item
    Set $x_0=2$ and $f(x) = x^2+1$. Then we have,
    \begin{align*}
      x_1 &\equiv 2^2 + 1\equiv 5\mbox{ mod }234643 \\
      x_2 &\equiv 5^2 + 1\equiv 26\mbox{ mod }234643 &\gcd(26-5, 234643)=1\\
      x_3 &\equiv 26^2 + 1\equiv 677\mbox{ mod }234643 \\
      x_4 &\equiv 677^2 + 1\equiv 223687\mbox{ mod }234643 &\gcd(223687-26, 234643)=1\\
      x_5 &\equiv 223687^2 + 1\equiv 131364\mbox{ mod }234643 \\
      x_6 &\equiv 131364^2 + 1\equiv 150348\mbox{ mod }234643 &\gcd(150348-677, 234643)=97
    \end{align*}
    So we get $97$ as a factor of $234643$.
\end{enumerate}
\subsubsection{}
Suppose we know $X$ and $-Y$.  Note that from $-Y$ we can easily get $Y$.
  Then observe that $X+Y\equiv a+a\equiv 2a\mbox{ mod }p$ and $X+Y\equiv a+ (-a)\equiv 0\mbox{ mod }q$.
  So we have that $q\mid (X+Y)$ and $p\nmid (X+Y)$ (we know that $p\nmid (X+Y)$ since $p\mid(X+Y)\implies p\mid 2a\implies p\mid a\implies a\equiv 0\mbox{ mod }p$
  which is a contradiction). So we know $\gcd(X+Y, n)=q$ then it follows that $p = \frac{n}{q}$ and now we have
  both $p$ and $q$, the factors of $n$.
\subsubsection{}
From Alice's choice of $p=67$ and $q=83$ we have $n=5561$, and she sends $n$ to Bob.
  Bob chooses $b=123$, and finds $S\equiv b^2\equiv 4007\mbox{ mod }5561$, then he sends $4007$ to Alice.
  Alice then solves the equation $x^2\equiv 4007\mbox{ mod }5561$,
  \begin{itemize}
    \item First we have,
    $$\begin{array}{c}
      x\equiv 4007^{(67+1)/4}\equiv 56\mbox{ mod }67 \\
      x\equiv 4007^{(83+1)/4}\equiv 40\mbox{ mod }83
      \end{array}\Big\}\implies X\equiv 123$$
    Which gives us $X=123$ and $-X=5438$.
    \item Then we have,
    $$\begin{array}{c}
      x\equiv 4007^{(67+1)/4}\equiv 56\mbox{ mod }67 \\
      x\equiv -4007^{(83+1)/4}\equiv 43\mbox{ mod }83
    \end{array}\Big\}\implies Y\equiv 458$$
    Which gives us $Y=458$ and $-Y=5103$.
  \end{itemize}
  Alice then sends Bob on from the set of these four numbers, 
  $\{123, 458, 5103, 5438\}$. If Alice chooses either $458$ or $5103$
  then Bob will be able to factor $n$ (Bob can do this through $\gcd(123 - a, 5561)$ where $a$ is either
  458 or 5103, this will result in a factor of $n$) and thus win the coinflip!
\subsubsection{}
\begin{enumerate}
  \item
    Choose $p=3001$ to find a primitive root of $p$ we need an $r$ such that $\gcd(r,p)=1$
    and $\ord_p r = \phi(p)\implies \ord_{3001} r = 3000\implies r^{3000} \mbox{ mod }3001\implies r = 14$.

  \item
    Suppose we choose $a = 2718$, then $b\equiv 14^{2718} \equiv 1079\mbox{ mod } 3001$.
    The public key is then $(p,r,b)\implies(3001,14,1079)$.

  \item
    Suppose Alice wants to send Bob \verb|ELGAMALCRYPTOSYSTEM|. Then she uses
    the encryption function $\mathcal{E}(P) = (14^{k}, P\cdot 1079^{k})\mbox{ mod }3001$ on the plaintext, where $1\leq k\leq p-2$.

    \begin{table}[htb]
      \centering
      \begin{tabular}{ccccccccccc}
        &\verb|EL| &\verb|GA| &\verb|MA| &\verb|LC| &\verb|RY| &\verb|PT| &\verb|OS| &\verb|YS| &\verb|TE| &\verb|MX|\\
        $P =$ &0411 &0600 &1200 &1102 &1724 &1519 &1418 &2418 &1904 &1223
      \end{tabular}
    \end{table}
    \begin{align*}
      &k = 1, &\E(411)  = (14^{1},  411\cdot 1079^{1}) &\equiv (14, 2322) \mbox{ mod } 3001 \\
      &k = 2, &\E(600)  = (14^{2},  600\cdot 1079^{2}) &\equiv (196, 1830) \mbox{ mod } 3001 \\
      &k = 3, &\E(1200) = (14^{3}, 1200\cdot 1079^{3}) &\equiv (2744, 2825) \mbox{ mod } 3001 \\
      &k = 4, &\E(1102) = (14^{4}, 1102\cdot 1079^{4}) &\equiv (2404, 183) \mbox{ mod } 3001 \\
      &k = 5, &\E(1724) = (14^{5}, 1724\cdot 1079^{5}) &\equiv (645, 2838) \mbox{ mod } 3001 \\
      &k = 6, &\E(1519) = (14^{6}, 1519\cdot 1079^{6}) &\equiv (27, 1407) \mbox{ mod } 3001 \\
      &k = 7, &\E(1418) = (14^{7}, 1418\cdot 1079^{7}) &\equiv (378, 682) \mbox{ mod } 3001 \\
      &k = 8, &\E(2418) = (14^{8}, 2418\cdot 1079^{8}) &\equiv (2291, 872) \mbox{ mod } 3001 \\
      &k = 9, &\E(1904) = (14^{9}, 1904\cdot 1079^{9}) &\equiv (2064, 570) \mbox{ mod } 3001 \\
      &k = 1, &\E(1223) = (14^{1}, 1223\cdot 1079^{1}) &\equiv (14, 2178) \mbox{ mod } 3001
    \end{align*}
    Then our encrypted message is:
    \begin{center}
      \verb|(14, 2322) (196, 1830) (2744, 2825) (2404, 183) (645, 2838)|
      \verb|(27, 1407) (378, 682) (2291, 872) (2064, 570) (14, 2178)|
    \end{center}

  \end{enumerate}

%%%%%%%%%%%%%%%%%%%%%%%%%%%%%%%%%%%%%%%%%%%%%%%
% ███████╗██╗░░██╗░█████╗░███╗░░░███╗░██████╗ %
% ██╔════╝╚██╗██╔╝██╔══██╗████╗░████║██╔════╝ %
% █████╗░░░╚███╔╝░███████║██╔████╔██║╚█████╗░ %
% ██╔══╝░░░██╔██╗░██╔══██║██║╚██╔╝██║░╚═══██╗ %
% ███████╗██╔╝╚██╗██║░░██║██║░╚═╝░██║██████╔╝ %
% ╚══════╝╚═╝░░╚═╝╚═╝░░╚═╝╚═╝░░░░░╚═╝╚═════╝░ %
%%%%%%%%%%%%%%%%%%%%%%%%%%%%%%%%%%%%%%%%%%%%%%%

% \setcounter{section}{5}
% \setcounter{subsection}{0}

% %%%%%%%%%%%%%%%
% %%% Section %%%
% %%%%%%%%%%%%%%%
% \subsection{Sample A}
% \rule{\textwidth}{1pt}\\
% \subsubsection{}
% Let $10!$ be written as,
% 	\begin{align*}
% 		10! &= 1\cdot 2\cdot 3\cdot 4\cdot 5\cdot 6\cdot 7\cdot 8\cdot 9\cdot 10 \\
% 		&= 1\cdot 2\cdot 3\cdot 2^2\cdot 5\cdot (2\times 3)\cdot 7\cdot 2^3\cdot 3^2\cdot (2\times 5) \\
% 		&= 1\cdot 2^8\cdot 3^4\cdot 5^2\cdot 7
% 	\end{align*}
% 	Therefore, the prime factorization of $10!$ is $2^8 3^4 5^2 7$.

% \subsubsection{}
% Using Fermat's Little Theorem. 
% 	Well $13\nmid 11$ so $11^{12} \equiv 1\mod 13$. Then $67= 12(5) + 7$ so,
% 	\begin{align*}
% 		11^{67} = 11^{12(5)+7} = 11^{12^5}11^7 &\equiv (1)^{10} 11^{7} \mod 13 \\
% 		&\equiv 11^7 \mod 13 \\
% 		&\equiv 11 \cdot 1771561 \mod 13 \\
% 		&\equiv 11 (-1) \mod 13 \\
% 		&\equiv -11 \mod 13 \\
% 		&\equiv 2 \mod 13
% 	\end{align*}
% 	So $2$ is the least non-negative residue.

% \subsubsection{}
% Since $\gcd(12,40)=4\mid 28$ there exists a solution. We use the Euclidean Algorithm to solve
% 	$12x' + 40y' = 4$. This gives us $12(-3)+40(1)=4$, we want a 28 on the right hand side so 
% 	multiple by $7$. We then get $12(-21)+40(7)=28$, so $12(-21)\equiv 28\mod 40$. Therefore,
% 	$x_0 \equiv 19\mod 40$, so all solutions are then
% 	$$x\equiv 19 + 10k\mod 40k, k=0,1,2,3$$
% 	That is $x\equiv 19, 29, 39, 9\mod 40$

% \subsubsection{}
% Using the Euclidean Algorithm we do the following:
% 	\begin{align*}
% 		390 &= 5(72) + 30 \\
% 		72 &= 2(30) + 12 \\
% 		30 &= 2(12) + 6 \\
% 		12 &= 2(6) + 0
% 	\end{align*}
% 	So the gcd is $6$. Now the find the linear combination.
% 	\begin{align*}
% 		6 &= 1(30) - 2(12) \\
% 		&= 1(30) - 2(72 - 2(30)) \\
% 		&= 5(30) - 2(72) \\
% 		&= 5(390 - 5(72)) - 2(72) \\
% 		&= 5(390) - 27(72)
% 	\end{align*}
% 	Where $\alpha = 5$ and $\beta = -27$.

% \subsubsection{}
% Test to see if all $m_i$ are pairwise coprime, $\gcd(5,6)=\gcd(5,7)=\gcd(6,7)$. This means that
% 	$M=210$, $M_1=42$, $M_2=35$, and $M_3=30$.
% 	\begin{itemize}
% 		\item[] Solve for $y_1$:
% 			\begin{align*}
% 				42y_1 &\equiv 1\mod 5 \\
% 				2y_1 &\equiv 1\mod 5 \\
% 				y_1 &= 3
% 			\end{align*}
		
% 		\item[] Solve for $y_2$:
% 			\begin{align*}
% 				35y_2 &\equiv 1\mod 6 \\
% 				5y_2 &\equiv 1\mod 6 \\
% 				y_2 &= 5
% 			\end{align*}
		
% 		\item[] Solve for $y_3$:
% 			\begin{align*}
% 				30y_3 &\equiv 1\mod 7 \\
% 				2y_3 &\equiv 1\mod 7 \\
% 				y_3 &= 4
% 			\end{align*}
% 	\end{itemize}
% 	So we then get
% 	$$x = (2)(42)(3) + (1)(35)(5) + (4)(30)(4) \equiv 907\mod 210$$
% 	$$x\equiv 67\mod 210$$
% 	So the least non-negative residue is $67$.

% \subsubsection{}
% \begin{proof}
% 	$ $\\
% 	\begin{itemize}
% 		\item[] \textbf{Base Case:}
% 		\begin{itemize}
% 			\item[] Let $n=6$, $n! = 720$ and $6^3=216$, $720\geq 216$ so the base case is valid.
% 		\end{itemize}
% 		\item[] \textbf{Inductive Hypothesis:}
% 		\begin{itemize}
% 			\item[] Assume from the inductive hypothesis that the conclusion is true for some $n\geq 6$.
% 			This implies that $n!\geq n^3$.
% 		\end{itemize}
% 		\item[] \textbf{Inductive Step:}
% 		\begin{itemize}
% 			\item[] Then consider the equation to $n+1$:
% 			\begin{align*}
% 				(n+1)! &\geq (n+1)^3 \\
% 				(n+1)(n!) &\geq (n+1)^3 \\
% 				(n+1)n^3 &> (n+1)^3 \hspace{5mm}\text{by IH} \\
% 				n^3 &> (n+1)^2 \\
% 				n^3 &> n^2 + 2n + 1
% 			\end{align*}
% 			Which is true for any $n\geq 3$.
% 		\end{itemize}
% 		\item[] Thus for all $n\geq 6$,
% 		$$n!\geq n^3$$
% 	\end{itemize}
% \end{proof}

% \subsubsection{}
% The set $S_1$ is not well-ordered because the subset $(0,0)\cap \reals$ has no least element.
% Likewise, the set $S_2$ is also not well-ordered because the set itself has no least element.

% \subsubsection{}
% \begin{proof}
% 	Suppose that $\sqrt{2}$ is rational, this means that $\sqrt{2}$ is of the form $\frac{a}{b}$, $a,b\in\integers^+$.
% 	Then $2=\frac{a^2}{b^2}$ so $a^2=2b^2$. Because $a^2$ and $b^2$ are both squared the prime factorizations
% 	of both are even, but $a^2=2b^2$ implies there is an odd number of prime factorizations for $2$. This contradicts
% 	uniqueness of prime factors.
% \end{proof}

% \subsubsection{}
% \begin{proof}
% 	Given that $a\mid c$, $b\mid c$, and $d^2\mid c$ given that $d=\gcd(a,b)$ we can \emph{not} conclude
% 	that $ab\mid c$. We will show this with a simple contradiction, let $a=2$, $b=4$, $c=4$.
% 	We know that $2\mid 4$ and $4\mid 4$, it follows that $\gcd(2,4)=2^2\mid 4$ but 
% 	$ab\nmid c$ because $2\cdot 4\nmid 4$ because $8>4$. So the statement is false.
% \end{proof}

% %%%%%%%%%%%%%%%
% %%% Section %%%
% %%%%%%%%%%%%%%%
% \setcounter{subsection}{0}
% \subsection{Sample B}
% \rule{\textwidth}{1pt}\\
% \subsubsection{}
% \begin{enumerate}[(a)]
% 	\item
% 	We first list the primes up to $18$, $\{2,3,5,7,11,13,17\}$. We see that there are $7$ primes,
% 	therefore $\pi(18)=7$.
	
% 	\item
% 	Since $a>b$ we know a subset 
% 	$\{\frac{2}{1}, \frac{3}{2}, \frac{4}{3},\cdots\}$ exists, and it does not have a least element.
% 	Since the subset does not have a least element, the set is not well-ordered.
	
% 	\item
% 	From section 2.2 we know that for very large $x$, $\pi(x)=\frac{x}{\ln x}$.
% 	So there are, approximately, 
% 	$$\frac{2000000000}{\ln(2000000000)} - \frac{1000000000}{\ln(1000000000)}$$
% 	primes between one and two billion.

% \end{enumerate}

% \subsubsection{}
% Zeros at the end of numbers are from multiples of 10 which are pairs of 2 and 5, so
% we find the number of pairs of 2's and 5's to find the number of zeros. Let $d_n(x)$
% represent the sum of the numbers divisible by all powers of $n$ less than $x$. 
% $$d_2(1000!) = 500 + 250 + 125 + 62 + 31 + 15 + 7 + 3 + 1 = 994$$
% $$d_5(1000!) = 200 + 40 + 8 + 1 = 249$$
% Thus, there can only be 249 pairs of 2's and 5's, so there are only 249 
% 10's, so there are 249 zeros at the end of $(1000!)$.

% \subsubsection{}
% \begin{enumerate}[(a)]
% 	\item
% 	$6\mid 3\cdot4$ but $6\nmid 3$ and $6\nmid 4$.
	
% 	\item
% 	$2\mid 4$ and $2\mid 6$ but $4\nmid 6$.
	
% 	\item
% 	$3\mid 6$ and $3\mid 12$ but $\gcd(6,12)=6\neq 3$.

% \end{enumerate}

% \subsubsection{}
% $$\prod_{j=1}^{n}\left(1+\frac{2}{j}\right)= \prod_{j=1}^{n}\left(\frac{j+2}{j}\right) = \frac{3}{1}\times\frac{4}{2}\times\cdots\times\frac{n+2}{n} = \frac{(n+2)(n+1)}{2}$$

% \subsubsection{}
% \begin{proof}
% 	$ $
% 	\begin{itemize}
% 		\item[] \textbf{Base Case:}
% 		\begin{itemize}
% 			\item[] Let $n=1$, $2^1=2^{1+1}-2$ is true, so the base case is valid.
% 		\end{itemize}
% 		\item[] \textbf{Inductive Hypothesis:}
% 		\begin{itemize}
% 			\item[] Assume from the inductive hypothesis that the conclusion is true for some $n\geq 1$.
% 			This implies that $2^1+2^2+\cdots+2^n=2^{n+1}-2$.
% 		\end{itemize}
% 		\item[] \textbf{Inductive Step:}
% 		\begin{itemize}
% 			\item[] Then consider the equation to $n+1$:
% 			\begin{align*}
% 				2^1+2^2+\cdots+2^{n+1} &= 2^1+2^2+\cdots2^n+2^{n+1} \\
% 				&= 2^{n+1}-2 + 2^{n+1} \hspace{5mm}\text{ by IH} \\
% 				&= 2^{(n+1)+1}-2
% 			\end{align*}
% 		\end{itemize}
% 		\item[] Thus for all $n\geq 1$, $$2^1+2^2+\cdots+2^n=2^{n+1}-2$$    
% 	\end{itemize}
% \end{proof}

% \subsubsection{}
% Factor $n^2-5n+6$ out to be of the form $(n-2)(n-3)$. For this polynomial to be prime
% 	we need one factor to be $\pm1$ and the other to be a prime. We have four cases:
% 	\begin{itemize}
% 		\item If $n-2=1\implies n=3$ then $n^2-5n+6=0$ which is not prime.
% 		\item If $n-2=-1\implies n=1$ then $n^2-5n+6=2$ which is prime.
% 		\item If $n-3=1\implies n=4$ then $n^2-5n+6=2$ which is prime.
% 		\item If $n-3=-1\implies n=2$ then $n^2-5n+6=0$ which is not prime.
% 	\end{itemize}
% 	So the only values of $n$ such that $n^2-5n+6$ is prime is $n=1,4$.

% \subsubsection{}
% We know that $\gcd(a,7a+p)=\gcd(a,p)$, but since $a<p$ and the only divisors of $p$ are $1$ and
% $p$ we know that $a\nmid p$, therefore $\gcd(a,p)=1$.

% \subsubsection{}
% \begin{proof}
% 	Suppose that $\sqrt{6}$ is rational, this means that $\sqrt{6}$ is of the form $\frac{a}{b}$, $a,b\in\integers^+$.
% 	Then $6=\frac{a^2}{b^2}$ so $a^2=6b^2$. Because $a^2$ and $b^2$ are both squared the prime factorizations
% 	of both are even, but $a^2=6b^2$ implies there is an odd number of prime factorizations for $2$ and $3$. This contradicts
% 	uniqueness of prime factors.
% \end{proof}

% \subsubsection{}
% \begin{proof}
% 	Suppose that $a^n\mid b^n$, this implies that $b^n = ka^n$ for some $k\in\integers$.
% 	We know that any prime in the prime factorization of $k$ must be to the power of $\alpha n$.
% 	This implies that $k=p_1^{\alpha_1 n} p_2^{\alpha_2 n} \cdots p_i^{\alpha_i n}$ which in turn
% 	implies that $k=(p_1^{\alpha_1} p_2^{\alpha_2}\cdots p_i^{\alpha_i})^n$. From this we know that 
% 	$k$ is a perfect square, meaning that $\sqrt{k}\in\integers^+$, thus $a\sqrt{k}=b$ and
% 	$a\mid b$.
% \end{proof}

% %%%%%%%%%%%%%%%
% %%% Section %%%
% %%%%%%%%%%%%%%%
% \setcounter{subsection}{1}
% \subsection{Sample A}
% \rule{\textwidth}{1pt}\\
% \subsubsection{}
% \subsubsection{}
% \begin{proof}
% 	Since $\gcd(6,n)=1$ we know that $\gcd(2,n)=\gcd(3,n)=1$.
% 	Then observe:
% 	$\phi(3n) = \phi(3)\phi(n) = 2 \phi(n)$, and
% 	$2\phi(2n) = 2\phi(2)\phi(n) = 2\phi(n)$.
% 	So we see that $\phi(3n) = 2\phi(2n)$.
% \end{proof}
% \subsubsection{}
% If $n=p_1^{\alpha_1}\cdots p_k^{\alpha_k}$ then $\tau(n) = (\alpha_1 +1) \cdots (\alpha_k +1)=12$.
% We see that we can have at most three distinct primes ($12 = 2\cdot 6 = 3\cdot 4 = 2\cdot 2 \cdot 3$).
% It then follows that $n=p^{11}$, $n=p_1 p_2^{5}$, $n=p_1^{2} p_2^{3}$, or $n=p_1 p_2 p_3^{2}$.
% \subsubsection{}
% \begin{proof}
% 	Suppose that $\gcd(p,n)=1$, then by the definition of a perfect number
% 	we have $2pn = \sigma(pn)= \sigma(p)\sigma(n) = \sigma(p) (2n)$.
% 	Then $\sigma(p) = p$, but this contradicts the fact that $\sigma(p)=p+1$.
% \end{proof}
% \subsubsection{}
% \begin{proof}
% 	Suppose $\gcd(a,b) = 1$ we then get:
% 	\begin{itemize}
% 		\item $a^{\phi(b)} \equiv 1 \mbox{ mod } b \implies a^{\phi(b)} + b^{\phi(a)} \equiv 1\mbox{ mod } b$
% 		\item $b^{\phi(a)} \equiv 1 \mbox{ mod } a \implies a^{\phi(b)} + b^{\phi(a)} \equiv 1\mbox{ mod } a$
% 	\end{itemize}
% 	Then it follows that $a^{\phi(b)} + b^{\phi(a)} \equiv 1\mbox{ mod }ab$.
% \end{proof}
% \subsubsection{}
% \begin{proof}
% 	$ $\\
% 	$\rightarrow$ Suppose $p\nmid n$, this implies that $\gcd(p, n) =1$. Then, 
% 	$$\phi(pn) = \phi(p)\phi(n) = (p-1)\phi(n)$$
% 	$\leftarrow$ Suppose $p\mid n$, then $n = p^{\alpha} \cdot k$ for some $\alpha, k\in\integers$ and
% 	$\gcd(p, k) =1$. Then,
% 	$$\phi(pn) = \phi(p \cdot p^{\alpha}k) = \phi(p^{\alpha +1} k) = \phi(p^{\alpha +1}) \phi(k) = (p^{\alpha + 1} - p^{\alpha}) \phi(k)$$
% 	$$ = p\phi(p^{\alpha})\phi(k) = p\phi(n) \neq (p-1)\phi(n)$$
% 	Therefore $p\nmid n \iff \phi(pn) = (p-1)\phi(n)$.
% \end{proof}
% \subsubsection{}
% \begin{enumerate}[(a)]
% 	\item $\ord_{17}3= 16$ and $\phi(17)=17-1=16$ therefore $3$ is a primitive root mod 17.
% 	\item 
% \end{enumerate}
% \subsubsection{}
% \subsubsection{}
% \subsubsection{}

% %%%%%%%%%%%%%%%
% %%% Section %%%
% %%%%%%%%%%%%%%%
% \setcounter{subsection}{1}
% \subsection{Sample B}
% \rule{\textwidth}{1pt}\\
% \subsubsection{}
% \subsubsection{}
% \subsubsection{}
% \subsubsection{}
% \subsubsection{}
% \subsubsection{}
% \subsubsection{}
% \subsubsection{}
% \subsubsection{}
% \subsubsection{}

\end{document}