\documentclass[class=article, crop=false]{standalone}
\usepackage[utf8]{inputenc}
\usepackage{import}
\usepackage[subpreambles=true]{standalone}
\usepackage{graphicx}
\usepackage{amsfonts}
\usepackage{mathrsfs}
\usepackage{mathtools}
\usepackage{enumerate}
\usepackage{fancyhdr}
\usepackage[colorlinks=true,linkcolor=black,anchorcolor=black,citecolor=black,filecolor=black,menucolor=black,runcolor=black,urlcolor=blue]{hyperref}
\usepackage{epsfig,amssymb,amsmath,multicol,tikz,pgfplots,amsthm,enumerate}

\def\naturals{{\mathbb N}}
\def\reals{{\mathbb R}}
\def\complex{{\mathbb C}}
\def\poly{{\mathbb P}}
\def\integers{{\mathbb Z}}
\def\rationals{{\mathbb Q}}
\def\irrationals{{\mathbb I}}
\def\inlinesum#1#2{\overset{#2}{\underset{#1}{\sum}}}
\def\inlineprod#1#2{\overset{#2}{\underset{#1}{\prod}}}


\begin{document}
	
\section{Answers to Problems}

%%%%%%%%%%%%%%%
%%% Section %%%
%%%%%%%%%%%%%%%
\subsection{The Integers}
\hfill \fbox{91/100}
\rule{\textwidth}{1pt}
\begin{enumerate}
%%%%%%%%%%%%%%%%%%%%%
%%%%% Problem 1 %%%%%
%%%%%%%%%%%%%%%%%%%%%
\item
  Determine whether each of the following sets is well-ordered.  If
  so, give a proof which relies on the fact that $\integers^+$ is well-ordered.
  If not, give an example of a subset with no least element.\hfill\fbox{10/10}
  \begin{enumerate}
  \item
	$\left\{a\,\big|\, a\in\integers,a>3\right\}$

	Is a subset of $\integers^+$ and therefore is well-ordered.
  \item
	$\left\{a\,\big|\, a\in\rationals,a>3\right\}$

	There is no least element so the set is not well-ordered.
  \item
	$\left\{\frac a2\,\big|\, a\in\integers,a\geq 10\right\}$

	Consider the set $\left\{a\,\big|\, a\in\integers,a\geq 10\right\}$, it is apparent that this is a subset of $\integers^+$ and therefore is well-ordered.
	So the set $\left\{\frac a2\,\big|\, a\in\integers,a\geq 10\right\}$ is also well-ordered because it holds a least element ($\frac{10}{5}$).
  \item
	$\left\{\frac 2a\,\big|\, a\in\integers,a>10\right\}$

	There is no least element so the set is not well-ordered.
  \end{enumerate}

%%%%%%%%%%%%%%%%%%%%%
%%%%% Problem 2 %%%%%
%%%%%%%%%%%%%%%%%%%%%
\item
  Suppose $a,b\in\integers^+$ are unknown.  Let
  $S=\left\{a-bk\,\big|\, k\in\integers,a-bk>0\right\}$.
  Explain why $S$ has a smallest element but no largest element.\hfill\fbox{3/10}

  Since $S$ is a subset of $\integers^+$ by well-ordering we know that $S$ has a least element, and because $k\in\integers$,
  $k$ can be $0$ and therefore there is no most element.

%%%%%%%%%%%%%%%%%%%%%
%%%%% Problem 3 %%%%%
%%%%%%%%%%%%%%%%%%%%%
\item
  Use the well-ordering property to show that
  $\sqrt 5$ is irrational.\hfill\fbox{10/10}
  \begin{proof}
	Suppose $\sqrt{5}$ is rational and is of the form $\frac{a}{b}$ where $a,b\in\mathbb{Z}^+$ and $b\neq 0$.
	Consider the set $S$,
	$$S = \left\{ k \mid k, k\sqrt{5}\in\mathbb{Z}^+ \right\}$$
	We know that $S$ is a subset of $\mathbb{Z}^+$ and that $b\in S$, by well-ordering this implies that $S$ has a least element.
	Let $l$ be the least element in $S$.\\
	Consider the properties of $l'$ where $l' = l\sqrt{5}-2l$,
	\begin{itemize}
	  \item $l'=l\sqrt{5}-2l= l(\sqrt{5}-2) \implies 0 < l' < l$.
	  \item Since $l\in S$ and $S\subset \mathbb{Z}^+$, both $l \text{ and }l\sqrt{5}\in\mathbb{Z}^+$ which implies $l'\in\mathbb{Z}^+$.
	  \item Since $l\in\mathbb{Z}^+$ we have $5l\in\mathbb{Z}^+$ and since $l\sqrt{5}\in\mathbb{Z}^+$ we have $l'\sqrt{5}= (l\sqrt{5}-2l)\sqrt{5}= 5l-2l\sqrt{5}\in\mathbb{Z}^+$.
	\end{itemize}
	It follows that $l'\in S$ but $l' < l$ which contradicts $l$ being the least element in $S$.
  \end{proof}

%%%%%%%%%%%%%%%%%%%%%
%%%%% Problem 4 %%%%%
%%%%%%%%%%%%%%%%%%%%%
\item
  Use the identity
  $$\frac1{k^2-1}=\frac12\left(\frac1{k-1}-\frac1{k+1}\right)$$
  to evaluate the following:\hfill\fbox{10/10}
  \begin{enumerate}

  \item
	$\inlinesum{k=2}{10}\frac1{k^2-1}$

	\begin{align*}
	  \inlinesum{k=2}{10}\frac{1}{k^2-1} &=\inlinesum{k=2}{10}\frac{1}{2}\left(\frac{1}{k-1}-\frac{1}{k+1}\right)= \frac{1}{2}\inlinesum{k=2}{10}\left(\frac{1}{k-1}-\frac{1}{k+1}\right) \\
	  &= \frac{1}{2}\left[\left(\frac{1}{1}-\frac{1}{3}\right) + \cdots + \left(\frac{1}{8}-\frac{1}{10}\right) + \left(\frac{1}{9}-\frac{1}{11}\right)\right] \\
	  &= \frac{1}{2}\left[\frac{1}{1} + \frac{1}{2} - \frac{1}{10} - \frac{1}{11}\right] \\
	  &= \frac{1}{2}\left(\frac{72}{55}\right) = \frac{36}{55}
	\end{align*}

  \item
	$\inlinesum{k=2}{n}\frac1{k^2-1}$

	$$\inlinesum{k=2}{n}\frac1{k^2-1}= \frac{1}{2}\left[\frac{1}{1}+\frac{1}{2} - \frac{1}{n} - \frac{1}{n+1}\right]$$

  \item
	$\inlinesum{k=1}{n}\frac1{k^2+2k}$
	\hspace{10pt}Hint: $k^2+2k=(???)^2-1$

	$$\inlinesum{k=1}{n} \frac{1}{k^2+2k} = \inlinesum{k=1}{n}\frac{1}{(k+1)^2 - 1} = \inlinesum{k=2}{n+1}\frac{1}{k^2 - 1}$$
	$$\inlinesum{k=2}{n+1}\frac{1}{k^2 - 1} = \frac{1}{2} \left[\frac{1}{1} + \frac{1}{2} - \frac{1}{n+1} - \frac{1}{n+2}\right]$$

  \end{enumerate}

%%%%%%%%%%%%%%%%%%%%%
%%%%% Problem 5 %%%%%
%%%%%%%%%%%%%%%%%%%%%  
\item
  Find the value of each of the following:\hfill\fbox{10/10}

  \begin{enumerate}

  \item
	$\inlineprod{j=2}{7}\left(1-\frac1j\right)$

	\begin{align*}
	  \inlineprod{j=2}{7}\left(1-\frac1j\right)
	&= \left[\frac{1}{2}\cdot\frac{2}{3}\cdot\frac{3}{4}\cdot\frac{4}{5}\cdot\frac{5}{6}\cdot\frac{6}{7}\right] \\
	&= \frac{1}{7}
	\end{align*}
	
  \item
	$\inlineprod{j=2}{n}\left(1-\frac1j\right)$

	$$\inlineprod{j=2}{n}\left(1-\frac1j\right) = \frac{1}{n}$$

  \item
	$\inlineprod{j=2}{n}\left(1-\frac1{j^2}\right)$
	\hspace{10pt}Hint: Be sneaky!

	$$\inlineprod{j=2}{n}\left(1-\frac1{j^2}\right) = \frac{n+1}{2n}$$

  \end{enumerate}

%%%%%%%%%%%%%%%%%%%%%
%%%%% Problem 6 %%%%%
%%%%%%%%%%%%%%%%%%%%%
\item
  Use weak mathematical induction to prove that
  $$\inlinesum{j=1}{n}j(j+1)=\frac{n(n+1)(n+2)}3$$
  for every positive integer $n$.\hfill\fbox{9/10}
  \begin{proof}
	$ $
	\begin{enumerate}
	  \item[] \textbf{Base Case:}
		\begin{enumerate}
		  \item[] Let $n=1$, $\sum_{j=1}^{1} j(j+1) = 2$ and $\frac{1(1+1)(1+2)}{3} = 2$, so the
		  base case is valid.
		\end{enumerate}
	  \item[] \textbf{Inductive Hypothesis:}
		\begin{enumerate}
		  \item[] Assume from the inductive hypothesis that the conclusion is true for some $n$.\\
		  This implies that $\sum_{j=1}^{n} j(j+1) = \frac{n(n+1)(n+2)}{3}$.
		\end{enumerate}
	  \item[] \textbf{Inductive Step:}
		\begin{enumerate}
		  \item[] Then consider the sum to $n+1$:
		  \begin{align*}
			\sum_{j=1}^{n+1}j(j+1) &= \sum_{j=1}^{n}j(j+1) + (n+1)((n+1)+1) \\
			&= \left[\frac{n(n+1)(n+2)}{3}\right] + (n+1)((n+1)+1) \text{ by IH} \\
			&= \frac{1}{3}\left(n(n+1)(n+2) + 3(n+1)(n+2)\right) \\
			&= \frac{1}{3}\left(n^3 + 3n^2 + 2n + 3n^2 + 9n + 6\right) \\
			&= \frac{1}{3}\left(n^3 + 6n^2 + 11n + 6\right) \\
			&= \frac{1}{3}\left((n+1)(n+2)(n+3)\right)
		  \end{align*}  
		\end{enumerate}
	  \item[] Thus for all $n\geq 1$, $$\sum_{j=1}^{n} j(j+1) = \frac{n(n+1)(n+2)}{3}$$     
	\end{enumerate}
  \end{proof}

%%%%%%%%%%%%%%%%%%%%%
%%%%% Problem 7 %%%%%
%%%%%%%%%%%%%%%%%%%%%
\item
  Use Weak Mathematical Induction
  to show that $f_nf_{n+2}=f_{n+1}^2+(-1)^{n+1}$ for all $n\geq 1$.\hfill\fbox{9/10}
  \begin{proof}
	$ $
	\begin{enumerate}
	  \item[] \textbf{Base Case:}
		\begin{enumerate}
		  \item[] Rewrite the statement $f_nf_{n+2}=f_{n+1}^2+(-1)^{n+1}$ to be $f_nf_{n+2}-f_{n+1}^2 =(-1)^{n+1}$. \\
		  Let $n=1$, $f_1 f_{1+2} - f_{1+1}^2 = 1\cdot2 - 1 = 1$ and $(-1)^{1+1} = 1$, so the base case is valid.
		\end{enumerate} 
	  \item[] \textbf{Inductive Hypothesis:}
		\begin{enumerate}
		  \item[] Assume from the inductive hypothesis that the conclusion is true for some $n$.\\
		  This implies that $f_nf_{n+2}-f_{n+1}^2 =(-1)^{n+1}$
		\end{enumerate}
	  \item[] \textbf{Inductive Step:}
		\begin{enumerate}
		  \item[] Then consider the equation to $n+1$:
			\begin{align*}
			  f_{(n+1)} f_{(n+1)+2} - f_{(n+1)+1}^2 &= f_{n+1} f_{n+3} - f_{n+2}^2 \\
			  &= f_{n+1} \left(f_{n+1} + f_{n+2}\right) - f_{n+2}^2 \\
			  &= f_{n+1}^2 + f_{n+1}f_{n+2} - f_{n+2}^2 \\
			  &= f_{n+1}^2 + f_{n+2} \left(f_{n+1} - f_{n+2}\right) \\
			  &= f_{n+1}^2 + f_{n+2} \left( - f_{n}\right) \\
			  &= -\left(f_n f_{n+2} - f_{n+1}^2\right) \\
			  &= - (-1)^{n+1} \hspace{5mm}\text{by IH} \\
			  &= (-1)^{n+2}
			\end{align*} 
		\end{enumerate}
	  \item[] Thus for all $n\geq 1$, $$f_nf_{n+2}-f_{n+1}^2 =(-1)^{n+1}$$
	\end{enumerate}
  \end{proof}

%%%%%%%%%%%%%%%%%%%%%
%%%%% Problem 8 %%%%%
%%%%%%%%%%%%%%%%%%%%%  
\item
  Use weak mathematical induction to show that
  a $2^n\times2^n$ chessboard with a corner missing can be tiled
  with pieces shaped like
  \begin{picture}(20,20)
	\put(0,-5){\line(0,1){20}}
	\put(10,-5){\line(0,1){20}}
	\put(20,-5){\line(0,1){10}}
	\put(0,-5){\line(1,0){20}}
	\put(0,5){\line(1,0){20}}
	\put(0,15){\line(1,0){10}}
  \end{picture}
  \,
  for every integer $n\geq 0$.\hfill\fbox{10/10}
  \begin{proof}
	$ $
	\begin{enumerate}
	  \item[] \textbf{Base Case:}
		\begin{enumerate}
		  \item[] Let $n=1$, $2^1 \times 2^1$ is a $2\times2$ chessboard with a corner missing and can 
		  be tiled by one tromino, so the base case is valid.
		\end{enumerate} 
	  \item[] \textbf{Inductive Hypothesis:}
		\begin{enumerate}
		  \item[] Assume from the inductive hypothesis that the conclusion is true for some $n$. This implies that any
		  $2^n \times 2^n$ chessboard with a corner missing can be tiled with trominoes.
		\end{enumerate}
	  \item[] \textbf{Inductive Step:}
		\begin{enumerate}
		  \item[] Then consider a $2^{n+1} \times 2^{n+1}$ chessboard.
		  \begin{itemize}
			\item Divide the $2^{n+1}\times 2^{n+1}$ chessboard into four quadrants of size $2^n \times 2^n$.
			\item By the Inductive Hypothesis we know that each $2^n\times 2^n$ has one corner missing.
			\item There are then four empty squares in the $2^{n+1}\times 2^{n+1}$ board.
			\item Rotate each quadrant such that the four empty squares are in the center of the board.
			\item Add another tromino into the board leaving only one empty square.
			\item Rotate the quadrant with the empty square such that the empty square is in the corner of the board.
			\item Therefore the $2^{n+1}\times 2^{n+1}$ chessboard can be tiled by trominoes with a corner missing.
		  \end{itemize} 
		\end{enumerate}
	  \item[] Thus, every $2^n \times 2^n$ chessboard with a corner missing can be tiled with trominoes.
	\end{enumerate}    
  \end{proof}

%%%%%%%%%%%%%%%%%%%%%
%%%%% Problem 9 %%%%%
%%%%%%%%%%%%%%%%%%%%%
\item
  Define:
  $$H_{2^n}=\inlinesum{j=1}{2^n}\frac1j$$
  Use weak mathematical induction to prove that
  for all $n\geq 1$ we have $H_{2^n}\leq 1+n$.\hfill\fbox{10/10}
  \begin{proof}
	$ $
	\begin{enumerate}
	  \item[] \textbf{Base Case:}
		\begin{enumerate}
		  \item[] Let $n=1$, $H_{2^1}=\inlinesum{j=1}{2^n}\frac1j = \frac32$ and $\frac32 \leq 2$, so the base case is valid.
		\end{enumerate} 
	  \item[] \textbf{Inductive Hypothesis:}
		\begin{enumerate}
		  \item[] Assume from the inductive hypothesis that the conclusion is true for some $n$. \\
		  This implies that $\inlinesum{j=1}{2^n}\frac1j \leq 1+n$.
		\end{enumerate}
	  \item[] \textbf{Inductive Step:}
		\begin{enumerate}
		  \item[] Then consider the equation to $n+1$:
			\begin{align*}
			  H_{2^{n+1}} &= \sum_{j=1}^{2^{n+1}} \frac{1}{j} \\
			  &= \sum_{j=1}^{2^n} \frac{1}{j} + \sum_{j=2^n +1}^{2^{n+1}} \frac{1}{j} \\
			  &\leq \left[1+n\right] + \sum_{j=2^n +1}^{2^{n+1}} \frac{1}{j} \hspace{5mm}\text{by IH} \\
			  &\leq \left[1+n\right] + \frac{1}{2^n +1}+\cdots +\frac{1}{2^{n+1}}\\
			  &\leq \left[1+n\right] + 2^n\cdot \frac{1}{2^{n+1}} \\
			  &\leq \frac{3}{2} + n \leq 2 + n
			\end{align*}
		\end{enumerate}
	  \item[] Thus for all $n\geq 1$, $$H_{2^n}\leq 1+n$$
	\end{enumerate}
  \end{proof}

%%%%%%%%%%%%%%%%%%%%%
%%%%% Problem 10 %%%%
%%%%%%%%%%%%%%%%%%%%%
\item
  Use strong mathematical induction to prove that every amount of
  postage over $53$ cents can be formed using $7$-cent and $10$-cent stamps.\hfill\fbox{10/10}
  \begin{proof}
	$ $
	\begin{enumerate}
	  \item[] \textbf{Inductive Step:}
		\begin{enumerate}
		  \item[]
		  Assume we can do $54,\cdots,k$. Because $k-6$ is in the $54,\cdots,k$ we can do $k-6$ then add a $7$-cent stamp.
		  $k-6$ is in $54,\cdots,k$ only if $k-6\geq 54 \equiv k\geq60$.
		  Thus, the inductive step is only valid for $k=60,61,\cdots$ to get to the next $k+1$.
		\end{enumerate} 
	  \item[] \textbf{Base Case:}
		\begin{enumerate}
		  \item[] Must do $54,55,56,57,58,59,60$ as base cases.
		  \begin{align*}
			54 &= 2(7\text{-cent}) + 4(10\text{-cent})\\
			55 &= 5(7\text{-cent}) + 2(10\text{-cent})\\
			56 &= 8(7\text{-cent})\\
			57 &= 1(7\text{-cent}) + 5(10\text{-cent})\\
			58 &= 4(7\text{-cent}) + 3(10\text{-cent})\\
			59 &= 7(7\text{-cent}) + 1(10\text{-cent})\\
			60 &= 6(10\text{-cent})
		  \end{align*} 
		\end{enumerate}
	\end{enumerate}
  \end{proof}

\end{enumerate}

%%%%%%%%%%%%%%%
%%% Section %%%
%%%%%%%%%%%%%%%
\subsection{Primes and GCDs}
\hfill \fbox{69/80}
\rule{\textwidth}{1pt}
\begin{enumerate}
\item
 Use the Euclidean Algorithm to calculate $d=\mbox{gcd}(510,140)$
  and then use the result to find $\alpha$ and $\beta$ so
  that $d=510\alpha+140\beta$.\hfill\fbox{10/10}

  Need to find gcd$(510,140)$.
  \begin{align*}
	510 &= 3(140) + 90 \\
	140 &= 1(90) + 50 \\
	90 &= 1(50) + 40 \\
	50 &= 1(40) + 10 \\
	40 &= 4(10) + 0
  \end{align*}

  So the gcd is 10. Now to find the linear combination.
  \begin{align*}
	10 &= 1(50) - 1(40) \\
	&= 1(50) - 1(90-1(50)) \\
	&= 2(50) - 1(90) \\
	&= 2(140-1(90)) - 1(90) \\
	&= 2(140) -3(90) \\
	&= 2(140) -3(510-3(140)) \\
	&= -3(510) +11(140) \\
	&= \alpha a+ \beta b
  \end{align*}
  where $\alpha = -3$ and $\beta = 11$.

\item
  Use the Euclidean Algorithm to show that if $k\in\integers^+$ that
  $3k+2$ and $5k+3$ are relatively prime.\hfill\fbox{8/10}

  Need to show that gcd$(3k+2, 5k+3)=1$ for all $k\in\integers^+$.
  \begin{align*}
	5k+3 &= 1(3k+2) + (2k+1) \\
	3k+2 &= 1(2k+1) + (k+1) \\
	2k+1 &= 1(k+1) + k \\
	k+1 &= 1(k) + 1
  \end{align*}
  So the gcd$(3k+2,5k+3)=1$, therefore $3k+2$ and $5k+3$ are relatively prime.

\item
  How many zeros are there at the end of $(1000!)$?  Do not do this
  by brute force.  Explain your method.\hfill\fbox{10/10}

  Zeros at the end of numbers are from multiples of 10 which are pairs of 2 and 5, so
  we find the number of pairs of 2's and 5's to find the number of zeros. Let $d_n(x)$
  represent the sum of the numbers divisible by all powers of $n$ less than $x$. 
  $$d_2(1000!) = 500 + 250 + 125 + 62 + 31 + 15 + 7 + 3 + 1 = 994$$
  $$d_5(1000!) = 200 + 40 + 8 + 1 = 249$$
  Thus, there can only be 249 pairs of 2's and 5's, so there are only 249 
  10's, so there are 249 zeros at the end of $(1000!)$.

\item
  Let $a=1038180$ and $b=92950$.
  First find the prime factorizations of $a$ and $b$.
  Then use these to calculate $\mbox{gcd}(a,b)$ and $\mbox{lcm}(a,b)$.\hfill\fbox{10/10}

  \begin{itemize}
	\item[] Find the prime factorization of $a$.
		\begin{align*}
		  1038180 &= 2^2 (259545) \\
		  &= 2^2 3^1 (86515) \\
		  &= 2^2 3^1 5^1 (17303) \\
		  &= 2^2 3^1 5^1 11^3 (13) \\
		  &= 2^2 3^1 5^1 11^3 13^1
		\end{align*} 
	\item[] Find the prime factorization of $b$. 
		\begin{align*}
		  92950 &= 2^1 (46475) \\
		  &= 2^1 5^2 (1859) \\
		  &= 2^1 5^2 11^1 (169) \\
		  &= 2^1 5^2 11^1 13^2
		\end{align*}
  \end{itemize}
  Now, to find the gcd($a,b$) and lcm($a,b$).
  $$\text{gcd}(a,b) = \text{gcd}(2^2 3^1 5^1 11^3 13^1, 2^1 5^2 11^1 13^2) = 2^1 5^1 11^1 13^1 = 1430$$
  $$\text{lcm}(a,b) = \text{lcm}(2^2 3^1 5^1 11^3 13^1, 2^1 5^2 11^1 13^2) = 2^2 3^1 5^2 11^3 13^2 = 67481700$$

\item
  Which pairs of integers have gcd of 18 and lcm of 540?  Explain.\hfill\fbox{10/10}

  \begin{itemize}
	\item[] Find the prime factorization of 18.
		\begin{align*}
		  18 &= 2^1 (9) \\
		  &= 2^1 3^2
		\end{align*} 
	\item[] Find the prime factorization of 540. 
		\begin{align*}
		  540 &= 2^2 (135) \\
		  &= 2^2 3^3 (5) \\
		  &= 2^2 3^3 5^1
		\end{align*}
  \end{itemize}
  
  From the prime factors of 18 and 540 we know that $x = 2^a 3^b 5^c$ and $y = 2^e 3^f 5^g$. The gcd is the
  minimum power of common prime factors, similarly the lcm is the maximum power of common prime factors. Therefore,
  the list of all possible pairs of integers is:
  \begin{align*}
	x=2^1 3^2 5^0 &, y=2^2 3^3 5^1 \\
	x=2^1 3^3 5^0 &, y=2^2 3^2 5^1 \\
	x=2^2 3^2 5^0 &, y=2^1 3^3 5^1 \\
	x=2^2 3^3 5^0 &, y=2^1 3^2 5^1
  \end{align*}

\item
  Suppose that $a\in\integers$ is a perfect square
  divisible by at least two distinct primes.
  Show that $a$ has at least seven distinct factors.\hfill\fbox{5/10}

  Since $a$ is a perfect square it can be represented by the form $a=b^2$, and since $a$ has 
  at \emph{least} 2 prime factors we can say that $b=p_1^{\alpha} p_2^{\beta}$. 
  It follows that $a=p_1^{2\alpha} p_2^{2\beta}$. Therefore $a$ has factors
  $1, p_1, p_2, p_1, p_2, p_1^2, p_2^2, a$.

\item
  Show that if $a,b\in\integers^+$ with $a^3\big|b^2$ then $a\big|b$.\hfill\fbox{10/10}

  Let $a= p_1^{\alpha_1} p_2^{\alpha_2}\cdots p_n^{\alpha_n}$ and $b= p_1^{\beta_1} p_2^{\beta_2}\cdots p_n^{\beta_n}$.
  Since $a^3\mid b^2$ we know that, 
  $$p_1^{3\alpha_1} p_2^{3\alpha_2}\cdots p_n^{3\alpha_n} \Big| p_1^{2\beta_1} p_2^{2\beta_2}\cdots p_n^{2\beta_n}$$
  Therefore, $3\alpha_n \leq 2\beta_n$. Now to show $a\mid b$ we need to show that $\alpha \leq \beta$.
  $$3\alpha\leq 2\beta \implies \alpha \leq \frac{2\beta}{3} \leq \beta$$
  Thus, if $a^3\mid b^2$ then $a\mid b$.

\item For which positive integers $m$ is each of the following statements true:\hfill\fbox{6/10}
\begin{enumerate}
\item
  $34\equiv 10 \mod m$ \\
  $$m = 12, 24$$
\item
  $1000\equiv 1 \mod m$ \\
  $$m = 3, 9, 27, 37, 111, 333, 999$$
\item
  $100\equiv 0 \mod m$ \\
  $$m = 1, 2, 4, 5, 10, 20, 25, 50, 100$$
\end{enumerate}

\end{enumerate}

%%%%%%%%%%%%%%%
%%% Section %%%
%%%%%%%%%%%%%%%
\subsection{Congruences}
\hfill \fbox{/100}
\rule{\textwidth}{1pt}
\begin{enumerate}
\item
  Calculate the least positive residues modulo 47 of each of
  the following with justification:
  \begin{enumerate}
  \item $2^{543}$ \\\\
  Using binary expansion we see that $2^1 \equiv 2\mbox{ mod } 47$, $2^2 \equiv 4\mbox{ mod } 47$,
  $2^4 \equiv 16\mbox{ mod } 47$, $2^8 \equiv 21\mbox{ mod } 47$, $2^{16} \equiv 18\mbox{ mod } 47$, $2^{32} \equiv 42\mbox{ mod } 47$,
  $2^{64}\equiv 25\mbox{ mod } 47$, $2^{128}\equiv 14\mbox{ mod } 47$, $2^{256}\equiv 8\mbox{ mod } 47$, and $2^{512}\equiv 17\mbox{ mod } 47$.\\
  Then $543 = 512 + 16 + 8 + 4 + 2 + 1$ so,
  \begin{align*}
	2^{543} = 2^{512}2^{16}2^{8}2^{4}2^{2}2^{1} &\equiv \\
	&\equiv 17\cdot18\cdot 21\cdot 16\cdot 4\cdot 2 \mbox{ mod } 47 \\
	&\equiv 822528 \mbox{ mod } 47 \\
	&\equiv 28 \mbox{ mod } 47
  \end{align*}
  So $28$ is the least non-negative residue.
  
  \item $32^{932}$ \\\\
  Using binary expansion we see that $32^1 \equiv 32\mbox{ mod }47$,
  $32^2 \equiv 37\mbox{ mod }47$, $32^4\equiv 6\mbox{ mod }47$, $32^8\equiv 47\mbox{ mod }47$,
  $32^{16}\equiv 27\mbox{ mod }47$, $32^{32}\equiv 24\mbox{ mod }47$, $32^{64}\equiv 12\mbox{ mod }47$,
  $32^{128}\equiv 3\mbox{ mod }47$, $32^{256}\equiv 9$, and $32^{512}\equiv 34$.
  Then $932 = 512 + 256 + 128 + 32 + 4$ so,
  \begin{align*}
	32^{932} = 32^{512}32^{256}32^{128}32^{32}32^4 &\equiv \\
	&\equiv 34\cdot 9\cdot 3\cdot 24\cdot 6 \mbox{ mod }47 \\
	&\equiv 132192 \mbox{ mod }47 \\
	&\equiv 28 \mbox{ mod }47
  \end{align*}
  So $28$ is the least non-negative residue.
  
  \item $46^{327349287323}$ \\\\
  Since $46\equiv -1\mbox{ mod } 47$ we know $46^{327349287323} \equiv (-1)^{327349287323}$.
  We also know $2\nmid 327349287323$ so $(-1)^{327349287323} \equiv (-1)^1 \equiv -1 \equiv 46\mbox{ mod } 47$.
  So $46$ is the least non-negative residue.

  \end{enumerate}

\item
  Exhibit a complete set of residues mod 17 composed entirely of
  multiples of 3. \\\\
  Let $S= \{0,1,2,\cdots,16\}$ be the set of residues mod 17. Because $\gcd(3,17)=1$ 
  the set consisting of only multiples of 3 would be,
  $$\{0,3,6,9,12,15,18,21,24,27,30,33,36,39,42,45,48\}$$

\item
  Show that if $a,b,m\in\integers$ with $m>0$ and if
  $a\equiv b\mbox{ mod } m$ then $\gcd(a,m)=\gcd(b,m)$. \\\\
  If $a\equiv b\mbox{ mod } m$ then $\exists x\in\integers$ such that $a=b+xm$.
  So $\gcd(a,m)=\gcd(b+xm,m)=\gcd(b,m)$.

\item
  Suppose $p$ is prime and $x\in\integers$ satisfies
  $x^2\equiv x\mbox{ mod } p$.  Prove that $x\equiv 0\mbox{ mod } p$ or $x\equiv 1\mbox{ mod } p$.
  Show with a counterexample that this fails if $p$ is not prime. \\\\
  Because $x^2\equiv x\mbox{ mod } p$ we know that $x^2-x\equiv 0\mbox{ mod } p$,
  which is the same as $x(x-1) \equiv 0\mbox{ mod } p$. This implies that
  either $p\mid x$, $p\mid (x-1)$, or both.
  $$p\mid x \implies x\equiv 0\mbox{ mod } p$$
  $$p\mid (x-1) \implies x\equiv 1\mbox{ mod } p$$
  If $p$ is not prime, say $p=6$ we see,
  $$3^2 \equiv 3 \mbox{ mod } 6$$
  Where $3\not\equiv 0\mbox{ mod }6$ and $3\not\equiv 1\mbox{ mod }6$. So $p$ must be prime for the
  statement to hold true.

\item
  Show that if $n$ is an odd positive integer or if $n$ is a positive
  integer divisible by $4$ that:
  $$1^3+2^3+...+(n-1)^3\equiv 0\mbox{ mod } n$$
  There are two cases to look at, when $n$ is an odd positive integer 
  and when $n$ is divisible by $4$.
  \begin{itemize}
	\item If $n$ is an odd positive integer then $n-1$ is even so we have an even amount of
	numbers. Consider the set $S = \{1^3, 2^3, \cdots, (n-1)^3\}$. Then consider
	two subsets of $S$, both with $(n-1)/2$ elements $S_1$ and $S_2$, where 
	$$\sum S = \sum S_1 + \sum S_2 = 1^3+ 2^3 + \cdots + (n-1)^3$$
	The set $S_1 = \{1^3,2^3,3^3,\cdots\}$ and the set $S_2 = \{\cdots, (n-3)^3, (n-2)^3, (n-1)^3\}$.
	Because we know that $a-b \equiv -b \mbox{ mod } a$ we also know that $(a-b)^3\equiv (-b)^3\mbox{ mod } a$.
	So we can say that for all elements in $S_2 \mbox{ mod } n$, $S_2 = \{\cdots, (-3)^3, (-2)^3, (-1)^n\}$.
	Now if we look at $\sum S_1 + \sum S_2 \mbox{ mod } n$ we see that the first element of $S_1$
	is cancelled out by the last element of $S_2$ and so forth until there are no elements left.
	Thus, $1^3+2^3+...+(n-1)^3\equiv 0\mbox{ mod } n$.

	\item If $n$ is divisble by 4 then $n-1$ is odd so we have an odd amount of numbers.
	Consider the set $S = \{1^3, 2^3, \cdots, (n-1)^3\}$. Then consider two subsets of $S$,
	both with $(n-1)/2 - 1$ elements $S_1$ and $S_2$, where
	$$\sum S = \sum S_1 + \left(\frac{n}{2}\right)^3 + \sum S_2 = 1^3+ 2^3 + \cdots + (n-1)^3$$
	The set $S_1 = \{1^3,2^3,3^3,\cdots\}$ and the set $S_2 = \{\cdots, (n-3)^3, (n-2)^3, (n-1)^3\}$.
	Because we know that $a-b \equiv -b \mbox{ mod } a$ we also know that $(a-b)^3\equiv (-b)^3\mbox{ mod } a$.
	So we can say that for all elements in $S_2 \mbox{ mod } n$, $S_2 = \{\cdots, (-3)^3, (-2)^3, (-1)^n\}$.
	Now if we look at $\sum S \mbox{ mod } n$, like in the case above we can see that sets $S_1$ and $S_2$ will
	cancel one another out. This leaves us with
	$$1^3+2^3+...+(n-1)^3\equiv \left(\frac{n}{2}\right)^3\mbox{ mod } n$$
	Because we know that $4\mid n$ we know that $n=4x$ for some $x\in\integers$. It follows that,
	$$\left(\frac{n}{2}\right)^3 = \frac{n^3}{8} = \frac{64x^3}{8} = 8x^3 = (2x^2)n$$
	So $$\left(\frac{n^3}{2}\right)^3 \equiv (2x^2)n \equiv 0\mbox{ mod } n$$
	Thus, $1^3+2^3+...+(n-1)^3\equiv 0\mbox{ mod } n$.
  \end{itemize}

\item
  Find all solutions (mod the given value) to each of the following.
  \begin{enumerate}
  \item $10x \equiv 25 \mbox{ mod } 75$\\\\
  Because the $\gcd(10,75)=5$ and $5\mid 25$ we know that solutions exist.
  Let $x_0 \equiv10\mbox{ mod } 75$, so all solutions are then
  $$x\equiv 10 + k \cdot \frac{75}{\gcd(10,75)}\mbox{ mod } 75,\text{ for } k=0,1,2,3,4$$
  $$x\equiv 10 + 15k\mbox{ mod } 75,\text{ for } k=0,1,2,3,4$$
  Therefore, $x\equiv 10, 25, 40, 55, 70$.

  
  \item $9x \equiv 8\mbox{ mod } 12$\\\\
  Because the $\gcd(9,12)=3$ and $3\nmid 8$ so there are no solutions.
  
  \end{enumerate}

\item
  Solve each of the following linear congruences using inverses.
  \begin{enumerate}
  \item $3x \equiv 5\mbox{ mod } 17$\\\\
  Since $6$ is the inverse of $3\mbox{ mod } 17$ we get, 
  $6\cdot3x\equiv 6\cdot 5\mbox{ mod } 17$ which implies
  $$x\equiv 30\mbox{ mod } 17\equiv 13\mbox{ mod } 17$$
  Therefore, $x\equiv 13$.

  \item $10x \equiv 3\mbox{ mod } 11$\\\\
  Since $10$ is the inverse of $10\mbox{ mod } 11$ we get,
  $10\cdot10x\equiv 10\cdot 3\mbox{ mod } 11$ which implies
  $$x\equiv 30\mbox{ mod } 11\equiv 8\mbox{ mod } 11$$
  Therefore, $x\equiv 8$.

  \end{enumerate}

\item
  What could the prime factorization of $m$ look like so that
  $6x\equiv 10\mbox{ mod } m$ has at least one solution?  Explain.\\\\
  In order for $ax\equiv b\mbox{ mod } m$ to have a solution(s),
  $\gcd(a,m)\mid b$. So in the context of this problem we have that
  $\gcd(6,m)\mid 10$. We are looking for an $m$ such that $\gcd(6,m)=2$.
  One possible $m$ could be $m=2^1$.

\item
  Use the Chinese Remainder Theorem to solve:
  \\A troop of monkeys has a store of bananas.
  When they arrange them into 7 piles, none remain.
  When they arrange them into 10 piles there are 3 left over.
  When they arrange them into 11 piles there are 2 left over.
  What is the smallest positive number of bananas they can have?
  What is the second smallest positive number?\\\\
  Let $x$ be the number of bananas, we have
  \begin{align*}
	x &\equiv 0\mbox{ mod } 7 \\
	x &\equiv 3\mbox{ mod } 10 \\
	x &\equiv 2\mbox{ mod } 11
  \end{align*}
  Test to see if all $m_i$ are pairwise coprime, $\gcd(7,10)=\gcd(7,11)=\gcd(10,11)=1$.
  This means that $M=770$, $M_1=110$, $M_2=77$, and $M_3=70$.
  \begin{itemize}
	\item[] Solve for $y_1$:
	  \begin{align*}
		110y_1 &\equiv 1\mbox{ mod } 7 \\
		5y_1 &\equiv 1\mbox{ mod } 7 \\
		y_1 &= 3
	  \end{align*}

	\item[] Solve for $y_2$:
	  \begin{align*}
		77y_2 &\equiv 1\mbox{ mod } 10 \\
		7y_2 &\equiv 1\mbox{ mod } 10 \\
		y_2 &= 3
	  \end{align*}
	
	\item[] Solve for $y_3$:
	  \begin{align*}
		70y_3 &\equiv 1\mbox{ mod } 11 \\
		4y_3 &\equiv 1\mbox{ mod } 11 \\
		y_3 &= 3
	  \end{align*}
  \end{itemize}
  So we then get 
  $$x = (0)(110)(3) + (3)(77)(3) + (2)(70)(3) \equiv 1113 \mbox{ mod } 770$$
  $$x \equiv 343 \mbox{ mod } 770$$
  The smallest number of bananas they can have is $343$ and the second smallest is $1113$.

\item
  Solve the system of linear congruences:
  \begin{align*}
	2x+1 & \equiv 3\mbox{ mod } 10 \\
	x+2 & \equiv 7\mbox{ mod } 9 \\
	4x & \equiv 1\mbox{ mod } 7
  \end{align*}
  Rewrite the system of linear congruences to be (properties of congruences):
  \begin{align*}
	2x &\equiv 2\mbox{ mod } 10 \\
	x &\equiv 5\mbox{ mod } 9 \\
	4x &\equiv 1\mbox{ mod } 7
  \end{align*}
  Which then becomes
  \begin{align*}
	x &\equiv 1\mbox{ mod } 5 \\
	x &\equiv 5\mbox{ mod } 9 \\
	x &\equiv 2\mbox{ mod } 7
  \end{align*}
  Then test to see if all $m_i$ are pairwise coprime, 
  $\gcd(5,9)=\gcd(5,7)=\gcd(7,9)=1$. This means that
  $M=315$, $M_1=63$, $M_2=35$, and $M_3=45$.
  \begin{itemize}
	\item[] Solve for $y_1$:
	  \begin{align*}
		63y_1 &\equiv 1\mbox{ mod }5 \\
		3y_1 &\equiv 1\mbox{ mod }5 \\
		y_1 &= 2
	  \end{align*}
	\item[] Solve for $y_2$:
	  \begin{align*}
		35y_2 &\equiv 1\mbox{ mod }9 \\
		8y_2 &\equiv 1\mbox{ mod }9 \\
		y_2 &= 8
	  \end{align*}
	\item[] Solve for $y_3$: 
	  \begin{align*}
		45y_3 &\equiv 1\mbox{ mod }7 \\
		3y_3 &\equiv 1\mbox{ mod }7 \\
		y_3 &=5
	  \end{align*}
  \end{itemize}
  So we then get
  $$x=(1)(63)(2) + (5)(35)(8) + (2)(45)(5) \equiv 1976 \mbox{ mod } 315$$
  $$x\equiv 86\mbox{ mod }315$$

\end{enumerate}


%%%%%%%%%%%%%%%
%%% Section %%%
%%%%%%%%%%%%%%%
%\subsection{Special Congruences}
%\hfill \fbox{/100}
%\rule{\textwidth}{1pt}

\end{document}