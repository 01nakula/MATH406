\documentclass[class=article, crop=false]{standalone}
\usepackage[utf8]{inputenc}
\usepackage{import}
\usepackage[subpreambles=true]{standalone}
\usepackage{graphicx}
\usepackage{amsfonts}
\usepackage{mathrsfs}
\usepackage{mathtools}
\usepackage{enumerate}
\usepackage{fancyhdr}
\usepackage[colorlinks=true,linkcolor=black,anchorcolor=black,citecolor=black,filecolor=black,menucolor=black,runcolor=black,urlcolor=blue]{hyperref}
\usepackage{epsfig,amssymb,amsmath,multicol,tikz,pgfplots,amsthm,enumerate}

\def\naturals{{\mathbb N}}
\def\reals{{\mathbb R}}
\def\complex{{\mathbb C}}
\def\poly{{\mathbb P}}
\def\integers{{\mathbb Z}}
\def\rationals{{\mathbb Q}}
\def\irrationals{{\mathbb I}}
\def\inlinesum#1#2{\overset{#2}{\underset{#1}{\sum}}}
\def\inlineprod#1#2{\overset{#2}{\underset{#1}{\prod}}}


\begin{document}
    
\section{Special Congruences}

%%%%%%%%%%%%%%%
%%% Section %%%
%%%%%%%%%%%%%%%
\subsection{Wilson's Theorem \& Fermat's Little Theorem}
\rule{\textwidth}{1pt}\\
\begin{enumerate}[1.]
	\item \textbf{Wilson's Theorem:}
	If $p$ is prime then $$(p-1)!\equiv -1\mod p$$
	\begin{proof}
		The case where $p=2$ is trivial to show, so let's look at primes $p\geq 3$.
		Consider the set of numbers $\underbrace{\{1,2,3,4,5,\cdots,p-1\}}_{\text{even number of integers}}$.
		Suppose $a$ is one of these, then $\exists b\in\integers$ such that $ab\equiv 1\mod p$ (a multiplicative inverse).
		Because the equation $ax\equiv 1\mod p$ has one solution because $\gcd(a,p)=1\mid1$.
		Note that $\gcd(a,p)=1$ because $a$ is one of $\{1,2,3,\cdots,p-1\}$.\\
		Could we have, for some $a\in\{1,2,3,\cdots, p-1\}$ that $a^2\equiv 1\mod p$?\\
		Suppose $a^2\equiv 1\mod p$, then $p\mid a^2-1$ so $p\mid (a+1)(a-1)$, either
		$p\mid (a+1)$ or $p\mid (a-1)$.
		If $p\mid (a+1)$ then $a\equiv -1\mod p$ or $a\equiv p-1\mod p$. If $p\mid(a-1)$
		then $a\equiv 1\mod p$.\\\\
		\textbf{Ex.} Suppose $p=11$, the set is $\{1,2,3,4,5,6,7,8,9,10\}$. Then the respective
		pairs would be $2\cdot6$, $3\cdot4$, $5\cdot9$, and $7\cdot8$. Notice that $1$ and $10$
		do not have a pair that results in congruence $\mod 11$.\\\\
		In general in $\{1,2,3,\cdots,p-1\}$ the integers all pair up such that their products
		are congruent $1\mod p$, except for $1$ and $p-1$. Thus,
		$$(p-1)!=(1)(2)(3)\cdots(p-1)\equiv p-1\equiv -1\mod p$$
	\end{proof}
	\noindent\textbf{Ex.} Find the least non-negative residue of $20!\mod 23$.\\
	Note: We see $20!$ and think $20!\equiv -1\mod 21$, but $21$ is not prime so there is
	no guarantee and it does not apply anyways because we have $\mod 23$.\\
	However, $22!\equiv -1\mod 23$
	\begin{align*}
		22! &\equiv -1\mod 23 \\
		(22)(21)(20!) &\equiv -1\mod 23 \\
		(-1)(-2)(20!) &\equiv -1\mod 23 \\
		(2)(20!) &\equiv -1\mod 23 \\
		(2)(20!) &\equiv 22\mod 23 \\
		20! &\equiv 11\mod 23
	\end{align*}
	In this case, $11$ is the least non-negative residue.
	\item \textbf{Fermat's Little Theorem:}
	Suppose $p$ is prime and $a\in\integers$ with $p\nmid a$. Then,
	$$a^{p-1} \equiv 1\mod p$$\\
	\textbf{Ex.} $p=97$ and $a=10$, so $10^{96}\equiv 1\mod 97$.
	\begin{proof}
		Consider the set of integers $S=\{a,2a,3a,\cdots,(p-1)a\}$ (there are $p-1$ integers in this set).
		\begin{itemize}
		\item First observe that none are congruent $0\mod p$ because if $p\mid ka$ for some $1\leq k\leq (p-1)$.
		Then $p\mid k$ or $p\mid a$ but $p\nmid a$ so $p\mid k$ but $1\leq k\leq p-1$.
		
		\item Second, no two are congruent one another $\mod p$ because if $k_1a\equiv k_2a\mod p$ for
		some $1\leq k_1\leq p-1$ and $1\leq k_2\leq p-1$. Then 
		$p\mid(k_1a - k_2a)=p\mid a(k_1-k_2)$, since $p\nmid a$ then $p\mid (k_1-k_2)$.
		But this is impossible because $1-(p-1)\leq k_1-k_2\leq (p-1)-1$.
		\end{itemize}
		Thus the set $S$, is we take all $\mod p$, is equivalent to the set $T=\{1,2,3,\cdots,p-1\}$
		in some order. Since, mod $p$, all the numbers in $S$ is congruent to all the numbers in $T$,
		we have
		\begin{align*}
			(a)(2a)(3a)\cdots((p-1)a) &\equiv (1)(2)(3)\cdots(p-1)\mod p \\
			a^{p-1}(p-1)! &\equiv (p-1)!\mod p \\
			a^{p-1}(-1) &\equiv (-1) \mod p \\
			a^{p-1} &\equiv 1\mod p
		\end{align*} 
		Notice that we can canel all of the $1,2,3,\cdots,p-1$ without affecting the modulus
		because they are coprime to $p$.
	\end{proof}
	\noindent\textbf{Ex.} Find the least non-negative residue of $5^{123}\mod 13$.\\
	Well $13\nmid 5$ so $5^{12}\equiv 1\mod 13$. Then $123=12(10)+3$ so
	\begin{align*}
		5^{123}=5^{12(10)+3}= 5^{12^{10}}5^3 &\equiv (1)^{10}5^{3}\mod 13 \\
		&\equiv 5^3 \mod 13 \\
		&\equiv 5\cdot 25 \mod 13 \\
		&\equiv 5(-1) \mod 13 \\
		&\equiv -5 \mod 13 \\
		&\equiv 8 \mod 13
	\end{align*}
	So $8$ is the least non-negative residue.\\\\
	\textbf{Corollary:} From $a^{p-1}\equiv 1\mod p$ we get $a^{p}\equiv a\mod p$.
	Note that $a^p \equiv a \mod p$ even when $p\mid a$ because if $p\mid a$ then 
	$a\equiv 0\mod p$ and $a^p \equiv a\mod p$ is saying $0\equiv 0\mod p$.

	\item \textbf{Closing Notes:}
	This is relevant to cryptography for one of two reasons.
	\begin{itemize} 
	\item Encryption (which involved big exponents) is both practical and theoretically possible
	based on Fermat's Little Theorem and Euler's Theorem.
	\item Pseudoprime is a non-prime which "behaves like a prime". e.g. in FLiT
	maybe $p$ is not prime but still when $p\nmid a$ we get $a^{p-1}\equiv 1\mod p$.
	\end{itemize}
\end{enumerate}

%%%%%%%%%%%%%%%
%%% Section %%%
%%%%%%%%%%%%%%%
\subsection{Fermat Pseudoprimes \& Carmichael Numbers}
\rule{\textwidth}{1pt}\\
\begin{enumerate}[1.]
	\item \textbf{Introduction:}
	Primes are useful. Given $n\in\integers^+$ how can we check if $n$ is prime?
	We could divide by everything (computationally intensive).
	Or we could use some tests which give insight.
	
	\item \textbf{Fermat Pseudoprimes:}
		\begin{enumerate}[(a)]
			\item \textbf{Reminder:}
				FLiT: If $p$ is prime and $p\nmid a$ then $a^{p-1}\equiv 1\mod p$.
				Suppose we have some $n\in\integers$ with $n\geq 2$. Suppose we find
				some $a$ with $n\nmid a$ and $a^{n-1}\not\equiv 1\mod n$. We can
				conclude that $n$ is not prime.\\
				\textbf{Ex:}
				Let $n=63$, observe that if $a=2$ then $n\nmid a$ clearly and
				$2^{62}\equiv 4\not\equiv 1\mod 63$. Thus, $63$ is not prime.\\
				\textbf{Definition:} $a=2$ is a \emph{Fermat Witness} to the fact that
				$63$ is composite.\\\\
				However, we might have some $n$ and $a$ with $n\nmid a$ and $a^{n-1}\equiv 1\mod n$
				but still have $n$ composite.\\
				\textbf{Ex.} Let $n=341$ and $a=2$, then $341\nmid 2$ and observe
				$$2^{340} \equiv 1\mod 341$$
				Even though $n=341=11\cdot 31$ is not prime it still "passes Fermat's
				Little Theorem with $a=2$."\\
				\textbf{Definition:} $a=2$ is a \emph{Fermat Liar} for $n=341$.

			\item \textbf{Definition:} 
				Suppose $n$ is composite and $b\in\integers$ satisfies
				$\gcd(n,n)=1$ and $b^{n-1}\equiv 1\mod n$. 
				Then we say $n$ is a \emph{Fermat Pseudoprime to the base $b$}.\\
				\textbf{Ex:} So $341$ is a \emph{Fermat Pseudoprime with the base $b=2$}.\\
				\textbf{Ex:} Likewise, $645$ is a \emph{Fermat Pseudoprime with the base $b=2$}.
		\end{enumerate}
	
	\item \textbf{Carmichael Numbers:}
		\begin{enumerate}[(a)]
			\item \textbf{Introduction:}
				Given some $n$ we wish to test if it is prime.
				\begin{itemize}
					\item Pick some $b$ with $\gcd(b,n)=1$.
						Suppose we find $b^{n-1} \equiv 1\mod n$.
						Either $n$ is prime or $b$ is a liar and $n$ is a Fermat
						Pseudoprime with base $b$.
					
					\item Try another $b$ with $\gcd(b,n)=1\cdots$
				\end{itemize}
				So, is it possible that we could try all $b$ with $\gcd(b,n)=1$
				and always get $b^{n-1}\equiv 1\mod n$ and still have a composite $n$?
				The answer, yes!
			
			\item \textbf{Definition:}
				A number $n$ is a \emph{Carmichael Number} if it is a Fermat Pseudoprime for
				every base $b$ with $\gcd(b,n)=1$. These are sometimes called Absolute Pseudoprimes.\\
				\textbf{Ex:} $n=561$ is a Carmichael Number. Note that $561=3\cdot 11\cdot 17$.
				Suppose $b$ satisfies $\gcd(b,561)=1$. Then
				\begin{itemize}
					\item $\gcd(b,3)=1$ so by FLiT $b^2 \equiv 1\mod 3$. So $b^{560}=(b^2)^{280}\equiv 1\mod 3$
						so $3\mid b^{560}-1$.
					\item $\gcd(b,11)=1$ so by FLiT $b^{10}\equiv 1\mod 11$. So
						$b^{560}=(b^{10})^{56}=(1)^{56}\equiv 1\mod 11$ so $11\mid b^{560}-1$.
					\item $\gcd(b,17)=1$ so by FLiT $b^{16}\equiv 1\mod 17$. So
						$b^{560}=(b^{16})^{35}\equiv(1)^{35}\equiv 1\mod 17$ so $17\mid b^{560}-1$.
				\end{itemize}
				So $3\cdot 11\cdot 17\mid b^{560}-1 \implies 561\mid b^{560}-1$. Therefore $b^{560}\equiv 1\mod 561$.

			\item \textbf{Theorem:}
				Suppose $n=p_1 p_2\cdots p_k$ such that $\forall i$ we have $p_i -1\mid n-1$.
				Then $n$ is a Carmichael Number.
				\begin{proof}
					Suppose $\gcd(b,n)=1$. Claim that $b^{n-1}\equiv 1\mod n$ well, for each $i$ we have
					$\gcd(b, p_i)=1$. By FLiT we have $b^{p_i -1} \equiv 1\mod p_i$ then
					$b^{n-1}=b^{\alpha(p_i-1)}\equiv (1)^{\alpha}\equiv 1\mod p_i$. Thus,
					$p_i\mid b^{n-1}-1$ for all $i$. Therefore, $n\mid b^{n-1}-1$ so 
					$b^{n-1}\equiv 1\mod n$.
				\end{proof}
				\noindent
		\end{enumerate}
\end{enumerate}

%%%%%%%%%%%%%%%
%%% Section %%%
%%%%%%%%%%%%%%%
\subsection{Euler's Theorem}
\rule{\textwidth}{1pt}\\
\begin{enumerate}[1.]
	\item \textbf{Introduction:}
	Fermat's Little Theorem tells us that is $p$ is a prime and if $p\nmid a$ then
	$a^{p-1}\equiv 1\mod p$. This is relevant for both calculation and cryptography.
	Since this is useful for reducing large powers of $a\mod p$
	it might be helpful if we had a version for when the modulus is not prime.

	\item \textbf{Preliminaries:}
	\begin{enumerate}[(a)]
		\item \textbf{Definition:} Define the \emph{Euler Phi-Function}
		$\phi: \integers^+ \rightarrow \integers$. For $n\in\integers^+$ we define
		$\phi(1)=1$ and $\phi(n)=$ the number of positive integers less than $n$ which are coprime
		to $n$.\\
		\textbf{Ex.} $\phi(10)=4$ because the set $\{1,3,7,9\}$ is all coprime to 10. \\
		\textbf{Ex.} $\phi(97)=96$ because $\{1,2,\cdots,96\}$ are all coprime to 96. \\
		\textbf{Definition:} If $n$ is prime then $\phi(n)=n-1$.

		\item \textbf{Recall:} A complete residue system mod $n$ is a set of $n$ integers,
		none of them congruent to each other mod $n$. CRS mod 8 is $\{0,1,2,\cdots,7\}$.
		
		
		\item \textbf{Definition:} A \emph{reduced residue system} mod $n$ is a set of
		$\phi(n)$ integers all of which are coprime to $n$ and no two of which are congruent to each
		other mod $n$.\\
		\textbf{Ex.} RRS mod 10 is $\{1,3,7,9\}$ or $\{11,-7,7,29\}$.
		
		\item \textbf{Theorem:} Suppose $\{r_1, r_2,\cdots r_{\phi(n)}\}$ is a RRS mod $n$.
		Then suppose $a\in\integers$ with $\gcd(a,n)=1$. Then
		$\{ar_1, ar_2, \cdots ar_{\phi(n)}\}$ is also a RRS mod $n$.
		\begin{proof}
			We see there are $\phi(n)$ of them. Claim that each is coprime to $n$.
			\begin{itemize}
			\item By means of contradiction, suppose we have some $ar_i$ not coprime to $n$, 
			that is $\gcd(ar_i,n)\neq 1$. Then $\exists $ a prime $p$ with
			$p\mid ar_i$ and $p\mid n$. Since $p\mid ar_i$ so $p\mid a$ or $p\mid r_i$.
			If $p\mid a$ then, along with $p\mid n$, we have a contradiction because
			$\gcd(a,n)=1$. If $p\mid r_i$ then, along with $p\mid n$, we have a contradiction
			because $\gcd(r_i,n)=1$. So the $ar_i$ are coprime to $n$.

			\item Suppose we have $ar_i\equiv ar_j\mbox{ mod }n$, since $\gcd(a,n)=1$ we can cancel.
			So $r_i \equiv r_j \mbox{ mod }n$. So no two new elements are congruent mod $n$.
			\end{itemize}
		\end{proof}
	\end{enumerate}

	\item \textbf{Euler's Theorem:} Suppose $n$ is a modulus and $\gcd(a,n)=1$. Then
	$a^{\phi(n)}\equiv 1\mbox{ mod }n$.\\
	\textbf{Note.} If $n=p=$ prime we have $\phi(n)=n-1$ and we get Fermat's Little Theorem.
	\begin{proof}
		Given a modulus $n$, let $S = \{r_1, \cdots, r_{\phi(n)}\}$ be any RRS.
		Then by the theorem above, $S'=\{ar_1, \cdots ar_{\phi(n)}\}$ is also a RRS.
		It follows that $S$ and $S'$ consist of the same integers mod $n$. Thus,
		\begin{align*}
			(ar_1)(ar_2)\cdots (ar_{\phi(n)}) &\equiv r_1 r_2\cdots r_{\phi(n)} \mbox{ mod }n \\
			a^{\phi(n)} &\equiv 1\mbox{ mod }n
		\end{align*}
	\end{proof}

	\item \textbf{Use For Calculation:}
	To reduce $9^{453} \mbox{ mod }16$, we note that $\gcd(9,16)=1$ so Euler's Theorem tells us that
	$9^{\phi(16)}\equiv 1\mbox{ mod }16$. Since $\phi(16)=8$ we ahve $9^8 \equiv 1 \mbox{ mod }9$ and so:
	$$9^{453} = 9^{8(56)+5} \equiv 9^5 \equiv 9(81)^2 \equiv 9\mbox{ mod }16$$

	\item \textbf{Note:} If $\gcd(a,n)=1$ then $a^{\phi(n)-1}$ is a multiplicative inverse of $a\mbox{ mod }n$. 
\end{enumerate}

%%%%%%%%%%%%%%%
%%% Section %%%
%%%%%%%%%%%%%%%
\subsection{Problems}
\rule{\textwidth}{1pt}\\
\begin{enumerate}
\item
  Use Fermat's Little Theorem
  to find the least nonnegative residue of
  $2^{1000003}\mod 17$.

\item
  Use Fermat's Little Theorem to solve the following,
  giving the result as the least nonnegative residue.
  \begin{enumerate}
  \item
	$7x\equiv 12\mod 17$
  \item
	$10x\equiv 13\mod 19$
\end{enumerate}

\item
  Use Fermat's Little Theorem
  to show that
  $30\big|(n^9-n)$
  for all positive integers $n$.

\item
  The definition of $n$ being a Fermat pseudoprime to base $b$
  does not actually require that $\gcd(b,n)=1$
  because it's not possible to have $b^{n-1}\equiv 1\mod n$ with $\gcd(b,n)\neq 1$.
  Prove this.

\item
  We didn't exclude even integers from the definition of a Fermat Pseudoprime.
  Some books do.
  Show that with our definition
  $4$ is a Fermat Pseudoprime to a certain base.

\item
  Prove that if $n$ is an odd Fermat Pseudoprime to some base
  then it must be so to an even number of bases.

\item
  Prove that 1105 is a Carmichael number.

\item
  Use Euler's Theorem to find the units digit of $7^{999999}$.

\item
  Solve each of the following using Euler's Theorem.
  Solutions should be least nonnegative residues.
\begin{enumerate}
\item
  $5x\equiv 3\mod 14$
\item
  $4x\equiv 7\mod 15$
\item
  $3x\equiv 5\mod 16$
\end{enumerate}

\item
  Prove that if $\text{gcd}(a,30)=1$ then $60\mid a^4+59$.


\end{enumerate}

\end{document}