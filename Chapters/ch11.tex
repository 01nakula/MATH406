\documentclass[class=article, crop=false]{standalone}
\usepackage[utf8]{inputenc}
\usepackage{import}
\usepackage[subpreambles=true]{standalone}
\usepackage{graphicx}
\usepackage{amsfonts}
\usepackage{mathrsfs}
\usepackage{mathtools}
\usepackage{enumerate}
\usepackage{fancyhdr}
\usepackage[colorlinks=true,linkcolor=black,anchorcolor=black,citecolor=black,filecolor=black,menucolor=black,runcolor=black,urlcolor=blue]{hyperref}
\usepackage{epsfig,amssymb,amsmath,multicol,tikz,pgfplots,amsthm,enumerate}

\def\naturals{{\mathbb N}}
\def\reals{{\mathbb R}}
\def\complex{{\mathbb C}}
\def\poly{{\mathbb P}}
\def\integers{{\mathbb Z}}
\def\rationals{{\mathbb Q}}
\def\irrationals{{\mathbb I}}
\def\inlinesum#1#2{\overset{#2}{\underset{#1}{\sum}}}
\def\inlineprod#1#2{\overset{#2}{\underset{#1}{\prod}}}
\def\ord{{\text{ord}}}
\def\ind{{\text{ind}}}


\begin{document}
    
\section{Quadratic Residues}
\textbf{Introduction:} The concept of Quadratic Residues is a fundamental tool
which has ramifications in lots of other number theory places: Cryptography, Factoring,
etc...
\subsection{Quadratice Residues \& Nonresidues}
\rule{\textwidth}{1pt}
\begin{enumerate}[1.]
\item \textbf{Introduction:} Suppose we asked the following, given a modulus $m$:
Which numbers are perfect squares mod $m$? \\\\
\textbf{Ex.} Let $m=7$. What are the perfect squares? We could of course work backwards,
squaring each value:
\begin{align*}
	0^2 &\equiv 0\mbox{ mod } 7 \\
	1^2 &\equiv 1\mbox{ mod } 7 \\
	2^2 &\equiv 4\mbox{ mod } 7 \\
	3^2 &\equiv 2\mbox{ mod } 7 \\
	4^2 &\equiv 2\mbox{ mod } 7 \\
	5^2 &\equiv 4\mbox{ mod } 7 \\
	6^2 &\equiv 1\mbox{ mod } 7
\end{align*}
Then the perfect squares are $0,1,2,4$ and $3,5,6$ are not.

\item \textbf{Quadratice Residues \& Nonresidues - Counting}
\begin{enumerate}[(a)]
	\item \textbf{Definition:} Let $m$ be a modulus and $a\in\integers$ with $\gcd(a,m)=1$.
	We say $a$ is a \textit{quadratic residue mod m} if $\exists x\in\integers$ such that 
	$x^2 \equiv a\mbox{ mod } m$. Otherwise, we say $a$ is a \textit{quadratic nonresidue mod m}
	if $\nexists x\in\integers$ such that $x^2 \equiv a\mbox{ mod } m$. \\\\
	\textbf{Ex.} If $m=7$ then QR:$1,2,4$, QNR:$3,5,6$, and Neither:$0$.
	
	\item \textbf{Theorem:} If $p$ is an odd prime and $a\in\integers$ with $p\nmid a \implies \gcd(p,a)=1$,
	then $x^2\equiv a\mbox{ mod } p$ has either no solutions or exactly two solutions mod $p$.
	\begin{proof}
		If there are none, we are done. Suppose $x$ is one solution to $x^2\equiv a\mbox{ mod }p$.
		Claim $-x$ is also a solution. Then $2x\equiv 0\mbox{ mod }p$.
		Since $p$ is odd we can do $x\equiv 0\mbox{ mod }p$ which implies $p\mid x\implies p\mid x^2$.
		Then, $x^2\equiv 0\mbox{ mod }p \implies a\equiv 0 \mbox{ mod }p$ which contradicts $p\nmid a$.\\\\
		Let's show that for any two solutions, they are negative of one another.
		Suppose $x_1^2 \equiv a\mbox{ mod }p$ and $x_2^2 \equiv a\mbox{ mod }p$.
		Then $x_1^2 - x_2^2 \equiv 0\mbox{ mod }p$ so $p\mid (x_1^2 - x_2^2)$ so
		$p\mid (x_1 - x_2) (x_1 + x_2)$ so $p\mid (x_1- x_2)$ or $p\mid (x_1 + x_2)$.\\
		If $p\mid (x_1 -x_2)$ then $x_1 \equiv x_2 \mbox{ mod }p$.
		If $p\mid (x_1 +x_2)$ then $x_1 \equiv -x_2 \mbox{ mod }p$.
		Thus, there can only be the two which are negatives of one another
	\end{proof}
	
	\item \textbf{Theorem:} Suppose $p$ is an odd prime. Then $\exists \frac{p-1}{2}$ QR and
	$\exists \frac{p-1}{2}$ QNR.
	\begin{proof}
		If we square all of $1,2,3,\cdots, p-1$ the results will be in pairs (two of every result)
		the $\frac{p-1}{2}$ we do get are the QR.
		We miss $\frac{p-1}{2}$ results, those are the QNR.
	\end{proof}

	\item \textbf{Theorem:} Let $p$ be an odd prime and $r$ a primitive root mod $p$.
	Suppose $p\nmid a$, then $a$ is a QR mod $p$ if and only if $\ind_r a$ is even.
	\begin{proof}
		$ $\\
		$\rightarrow$ Suppose $a$ is a quadratice residue mod $p$, $\exists x$ such that $x^2 \equiv a\mbox{ mod }p$.
		Then take the index of both sides to get $\ind_r x^2 \equiv \ind_r a\mbox{ mod }p-1$ and so
		$2\ind_r x \equiv \ind_r a\mbox{ mod }p-1$. From here we see $\ind_r a = 2\ind_r x + k(p-1)$ for some
		$k\in\integers$ and so since $p-1$ is even we know $\ind_r a$ is even. \\\\
		$\leftarrow$ Suppose $\ind_r a$ is even. Say $\ind_r a = 2k$ for $k\in\integers$ 
		so $r^{2k} \equiv a \mbox{ mod }p$ so $(r^k)^2\equiv a\mbox{ mod }p$. 
		Then, $a$ is a quadratice residue mod $p$.
	\end{proof}

	\textbf{To illustrate:} $r=3$ is a primitive root mod 17.
	\begin{center}
		\scalebox{.8}{
		\begin{tabular}{|c|c|c|c|c|c|c|c|c|c|c|c|c|c|c|c|c|}
			\hline
			$a\mbox{ mod }17$ & 1 & 2 & 3 & 4 & 5 & 6 & 7 & 8 & 9 & 10 & 11 & 12 & 13 & 14 & 15 & 16 \\
			\hline
			$\ind_3 a$ & \underline{16} & \underline{14} & 1 & \underline{12} & 5 & 15 & 11 & \underline{10} & \underline{2} & 3 & 7 & 13 & \underline{4} & 9 & \underline{6} & \underline{8} \\\hline
		\end{tabular}
		}
	\end{center}
	So what this theorem tells us is that $a=1,2,4,8,9,13,15,16$ are the quadratic residues

\end{enumerate}


\end{enumerate}

%%%%%%%%%%%%%%%
%%% Section %%%
%%%%%%%%%%%%%%%
%\subsection{Problems}
%\rule{\textwidth}{1pt}

\end{document}