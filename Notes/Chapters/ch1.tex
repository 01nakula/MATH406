\documentclass[class=article, crop=false]{standalone}
\usepackage[utf8]{inputenc}
\usepackage{import}
\usepackage[subpreambles=true]{standalone}
\usepackage{graphicx}
\usepackage{amsmath}
\usepackage{amsfonts}
\usepackage{mathrsfs}
\usepackage{mathtools}
\usepackage{enumerate}
\usepackage{fancyhdr}
\usepackage{amsthm}
\usepackage[colorlinks=true,linkcolor=black,anchorcolor=black,citecolor=black,filecolor=black,menucolor=black,runcolor=black,urlcolor=blue]{hyperref}

\DeclarePairedDelimiter\ceil{\lceil}{\rceil}
\DeclarePairedDelimiter\floor{\lfloor}{\rfloor}

\begin{document}
    
\section{The Integers}

\subsection{Numbers and Sequences}
This section will set the stage for what's to come. It is primarily about numbers.\\\\
Mostly we will be working with the \emph{integers} $\mathbb{Z} = \{\cdots, -3,-2,-1,0,1,2,3,\cdots\}$.
Additionally, we have the \emph{natural numbers} $\mathbb{N} = \{0,1,2,3,\cdots\}$ which are a subset of $\mathbb{Z}^+$.\\\\
\textbf{Definition.} We say a set of real numbers is \emph{well-ordered} if every non-empty subset has a smallest element.\\
\textbf{Ex.} $S = \{1,2,3,\cdots\}$ is well-ordered because every subset of $S$ has a least element.\\
\textbf{Ex.} $S = [0, \infty)$ is \emph{not} well-ordered because every subset does \emph{not} have a least element. 
Consider the subsets $(0,\infty)$, $(0,2)$, or $(1,5]$, none of them have least elements.\\\\
\textbf{Well-Ordering Principle.} $\mathbb{Z}^+$ is well-ordered. (This proof involves some serious set theory,
far beyond the scope of this course. See \href{https://math.berkeley.edu/~kpmann/Well-ordering.pdf}{this} as the proof.)\\\\
\textbf{Definition.} A real number is \emph{rational} if it can be expressed as $a/b$ where $a,b\in\mathbb{Z}$ and $b\neq 0$.
The set of all rational numbers is denoted as $\mathbb{Q}$.\\\\
\textbf{Ex.} Prove $\sqrt{2}$ is irrational (not rational).
\begin{proof}
	We need to prove that we cannot write $\sqrt{2}=\frac{a}{b}$ where $a,b\in\mathbb{Z}^+$ and $b\neq 0$.
	By way of contradiction, suppose $\sqrt{2}$ is rational. That is, suppose $$\sqrt{2} = \frac{a}{b}$$
	where $a,b\in\mathbb{Z}^+$ and $b\neq 0$. Then we have that $a=b\sqrt{2}$. Note that $b\in\mathbb{Z}^+$ and $b\sqrt{2}=a\in\mathbb{Z}^+$.\\\\
	Let $S = \{k \mid k\in\mathbb{Z}^+ \text{ and } k\sqrt{2}\in\mathbb{Z}^+\}$. Then $S \subset \mathbb{Z}^+$ and $S\neq \emptyset$ because
	$b\in S$. By the well-ordering principle, $S$ has a least element, denote it $m$. Consider $m' = m\sqrt{2}-m$.
	Observe the following:
	\begin{itemize}
		\item $m' = m\sqrt{2}-m = m(\sqrt{2}-1)$. Therefore $0 < m' < m$.
		\item Because $m\in S$ and $S\subset \mathbb{Z}^+$, $m, m\sqrt{2}\in\mathbb{Z}^+$. So $m'\in\mathbb{Z}^+$.
		\item Since $m \in \mathbb{Z}^+$ we have $2m\in\mathbb{Z}^+$, so now consider 
		$$m'\sqrt{2}=(m\sqrt{2}-m)\sqrt{2}=2m-m\sqrt{2} \in\mathbb{Z}^+$$
	\end{itemize}
	Thus, $m'\in S$, which contradicts the fact that $m$ is the least element in $S$. 
\end{proof}
\noindent\textbf{Definition.} A real number is \emph{algebraic} if it is the root of a polynomial with integer coefficients.\\
\textbf{Ex.} 
	\begin{itemize}
		\item Consider $x^3 + 3$. The roots are $x\pm \sqrt{3}$. So $\pm\sqrt{3}$ is algebraic.
		\item Is $7$ algebraic? Yes, $x-7$.
		\item Is $3/2$ algebraic? Yes, $3x-2$.
		\item Is $\sqrt[3]{2-\sqrt{7}}$ algebraic? Yes (although a bit more complicated)
			\begin{align*}
				x = \sqrt[3]{2-\sqrt[]{7}} &\implies x^3 = 2-\sqrt{7} \\
				&\implies x^3-2 = \sqrt{7} \\
				&\implies (x^3-2)^2 = 7 \\
				&\implies x^6 -4x^3+4=7 \\
				&\implies x^6 -4x^3-3=0
			\end{align*}
		\item Is $\pi$ algebraic? No! So what is it?
	\end{itemize}
\textbf{Definition.} A real number is not algebraic is \emph{transcendental} (it transcends the ability to be expressed as a root of a polynomial).
So $\pi$ is transcendental.\\
It is not difficult to prove the existence of transcendental numbers, but it is difficult to prove that any given number is transcendental.\\\\
\textbf{Definition.} Define $\floor{x}$ to be the largest integer $\leq x$. Similarly, define $\ceil{x}$ to be the smallest integer $\geq x$.\\
\textbf{Ex.}
	\begin{itemize}
		\item $\floor{5.2} = 5$
		\item $\floor{-3.8} = -4$
		\item $\ceil{5.2} = 6$
		\item $\ceil{-3.8} = -3$
	\end{itemize}
\textbf{Definition.} A set of numbers is \emph{countable} if it is either finite or it can be placed in one-to-one correspondance
with the positive integers.\\
\textbf{Ex.} The positive, even integers are countable, as are the integers and the rationals.\\
\textbf{Ex.} The real numbers are not countable. This is proved by \href{https://jlmartin.ku.edu/~jlmartin/courses/math410-S13/cantor.pdf}{Cantor's Argument}.\\\\
Consider all polynomials with integer coefficients. There are countably many of these, each having countably many roots.
Thus there are countably many algebraic numbers (the countable union of countable sets is countable). So out of $\mathbb{R}$,
which is uncountable, we must have uncountably transcendental numbers (because they are "everything else").
\end{document}