\documentclass[fullpage]{article}
\usepackage[utf8]{inputenc}
\usepackage{import}
\usepackage[subpreambles=true]{standalone}
\usepackage{graphicx}
\usepackage{amsmath}
\usepackage{amsfonts}
\usepackage{amssymb}
\usepackage{mathrsfs}
\usepackage{enumerate}
\usepackage{fancyhdr}
\usepackage[colorlinks=true,linkcolor=black,anchorcolor=black,citecolor=black,filecolor=black,menucolor=black,runcolor=black,urlcolor=blue]{hyperref}
\usepackage[margin=1.5in]{geometry}

\newcommand\blankpage{
    \null,
    \thispagestyle{empty},
    \addtocounter{page}{-1},
    \newpage}

\newif\ifanswers
\newif\ifintro

\answersfalse
\introfalse

\title{\textsc{Introduction To Number Theory}}
\author{Neelam Akula}
\date{}

\pagestyle{fancy}
\fancyhf{}
\rhead{Neelam Akula}
\lhead{MATH 406}
\cfoot{Page \thepage}

\begin{document}
\pagestyle{empty}
\maketitle
\pagebreak
{\null\vfill\small Last Updated: \today}
\pagebreak

%%%%%%%%%%%%%%%%%%%%%%%%%%%%
%%   Table of Contents    %%
%%%%%%%%%%%%%%%%%%%%%%%%%%%%
\setcounter{tocdepth}{2}
\tableofcontents
\newpage

%%%%%%%%%%%%%%%%%%%%%%%%%%%%
%%  Introduction Content  %%
%%%%%%%%%%%%%%%%%%%%%%%%%%%%
\ifintro
\section*{Introduction}
\rule{\textwidth}{1pt}
\vspace{1.5in}
\paragraph{}
    This is a compilation of notes and homeworks for MATH 406,
    Introduction to Number Theory, to aid both current and future students
    in fully understanding the material. 
    The primary text used is \emph{Elementary Number Theory}, by Kenneth H. Rosen, 6th
    Edition. While the text is not required it is an excellent resource for additional
    problems. Chapters covered from the text are 1, 3, 4, 6, 7, 9, 11, and 8 in that order. With two
    midterms following chapters 1, 3, 4 and chapters 6, 7, 9. The final is cumulative with an
    emphasis on chapters 8 and 11.
    Lastly, the course is taught by Dr. Justin Wyss-Gallifent, on his personal site there are
    brief versions of each section's lecture notes. A list of his notes can be found
    \href{https://www.math.umd.edu/~immortal/MATH406/}{here}.
\addcontentsline{toc}{section}{Introduction}
\newpage
\fi

%%%%%%%%%%%%%%%%%%%%%%%%%%%%
%% Import Chapter Content %%
%%%%%%%%%%%%%%%%%%%%%%%%%%%%
\pagestyle{plain}
\import{Chapters/}{ch1}
\newpage
\import{Chapters/}{ch3}
\newpage
\import{Chapters/}{ch4}
\newpage
\import{Chapters/}{ch6}
\newpage
\import{Chapters/}{ch7}
\newpage
\import{Chapters/}{ch9}
\newpage
\import{Chapters/}{ch11}
\newpage
\import{Chapters/}{ch8}
\newpage
\import{Chapters/}{p_exam}

%%%%%%%%%%%%%%%%%%%%%%%%%%%%
%%    Appendix Content    %%
%%%%%%%%%%%%%%%%%%%%%%%%%%%%
\appendix
\ifanswers
\newpage
\import{Chapters/}{answers}
\fi
%\newpage
%\import{Chapters/}{p_exam_a}
% \newpage
% \section{Symbols and Notation}
% \subsection*{Logic Notation}
% \begin{itemize}
%     \item $\exists$ - There exists at least one.
%     \item $\exists!$ - There exists one and one only.
%     \item $\nexists$ - There is no.
%     \item $\forall$ - For all.
%     \item $\neg$ - Logical not.
%     \item $\lor$ - Logical or.
%     \item $\land$ - Logical and.
%     \item $\implies$ - Implies.
%     \item $\iff$ - If and only if.
%     \item $\leftrightarrow$ - Equivalence.
% \end{itemize}
% \subsection*{Set Notation}
% \begin{itemize}
%     \item $\mathbb{N}$ - Set of natural numbers.
%     \item $\mathbb{Z}$ - Set of integers.
%     \item $\mathbb{Q}$ - Set of rational numbers.
%     \item $\mathbb{A}$ - Set of algebraic numbers.
%     \item $\mathbb{R}$ - Set of real numbers.
%     \item $\mathbb{C}$ - Set of complex numbers.
%     \item $\in$ - Is member of.
%     \item $\notin$ - Is not member of.
%     \item $\ni$ - Owns.
%     \item $\subset$ - Is proper subset of.
%     \item $\subseteq$ - Is subset of.
%     \item $\supset$ - Is proper superset of.
%     \item $\supseteq$ - Is superset of.
%     \item $\cup$ - Set union.
%     \item $\cap$ - Set intersection.
%     \item $\setminus$ - Set difference.
% \end{itemize}

\end{document}